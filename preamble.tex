% I may change the way this is done in a future version, 
%  but given that some people needed it, if you need a different degree title 
%  (e.g. Master of Science, Master in Science, Master of Arts, etc)
%  uncomment the following 3 lines and set as appropriate (this *has* to be before \maketitle)
% \makeatletter
% \renewcommand {\@degree@string} {Master of Things}
% \makeatother

%%%%%%%%%%%%%%%%%%%%%%%%%%%%%%%%%%%%%%
% this makes floats in float-only pages  go
% to top of the page rather than the usual
% vertical middle
% https://tex.stackexchange.com/questions/28556/how-to-place-a-float-at-the-top-of-a-floats-only-page
%%%%%%%%%%%%%%%%%%%%%%%%%%%%%%%%%%%%%%
\makeatletter
\setlength{\@fptop}{0pt}
\setlength{\@fpbot}{0pt plus 1fil}
\makeatother
%%%%%%%%%%%%%%%%%%%%%%%%%%%%%%%%%%%%%%

\title{The roles, practices and experiences of scientists at the climate and nature science-policy interface}
% \author{HHYY7}
\author{Isabel M J Sargent}
\department{Institute for Innovation and Public Purpose}
\raggedbottom
\maketitle

\newpage
\thispagestyle{plain}
In memory of my parents

Two lifetimes combining science and politics
%\izquote{If you had studied politics, you would know that science is often privileged [in policy making], but it’s just not as central to the discussion as someone who is ensconced in science might think}{Paul Cairney, 2017, in \cite{MontanaW2021}}
\makedeclaration

\begin{abstract} % 300 word limit
\CAN{} scientists wishing to engage with policy can study a body of advice on the roles and practices necessary for supplying our knowledge in response to the demands of policy. Should we delve deeper, we can learn that we would be submitting our knowledge to an interface, the \SPI, which is most recently conceptualised as expressing complex flows of information. Further still, we learn that \CAN{} science has been defined as \PNS{}, which demands transformational approaches to policymaking. Despite an ongoing and sometimes polarised debate about how scientists should engage with policymaking, there is little acknowledgement of the tension between the advice to scientists to align with a supply-demand model of policymaking and the complex and contested nature of the \CAN{} \SPI. Moreover, this debate has rarely included the voices of scientists who engage with policy regarding \CAN-related science, meaning that their experiences of navigating these tensions are largely unheard. This study advances the debate by uncovering the experiences scientists who have engaged at the \CAN{} \SPI, comparing their experiences to the advice literature. A behavioural lens is applied to transcripts from semi-structured interviews with scientists who have engaged at the \CAN{} \SPI. This reveals influences on scientists from three intersecting systems: \inte, \know{} and \scip. Scientists navigate these systems using roles and practices that are similar to those described in the advice literature. However, some scientists are experiencing considerable tensions in their roles. Further, a number of innovative practices are uncovered that help scientists mitigate these tensions. Finally, a range of experiences, common to many of the participants, are identified that demonstrate how the complex and contested nature of the \CAN{} \SPI{} interfere with their ability to easily engage with it.
\end{abstract}

%%%%%%%% commented out more of preamble for now
\iffalse
\begin{impactstatement}

	UCL theses now have to include an impact statement. \textit{(I think for REF reasons?)} The following text is the description from the guide linked from the formatting and submission website of what that involves. (Link to the guide: {\scriptsize \url{http://www.grad.ucl.ac.uk/essinfo/docs/Impact-Statement-Guidance-Notes-for-Research-Students-and-Supervisors.pdf}})

\begin{quote}
The statement should describe, in no more than 500 words, how the expertise, knowledge, analysis,
discovery or insight presented in your thesis could be put to a beneficial use. Consider benefits both
inside and outside academia and the ways in which these benefits could be brought about.

The benefits inside academia could be to the discipline and future scholarship, research methods or
methodology, the curriculum; they might be within your research area and potentially within other
research areas.

The benefits outside academia could occur to commercial activity, social enterprise, professional
practice, clinical use, public health, public policy design, public service delivery, laws, public
discourse, culture, the quality of the environment or quality of life.

The impact could occur locally, regionally, nationally or internationally, to individuals, communities or
organisations and could be immediate or occur incrementally, in the context of a broader field of
research, over many years, decades or longer.

Impact could be brought about through disseminating outputs (either in scholarly journals or
elsewhere such as specialist or mainstream media), education, public engagement, translational
research, commercial and social enterprise activity, engaging with public policy makers and public
service delivery practitioners, influencing ministers, collaborating with academics and non-academics
etc.

Further information including a searchable list of hundreds of examples of UCL impact outside of
academia please see \url{https://www.ucl.ac.uk/impact/}. For thousands more examples, please see
\url{http://results.ref.ac.uk/Results/SelectUoa}.
\end{quote}
\end{impactstatement}
\fi
\begin{acknowledgements}
I have been humbled by the willingness of participants to share their experiences and wealth of insights for this study. I have learned a great deal more than I ever anticipated and it is wonderful to know that so much expertise, energy and commitment is being devoted to climate and nature science and policy. 

Thank you to my supervisor, Carolina Alves, for belief and needle-sharp observations on my draft manuscripts and to Kris De Meyer for helping me hone in on this approach to understanding influences on scientists. Thank you also to Giovanni Tagliani, Sergey Astakhov, Matthew Billings and Paula Engelbrecht for advice on interviewing and transcript labelling. I am also extremely grateful to Matt Ryan, Rafa Mestre, John Boswell, James Wilsdon and Aaron Thierry for their early pointers.

Thank you to Alice Cowell and Dean Bennett for your regular, no-obligation, check-ins, which kept me connected to the world, and to my Aunt, Felicia Taylerson, for warming dinners and hot water bottles that allowed me to disconnect from the world.

Thank you to the 2+ cohorts of IIPP MPA students who have taught me so much about their worlds -- it has been the greatest of privileges -- and to the MPA ``part-timers'', especially Holly, Cath, Alex, Jaspal and Catherine, who eased my long commutes with accommodation and easy company.

Thank you also to the students and staff at ECS, University of Southampton, for welcoming me into the lab, especially Jonathon Hare for suggesting I complete this thesis there, it has been a huge blessing. %and TimBL's pianola for being silent company during each interview.

Most importantly, the biggest thank you to my family, Gary Llewellyn, Rowan Sargent Llewellyn and Spoon for tolerating my deep focus and deeper tiredness and showing their understanding in chocolate, shoulder-rubs and washing up.
\end{acknowledgements}


\setcounter{tocdepth}{2} 
% Setting this higher means you get contents entries for
%  more minor section headers.

\tableofcontents
\listoffigures
\listoftables
%
%\printnoidxglossaries[title=List of Abbreviations,type=\acronymtype]
\printglossary[title=List of Abbreviations,type=\acronymtype]
%\printglossary[title=List of Abbreviations,type=\glsxtrabbrvtype ]
