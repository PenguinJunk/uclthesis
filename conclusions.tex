\chapter{Conclusions}\label{ch:conclusions}

In many ways, the conclusions of this work exactly parallel existing literature regarding how scientists should interface with policy regarding timing, framing, ....

Found the there are some very real frustrations of some scientists but others have been able to provide more insight into the nature of these

\begin{itemize}
    \item the sense that its who you know is down to needing trusted advisers when the pace of policy speeds up 
    
\end{itemize}

It is starkly clear that simply providing information to anyone that we are hoping to influence is insufficient. The deficit model is as unapplicable to policymakers and politicians as it is to the general public. It is also therefore not applicable to scientists, many of whom have been frustrated by our inability to influence policy in matters that we care deeply and passionately about. Thus, simply \emph{telling} scientists how we should behave is likely to be insufficient. In many cases, scientists are not reaching for the policy science literature anyway. Instead, we need to find means to influence scientists ... It is hoped that by providing descriptive, real-world experiences, a different type of knowledge is conveyed to anyone interested in influencing CAN policy. 

There were a great many contradictory perspectives arising from these interviews. Whilst some may be genuinely in conflict, it is likely that many are a consequence of very different perspectives of the science-policy interface that arise from the ways that different scientific disciplines interact with different aspects of CAN policy. What is also worth lifting from this observation is that, like policymakers and policy, scientists and their science are not homogeneous.   


building trust - for when it matters
deficit model - applied to policy and applied to scientists
policy's unknown unknowns - what if policy isn't asking the right question?
