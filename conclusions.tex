\chapter{Conclusions}\label{ch:conclusions}

This research complements the existing theory and advice for scientists by studying the roles, practices and experiences of scientists who engage with policy. For \CAN{} scientists wishing to help close the gap between their science and \CAN{} policy, there is a large body of advice on how to engage at the \SPI. However, this advice is based on a rational supply-demand model of policymaking, instructing scientists to align to the needs of policymaking and policymakers. Conceptualisations of the \SPI{} as complex and dynamic, and of policymaking for \CAN{} crises -- as \PNS{} -- attest to a need for \CAN{} policymaking to be fundamentally transformed. Further, maturing understandings of behaviour change find that individuals -- policymakers and scientists alike -- are influenced by more than the provision of information, be it scientific findings or advice on how to engage. A better understanding is needed about the influences on scientists who engage with \CAN{} \SPI and how well, or otherwise, their experiences fit with the existing theory about how scientists should engage with policy.

A phenomenological descriptive-analytic approach was employed, using semi-structure interviews and behavioural frameworks to identify influential factors. Three interlocking systems of influence on scientists were identified: \skiinte, \skiknow{} and \skiscip. Further, whilst the theory and advice to scientists was largely born out in the experiences of scientists, this study has found that there were a many tensions experienced in roles at the \SPI, and that scientists used strategic practices that suggested innovative responses to these tensions, to the non-linearities of the \SPI{} and also the urgency of the \CAN{} crises. Five potential directions are identified to support better engagement at the \CAN{} \SPI, namely allyship across and beyond \CAN{} science and policy, strengthening of the practice of citation within policy, science training for potential policymakers, specific policy training for scientists, and further exploration of the experiences of those engaging with policy. 

Finally, it is clear that simply providing information to anyone that we are hoping to influence rarely has the impact we hope it will. The \IDM{} is as inapplicable to policymakers as it is to the general public. It is also therefore not applicable to scientists, who are frustrated by our inability to influence policy in matters that we care deeply and passionately about. Thus, simply \emph{telling} scientists how we should behave is insufficient. In many cases, scientists are not reading Policy Studies literature anyway. Instead, it is hoped that this descriptive-analytic account of real-world experiences can support improvements to policymaking in a field of knowledge that is critical to all of humanity. 