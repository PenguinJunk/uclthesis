\chapter{Introduction}\label{ch:intro}

\info{\# - indicates text to be completed later}
%\info{Overview of the paper: Problem / Challenge/issue / How analysed/solved / Transition to background}
There is wide acknowledgement and a broad literature reporting a widening gap between science and its relevant policy (e.g. \cite{Nau2009,EdlerKB2022}). Possibly the most stark example of this gap is between climate and nature science and policy; despite decades of scientists' warnings of catastrophic outcomes of climate change and biodiversity loss the gap, measured, for example, as the difference between levels of atmospheric greenhouse gases (GHGs) and policy to reduce those GHGs\footnote{Measures of GHG volumes are useful for monitoring human impact on climate but, as \textcite{MorenoSF2016} argue, this is a narrow measure of our impact on the natural world and risks perpetuating and exacerbating societal and global injustices.}, continues to widen (\cite{StoddardEtAl2021,IPBES2022,IPCC2023}). This gap between settled science and policy ambition is wide enough to threaten the survival of ecosystems (\cite{DiazEtAl2019,IPBES2022}), global security (\cite{WEF2024}), human civilisation (\cite{TschakertEAKO2019}) and potentially the future of humanity (\cite{McKayEtAl2022}). With so much at stake, it is essential that all parties at the interface between climate and nature (CAN) science and policy are effective in their roles. Yet, a debate continues about the roles for scientists at the interface with policy, particularly given the `wicked' nature of the issues and their possible solutions (\cite{Cairney2016}, p94). This thesis contributes to that debate by studying the roles that scientists already play at the interface with policy. It discovers the behaviours that scientists use to ease the flow of knowledge into policy, as well as the contexts (such as the field of science and nature of the policy) within which they have engaged. In so doing, it provides descriptive insights to a largely normative literature about scientists' roles in policymaking.

%\info{Background - Origins of problem - Economic, social, political, and/or technical factors at play - What previous analyses tell us about problem - transition to current}
The traditional ``linear'' model of policy-making is that ``knowledge shapes policy'' that is, knowledge is produced by science and this informs policy (\cite{Pielke2007}, p12-3; \cite{BoswellS2017}). In reality, policymaking can be much messier as a consequence of the complexity of policy problems (\cite{Cairney2016}), the contrasting ``cultures'' of science and policy (\cite{Dale2002,Obermeister2022}), and the competing influences of political and commercial interests (\cite{StoddardEtAl2021}). Recognising this messier reality, and the frustrations of scientists and policymakers at the interface, \textcite{Pielke2007} defined idealised roles for scientists at the interface, which have been further developed over recent years (e.g. \cite{RapleyD2014,GregoryBW2024}). Others, having studied the needs of policy the interface, have made recommendation for scientists, including improving their communication, their knowledge of policy processes, the relevance of their research, and the synthesis of evidence (\cite{KennyRHTB2017,LubchencoR2020,GluckmanBK2021,Bisbal2024}). To remain credible, scientists are also held to high standards of transparency, integrity and legitimacy (\cite{KennyRHTB2017,ColognaKMBMO2024,GregoryBW2024}). There is great emphasis on the actions and roles that scientists \emph{should} assume, which neglects to understand the actions and roles that scientists already \emph{do} assume based on their real-world experiences at the interface.

\info{Challenge and Issues: - Current challenge/issue being addressed - Why different from background - What makes this challenge/issue analysis different - Transition from what contributes to analysis to how conducted}
Scientists already engage at the CAN policy interface and yet little work exists to understand what are their experiences (\cite{KennyRHTB2017}) and what they have learned (with the exception of \cite{Obermeister2022}). The roles played by scientists when engaging with policy are tempered by the context within which they engage (\cite{EdlerKB2022}). Even with CAN-related policy, this context is mutlifaceted, involving a range of scientific fields, approaches and outcomes; policy issues and policy settings; and wider factors such as political and commercial perspectives. However, current analyses tends to assume homogeneity across all science-policy contexts. It is perhaps unreasonable to prescribe the actions and roles that scientists should assume without first understanding what are their experiences of the CAN policy interface. Moreover, there is potentially a great deal to learn from scientists who have engaged with policy both from their perceived successes and their frustrations. Therefore, this thesis takes a step back from current works that place the onus of policy impact and influence on scientists. Instead, it meets scientists where they are at the CAN policy interface, learning directly from their real-world experiences, to understand the roles they have played and the contexts within which they engaged. It then uses the insights of scientists themselves to understand how the CAN interface can be enhanced by scientists and policy makers\unsure{hopefully not too ambitious}.

%\info{How Analysis Was Conducted and Summary of Results - How was the analysis conducted (integration of framework, cases, and/or conceptual framework) - Summarize interesting points in the analysis - Summarize conclusions - Transition to how we got here (Roadmap)}

In this thesis, I reflect on the factors that influence the behaviours of scientists when they have engaged with CAN policy, using the Individual-Social-Material framework of \textcite{DarntonH2013} applied to \# semi-structured interviews. I consider how these behaviours match with the roles prescribed for scientists at the policy interface and I further contextualise these engagments by inductively identifying the different characteristics of the interface that scientists encounter. I find \#\emph{some interesting points}\# and draw \#\emph{some conclusions}\#. 

This study brings some balance to the discussion of roles at the policy interface by spotlighting the experience and learning of scientists in a particularly challenging arena - that of CAN policy. In so doing it offers \#advice\#/\#reflection\#/\#solace\#/\# to scientists and \#insights\#/\#recommendations\#/\#appeals\# to policymakers that are intended to support the shared aim of science and policy to [improve] how knowledge flows through the CAN policy interface.

\#\emph{paragraph overviewing each chapter}\#

