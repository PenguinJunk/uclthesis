\chapter{Introduction}\label{ch:intro}

%2 Existing literature: Provide a very brief overview of the different views on the topic, i.e., causes, consequences, proposed solutions, country/case study focus. Highlight the main gap or weakness in the literature
Scientists, faced with the stark prospects for humanity that their research reveals may feel an imperative to engage with policy, yet may not have the confidence or skills to do so effectively (\cite{BednarekSHG2015,KennyRHTB2017,KEU2021perceptions}). A body of literature provide advice to scientists on how to engage (\cite{OliverC2019}). This literature describes how scientists can align the \emph{supply} of their knowledge to the \emph{demands} of policy. Yet the evidence is that scientists are not accessing this literature (\cite{CairneyTS2023}), despite experiencing difficulties when engaging with policy (\cite{Stirling2010,Obermeister2022,Gerber2023,Hicks2024}).

%1 Motivation: Describe briefly the big social challenge you will address and why it is of great importance (Tip: here you can cite academic sources and non-academic sources to demonstrate that this topic is important for a wider audience and not just an academic exercise)
How scientists should engage with public policy has been an ongoing, and sometimes polarised, debate for many years (e.g. \cite{Lackey2004,Nau2009,Stirling2010,Milman2013,Tyler2013,Oreskes2020,GluckmanBK2021,GregoryBW2024,Bisbal2024,Hicks2024}). %Much of this debate takes a perspective that science supplies knowledge to policy, or policy demands knowledge  from science (e.g. \cite{McNie2007,KennyRHTB2017,Castree2019}).   
Such an academic debate can seem incongruous when policymaking itself is being challenged to transform to address the worsening \CAN{} crises (\cite{DiazEtAl2019,LaybournTS2023,VerfuerthDCWP2023,Hewitt2024,GuptaEtAl2024}). Further, the debate has rarely included the voices of scientists who engage with policy regarding \CAN-related science. Therefore not only is the normative debate about how scientists should engage missing a descriptive counterpart, but the specific tensions for scientists that arise around \CAN{} policy are under-reported. This study advances the debate by comparing, to the literature, the experiences of scientists who have engaged with \CAN{} policy and revealing how current policymaking creates tensions for some scientists as they encounter the \CAN{} \SPI.

%3 Your lens: Describe your analytical approach or theoretical lens. For example, if the dominant literature has used explanations based on geography to explain resource conflicts, an alternative lens would be to look at colonial history.
This study combines two common and related conceptualisations of \CAN{} science. The first is that \CAN{} science is ``post-normal'' (\cite{FuntowiczR1993,Gluckman2014}), that is that ``facts are uncertain, values in dispute, stakes high, and decisions urgent'' (\cite[p649]{Ravetz1999}). The second is that the relationship of \CAN{} science to \CAN{} policy is complex and dynamic (\cite{StrassheimK2014,BoswellS2017}), such that knowledge flows in many directions not just from science into policy. This framing thus challenges the advice to scientists to align to a linear \emph{supply}-\emph{demand} policy, inviting questions about the real-world experience at the complex and contentious \CAN{} \SPI. These questions are answered by taking a behavioural lens on the described accounts of scientists at the \CAN{} \SPI. As such, this study complements the largely normative-prescriptive advice literature with a descriptive-analytic study of the real-world experiences of scientists who engage with \CAN{} science-policy.

%4 Your findings and contribution: What using this lens has helped us to understand about the issue that we would not have seen before. Provide a brief overview of your key findings and how these relate to existing studies
This qualitative behavioural analysis of transcripts semi-structured interviews with scientists engaging at the \CAN{} \SPI{} reveals factors that influence scientists from three intersecting systems: \inte, \know{} and \scip. Whilst many of the roles and practices described in the advice literature bear out \emph{in vivo}, some scientists are experiencing considerable tensions in their roles. Further, a number of practices are uncovered that indicate that scientists are innovating how they mitigate and exploit specific conditions. 

%5 Thesis structure outline: Provide a very brief outline of the next sections (e.g. “Section two provides a review of the relevant literature on…”)
This thesis is outlined as follows: Section~\ref{ch:lit} reviews the relevant literature; Section~\ref{ch:methods} describes the motivation and method of this study; Section~\ref{ch:results} report on the results of the behavioural analysis to extract a range of influencing factors; Section~\ref{ch:discussion} synthesises the results in comparison to existing literature and suggests some potential directions for further work; Section~\ref{ch:conclusions} summarises and concludes the study.

%\info{Overview of the paper: Problem / Challenge/issue / How analysed/solved / Transition to background}
%There is wide acknowledgement and a broad literature reporting a widening gap between science and its relevant policy (e.g. \cite{Nau2009,EdlerKB2022}\improvement{I have better refs for this}). Possibly the most stark example of this gap is between climate and nature (CAN) science and policy; Despite decades of scientists' warnings of catastrophic outcomes of climate change and biodiversity loss, the gap - for example, the difference between levels of atmospheric greenhouse gases (GHGs) and policy to reduce those GHGs\footnote{Measures of GHG volumes are useful for monitoring human impact on climate but, as \textcite{MorenoSF2016} argue, this is a narrow measure of our impact on the natural world and risks perpetuating and exacerbating societal and global injustices.} - continues to widen (\cite{StoddardEtAl2021,IPBES2022,IPCC2023}). This gap between settled science and policy ambition is wide enough to threaten the survival of ecosystems (\cite{DiazEtAl2019,IPBES2022}), global security (\cite{WEF2024}), human civilisation (\cite{TschakertEAKO2019}) and, potentially, the future of humanity (\cite{McKayEtAl2022}). With so much at stake, it is essential that all parties at the CAN policy interface are effective in their roles. Thus, scholars have developed theory around the roles for scientists at the interface with policy, and recommended how they may improve their engagement. These theories are founded studies of the needs of policy which consider how knowledge flows from science into decision-making and where these knowledge flows may be interrupted. 

%\info{Background - Origins of problem - Economic, social, political, and/or technical factors at play - What previous analyses tell us about problem - transition to current}
%The traditional ``linear'' model of policy-making is that ``knowledge shapes policy'' that is, knowledge is produced by science and this informs policy (\cite{Pielke2007}, p12-3; \cite{BoswellS2017}). In reality, policy-making can be much messier as a consequence of the complexity of policy problems (\cite{Cairney2016}), the contrasting ``cultures'' of science and policy (\cite{Dale2002,Obermeister2022}), and the competing influences of political and commercial interests (\cite{StoddardEtAl2021}). Recognising this messier reality, and the frustrations of scientists and policymakers at the interface, \textcite{Pielke2007} defined idealised roles for scientists at the interface, which have been further developed over recent years (e.g. \cite{RapleyD2014,GluckmanBK2021,GregoryBW2024}). Others, having studied the needs of policy the interface, have made recommendations for scientists, including improving their communication, their knowledge of policy processes, the relevance of their research, and the synthesis of evidence (\cite{KennyRHTB2017,LubchencoR2020,GluckmanBK2021,Bisbal2024}). To remain credible\improvement{Cash et al. 2003 CRELE is relevant here}, scientists are also held to high standards of transparency, integrity and legitimacy (\cite{KennyRHTB2017,ColognaKMBMO2024,GregoryBW2024}). There is great emphasis on the actions and roles that scientists \emph{should} assume, but this neglects to understand the actions and roles that scientists already \emph{do} assume and their real-world experiences at the CAN policy interface.

%\info{Challenge and Issues: - Current challenge/issue being addressed - Why different from background - What makes this challenge/issue analysis different - Transition from what contributes to analysis to how conducted}
%Little work exists to understand what are the experiences of scientists engaged in policy (\cite{KennyRHTB2017}) or what they have learned (with the exception of \cite{Obermeister2022}). The roles played by scientists when engaging with policy are tempered by the contexts, such as the field of science and nature of the policy, within which they engage (\cite{EdlerKB2022}). With CAN-related policy, this context is mutlifaceted, involving a range of scientific fields, approaches and outcomes; policy issues and policy settings; and wider factors such as political and commercial perspectives. However, current analyses tend to assume homogeneity across all science-policy contexts. It is perhaps unreasonable to prescribe the actions and roles that scientists should assume without first understanding what are their experiences of the CAN policy interface. Moreover, there is potentially a great deal to learn from scientists who have engaged with policy both from their perceived successes and their frustrations. Therefore, this thesis takes a step back from current works that place the onus of policy impact and influence on scientists. Instead, it uses the insights of scientists themselves to understand how the CAN interface can be enhanced by scientists and by policymakers\unsure{hopefully not too ambitious}.

%\info{How Analysis Was Conducted and Summary of Results - How was the analysis conducted (integration of framework, cases, and/or conceptual framework) - Summarize interesting points in the analysis - Summarize conclusions - Transition to how we got here (Roadmap)}
%In this study, scientists are interviewed about their real-world experiences at the CAN policy interface to discover the actions they take to ease the flow of knowledge into policy. Using the Individual-Social-Material framework of \textcite{DarntonH2013}, it reflects on the factors that influence the behaviours of scientists when they have engaged with CAN policy. It further considers how these behaviours match with the roles prescribed for scientists at the policy interface and inductively contextualises these engagements to understand how and why roles and experiences differ from from prescribed roles and from each other. This study finds [\#\info{\# - indicates text to be completed later}\emph{some interesting points}\#] and draws [\#\emph{some conclusions}\#]. 

%This study brings some balance to the discussion of roles at the policy interface by spotlighting the real-world experience and learning of scientists in a particularly challenging arena - that of CAN policy. It provides descriptive insights to a largely normative literature about scientists' roles in policy-making. In so doing it offers [\#advice\# / \#reflection\# / \#solace\#] to scientists and [\#insights\# / \#recommendations\# / \#appeals\#] to policymakers, which are intended to support the shared aim of science and policy to enhance how knowledge flows through the CAN policy interface.


%from \cite{BA2024trust} ``Policymakers refers to those directly involved in formulating policy. Within this group,  we can distinguish between elected representatives (in the legislature and in government), whom we refer to as ‘politicians’; and civil servants and senior advisors (scientific or otherwise), whom we refer to as ‘officials’. Each of these groupings will use evidence, but they may invoke  it or consider it at different stages or in different ways, and so attention to this nuance in who  is using evidence in the science-for-policymaking system, and when, is important. We try  to carefully distinguish which type of policymaker we are referring to throughout this report.''
