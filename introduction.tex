\chapter{Introduction}\label{ch:intro}

\info{Overview of the paper: Problem / Challenge/issue / How analysed/solved / Transition to background}
Scientific expertise is essential for informing public policy and yet there is wide acknowledgement and a broad literature reporting a widening gap between science and its relevant policy (e.g. \cite{Nau2009,EdlerKB2022}). Climate and nature science and policy is probably the most stark example of such a gap, whereby scientists have been warning of catastrophic outcomes of climate change and biodiversity loss for decades and yet the gap, measured, for example, as the difference between levels of atmospheric greenhouse gases (GHGs) and policy to reduce those GHGs, continues to widen (\cite{StoddardEtAl2021,IPBES2022,IPCC2023})\footnote{Measures of GHG volumes are useful for monitoring human impact on climate but, as \textcite{MorenoSF2016} argue, this is a narrow measure of our impact on the natural world and risks perpetuating and exacerbating the societal and global injustices.}. This gap between the settled science and the policy ambition is wide enough to threaten the survival of ecosystems (\cite{DiazEtAl2019,IPBES2022}), global security (\cite{WEF2024}), human cultures (\cite{TschakertEAKO2019}) and potentially the future of humanity (\cite{McKayEtAl2022}). With so much at stake, it is essential that all parties at the interface between science and policy are effective in their roles. Yet, a debate continues about the roles for scientists at the interface with policy, particularly given the `wicked' nature of the issues and their possible solutions (\cite{Cairney2016}, p94). This thesis contributes to that debate by studying the roles that scientists already play at the interface with policy. It discovers the behaviours that scientists use to ease the flow of knowledge into policy, as well as the contexts (such as the field of science and nature of the policy) within which they have engaged. It builds on existing work in the field of policy studies, which theorises about roles played at the interface, by gaining behavioural and contextual insights from the scientists themselves.

\info{Background - Origins of problem - Economic, social, political, and/or technical factors at play}
The traditional ``linear'' model of policy-making is that ``knowledge shapes policy'' that is, knowledge is produced by science and this informs policy (\cite{Pielke2007}, p12-3; \cite{BoswellS2017}). In reality, policymaking can be much messier as a consequence of the complexity of policy problems (\cite{Cairney2016}), the contrasting ``cultures'' of science and policy (\cite{Dale2002,Obermeister2022}), and the competing influences of political and commercial interests (\cite{StoddardEtAl2021}). Recognising this messier reality, and the frustrations of scientists and policymakers at the interface, \textcite{Pielke2007} defined idealised roles for scientists at the interface, which have been developed to by others (e.g. \cite{RapleyD2014,GregoryBW2024}). It is vital that to have influence, scientists must choose roles that maintain their credibility (\cite{ColognaKMBMO2024,GregoryBW2024}). Yet, policymakers are also influenced by actors playing issue and policy and advocacy roles that may have much less legitimacy, accountability and transparency (\cite{Kingdon1993,Knaggard2015,Cairney2018,vonMalmborg2024strategies}), potentially putting scientists at an influential disadvantage. 


\info{Background - What previous analyses tell us about problem}

Focus on the roles of scientists in conveying knowledge into policy have suggested that scientists consider the clarity of their communication [e.g. using language that is not understood or understood differently outside of their field, refs SomervilleH2011], and to support decision makers by providing possible options [e.g. stating only the issues, not the possible means of addressing them, refs]. There is a conflict between these theoretical approaches to the communication of knowledge and the culture [e.g. objectivity, refs] and some works have attempted to describe how scientists can communicate their knowledge to better influence citizens and decision-makers [refs]. There is evidence that scientists themselves are reframing their knowledge under different contexts, perhaps with the aim of better influencing [self-censorship refs].

scientists instead turn to activism \cite{Pivovarchuk2024,GregoryBW2024}

scientist learns to answer the questions that policy asks \cite{Gerber2023}

Of course, the policy maker plays their role be selecting the knowledge to be used. W  Decisions makers can choose to ignore evidence (e.g. \cite{TennoyHLN2016}), rarely needing to justify what they include and what they leave out\footnote{not the case with EU Climate Change Advisory Board \cite{WardmanE2023}}.

\cite{EdlerKB2022} - `` critique the current literature's emphasis on the efforts of scientists to generate policy impact, because it neglects the role of `user' policymaking organisations''

Also, less well documented\unsure{is it?} are the power influences at play, that policy, even policy based on scientific understanding, such as the nature of climate change, is [pertinent to] other factors such as commercial interests, concerns about social acceptance of transition [activities] [etc]. [supranational nature of climate governance, colonialism and imperialism in geopolitics, power relations in national governance and decision-making, lobbying by vested interests, dominance of economics and financial interests in governance, etc.]

[Thus, to date, whilst UK can demonstrate some strong statitics c.f. other countries in terms of decarbonisation of electricity grid [refs], these transitions have been largely down to technological changes and have required little individual or societal change. Further decarbonisation of UK economy will mostly now require changes that affect individuals and their mobility, homes, and leisure habits.]




practice - Obermeister, needs more

 - Transition from previous analyses to current

\begin{quote}
Challenge and Issues:    
\end{quote}

Yet, there remain questions about the degree to which scientists can influence policy, or indeed the degree to which decision-makers can be influenced by science when they also need to balance demands from many other directions [refs]. Further, the nature of “influence” is highly ephemeral, particularly within the policy/political domain where apparent policy impact can be overturned by a change in leadership [refs].

Much work on how science can inform policy sit within the policy science arena arguably not particularly accessible to scientists. Further, its is somewhat theoretical - we do not know much about the scientists real-world experiences, and whether the theoretical roles can be performed in practice. Further, the work tends to group all policy contexts together. However, evidence that certain contexts (sciences, policy goals) are valued more highly, or at least have greater propensity to attract the attention of policymakers. What are the interactions of scientists behaviours at the interface and the contextual natures of that interface...? 

- Current challenge/issue being addressed

What can ordinary scientists do, what roles can they play, how can they behave, what is the impact of science and policy contexts?

- Why different from background

There is a considerable body of work determining and developing the demand (policy) side of the interface, but less on the supply (science) side.

- What makes this challenge/issue analysis different (cases, model, conceptual framework, some combination of these)

- Transition from what contributes to analysis to how conducted

This project aims to discover the behaviours and contexts in which scientists have worked to influence policy and from this, derive a deeper understanding of their experience at the science-policy interface. Confronting directly the question of the degree of influence that scientists have on policy, this research builds on a conjecture that certain behaviours are employed by scientists at the science-policy interface, but that the context of the decision process is at least as influential. 

Specifically focussing on climate science, I will thus seek to identify the conditions (behaviours and contexts) in which policy was influenced and, equally, when this was frustrated. Where possible, I will draw conclusions about desirable conditions and common impediments with a view to offering counsel (in the form of both advice and reflection) to scientists undertaking research relevant to climate policy. 

\begin{quote}
How Analysis Was Conducted and Summary of Results: How was the analysis conducted (integration of framework, cases, and/or conceptual framework) / Summarize interesting points in the analysis / Summarize conclusions / Transition to how we got here (Roadmap)
\end{quote}

The science related to climate and nature policy is extremely varied. Policy can require knowledge about the measurement, mitigation of and adaptation to the changes that humans are causing to climate and nature. Relevant science includes the methods for measurement and ongoing monitoring of the natural world and human infrastructure, technologies for ghg emissions reduction and drawdown, technologies to reduce impacts from climate destabilisation and nature depletion, and behavioural and social science to transform individual and societal activities to reduce ghg emissions and adapt to a changing environment. technology, physical, social science

The main source of data will be semi-structured interviews with scientists about their experiences at the climate science-policy interface, working a specific knowledge discovery and transfer case. These interviews will be analysed to understand the behaviours and the contexts of the specific case using both inductive coding and existing frameworks from Policy Studies [e.g. multiple streams approach, Kingdon, Cairney], behaviour science [e.g. the ISM model], science communication [e.g. boundary roles, \cite{RapleyD2014, GluckmanBK2021}] and, in anticipation that experts do not always have agency in their roles, other key influences on climate mitigation [e.g. \cite{StoddardEtAl2021}].

Participants will be identified using several sources: REF and UKRI databases summarising research impact and direct involvement with UK climate policy. The 2014 and 2021 REF (Research Excellence Framework) provide self-reported summaries of impact from UK universities and contain a number of references to influencing climate-related policy. Similarly, UKRI Gateway to Research offers [when the database is fixed] a means to identify research that has policy-influence. From these, I will identify specific projects, and thus scientists, who have worked to influence UK public policy on climate mitigation. I will extend this list of scientists by identifying those who have specifically worked with the UK Climate Change Committee, for instance in developing Integrated Assessment Models (IAMs) for the UK’s carbon budgets.

This study will bring some balance to the discussion of action at the science-policy interface by spotlighting the experience of scientists in a particularly challenging arena - that of climate science-policy. It will characterise the behaviours exhibited by scientists to influence policy, and the contexts within which these activities occur. Where possible, advice on behaviours and contexts that are desirable for climate scientists to influence climate policy shall be derived from these characteristics. This study will also offer a reflection of the experience of scientists at the climate science-policy interface to engender deeper understanding of this experience for scientists, policy-makers and the wider public. Finally, it will offer insights into the degree to which science can realistically aspire to influence policy within the wicked context of climate change.

\begin{quote}
    Roadmap: How does this paper develop this analysis and these conclusions? / Table of contents paragraph 
\end{quote}
