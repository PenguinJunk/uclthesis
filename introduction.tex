\chapter{Introduction}\label{ch:intro}


\section{Overview of the paper}

\subsection{Problem}

\subsection{Challenge/issue}

\subsection{How analyzed/solved}

\subsection{Transition to background}

Policy requires knowledge inputs from experts such as scientists providing scientific expertise. Yet, there is wide acknowledgement and a broad literature on the gap between knowledge or science and its relevant policy, indicating that the interface between science and policy is flawed [e.g. refs]. Climate science and climate policy is probably the most stark example of such a gap, whereby many climate science experts have been reporting extreme outcomes of climate change for years (in some cases decades) [refs] and yet even the current most ambitious public policy does not go far enough to limit the impact below catastrophic levels [refs]. In this case, the gap between the settled science and the policy ambition is wide enough to threaten the survival of ecosystems [ref], cultures [ref] and potentially millions of humans [refs]. How, therefore, should scientists behave at the interface with policy to ensure that their knowledge is effectively conveyed into policy and decision-making? This thesis begins to answer this question by examining the roles played by scientists at the interface with policy to determine what behaviours ease the flow of knowledge into policy. It builds on existing work in the field of policy studies, which suggests roles played at the interface with policy, by drawing on behavioural science to understand better the actions and motivations of scientists who are involved in climate and nature science.   


\section{Background}

\subsection{Origins of problem}

\subsection{Economic, social, political, and/or technical factors at play}

\subsection{What previous analyses tell us about problem}

\subsection{Transition from previous analyses to current}


\section{Challenge and Issues}

\subsection{Current challenge/issue being addressed}

\subsection{Why different from background}

\subsection{What makes this challenge/issue analysis different (cases, model, conceptual framework, some combination of these)}

\subsection{Transition from what contributes to analysis to how conducted}


There is a considerable body of work determining and developing the demand (policy) side of the interface, but less on the supply (science) side.

Frequently scientists are accused of communicating their science ambiguously [e.g. using language that is not understood or understood differently outside of their field, refs], vaguely [e.g. stating only the issues, not the possible means of addressing them, refs] or in a manner that is inappropriate for their role [e.g. lacking objectivity, refs] and some works have attempted to describe how scientists can communicate their knowledge to better influence citizens and decision-makers [refs]. There is evidence that scientists themselves are reframing their knowledge under different contexts, perhaps with the aim of better influencing [self-censorship refs]. Yet, there remain questions about the degree to which scientists can influence policy, or indeed the degree to which decision-makers can be influenced by science when they also need to balance demands from many other directions [refs]. Further, the nature of “influence” is highly ephemeral, particularly within the policy/political domain where apparent policy impact can be overturned by a change in leadership [refs].

This project aims to discover the behaviours and contexts in which scientists have worked to influence policy and from this, derive a deeper understanding of their experience at the science-policy interface. Confronting directly the question of the degree of influence that scientists have on policy, this research builds on a conjecture that certain behaviours are employed by scientists at the science-policy interface, but that the context of the decision process is at least as influential. 

Specifically focussing on climate science, I will thus seek to identify the conditions (behaviours and contexts) in which policy was influenced and, equally, when this was frustrated. Where possible, I will draw conclusions about desirable conditions and common impediments with a view to offering counsel (in the form of both advice and reflection) to scientists undertaking research relevant to climate policy. 

\section{How Analysis Was Conducted and Summary of Results}
\subsection{How was the analysis conducted (integration of framework, cases, and/or conceptual framework)}
\subsection{Summarize interesting points in the analysis}
\subsection{Summarize conclusions}
\subsection{Transition to how we got here (Roadmap)}


The science related to climate and nature policy is extremely varied. Policy can require knowledge about the measurement, mitigation of and adaptation to the changes that humans are causing to climate and nature. Relevant science includes the methods for measurement and ongoing monitoring of the natural world and human infrastructure, technologies for ghg emissions reduction and drawdown, technologies to reduce impacts from climate destabilisation and nature depletion, and behavioural and social science to transform individual and societal activities to reduce ghg emissions and adapt to a changing environment. 

The main source of data will be semi-structured interviews with scientists about their experiences at the climate science-policy interface, working a specific knowledge discovery and transfer case. These interviews will be analysed to understand the behaviours and the contexts of the specific case using both inductive coding and existing frameworks from Policy Studies [e.g. multiple streams approach, Kingdon, Cairney], behaviour science [e.g. the ISM model], science communication [e.g. boundary roles, \cite{RapleyD2014, GluckmanBK2021}] and, in anticipation that experts do not always have agency in their roles, other key influences on climate mitigation [e.g. \cite{StoddardEtAl2021}].

Participants will be identified using several sources: REF and UKRI databases summarising research impact and direct involvement with UK climate policy. The 2014 and 2021 REF (Research Excellence Framework) provide self-reported summaries of impact from UK universities and contain a number of references to influencing climate-related policy. Similarly, UKRI Gateway to Research offers [when the database is fixed] a means to identify research that has policy-influence. From these, I will identify specific projects, and thus scientists, who have worked to influence UK public policy on climate mitigation. I will extend this list of scientists by identifying those who have specifically worked with the UK Climate Change Committee, for instance in developing Integrated Assessment Models (IAMs) for the UK’s carbon budgets.

This study will bring some balance to the discussion of action at the science-policy interface by spotlighting the experience of scientists in a particularly challenging arena - that of climate science-policy. It will characterise the behaviours exhibited by scientists to influence policy, and the contexts within which these activities occur. Where possible, advice on behaviours and contexts that are desirable for climate scientists to influence climate policy shall be derived from these characteristics. This study will also offer a reflection of the experience of scientists at the climate science-policy interface to engender deeper understanding of this experience for scientists, policy-makers and the wider public. Finally, it will offer insights into the degree to which science can realistically aspire to influence policy within the wicked context of climate change.

\section{Roadmap}
\subsection{How does this paper develop this analysis and these conclusions?}
\subsection{Table of contents” paragraph}
