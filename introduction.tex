\chapter{Introduction}\label{ch:intro}

%1 Motivation: Describe briefly the big social challenge you will address and why it is of great importance (Tip: here you can cite academic sources and non-academic sources to demonstrate that this topic is important for a wider audience and not just an academic exercise)
\CAN{} scientists, whose research reveals stark prospects for humanity, may feel an imperative to engage with policy. However, many may not have the confidence or skills to engage effectively (\cite{BednarekSHG2015,KennyRHTB2017,KEU2021perceptions}). Others experience difficulties when they do engage (\cite{Stirling2010,Gerber2023,Hicks2024}). A body of literature provides advice to scientists on how to engage with policy (\cite{OliverC2019}). Although, the evidence is that scientists are not applying this advice (\cite{CairneyTS2023}). Improving this advice to scientists, indeed improving the efficacy of science-policy engagements, requires deeper insights into the real-world experiences of scientists to discover the roles and practices that they use and what influences their actions  (\cite{KennyRHTB2017}). 

%2 Existing literature: Provide a very brief overview of the different views on the topic, i.e., causes, consequences, proposed solutions, country/case study focus. Highlight the main gap or weakness in the literature
How scientists should engage with public policy has been an ongoing, and sometimes polarised, debate for many years (e.g. \cite{Lackey2004,Nau2009,Stirling2010,Milman2013,Tyler2013,Oreskes2020,GluckmanBK2021,GregoryBW2024,Bisbal2024,Hicks2024}). Much of this debate assumes a model that science \emph{supplies} knowledge to policy, or policy \emph{demands} knowledge from science (e.g. \cite{McNie2007,KennyRHTB2017,Castree2019}). However, this model is challenged by two bodies of research. The first finds that the interface between science and policy, the \SPI, comprises complex and dynamic, non-linear, flows of knowledge (\cite{StrassheimK2014,BoswellS2017}). The second finds that \CAN{} science is ``post-normal'' (\cite{FuntowiczR1993}), in that ``facts are uncertain, values in dispute, stakes high, and decisions urgent'' (\cite[p649]{Ravetz1999}). This second conceptualisation demands a transformation of \CAN{} decision-making (\cite{FuntowiczR1993,Ravetz1999,Jasanoff2003,Hewitt2024}), which mirrors calls in the wider literature for societal and political transformation to address \CAN{} crises (\cite{DiazEtAl2019,LaybournTS2023,VerfuerthDCWP2023,GuptaEtAl2024}).

%3 Your lens: Describe your analytical approach or theoretical lens. For example, if the dominant literature has used explanations based on geography to explain resource conflicts, an alternative lens would be to look at colonial history.
Considering the \SPI{} as non-linear and \CAN{} science as \PNS, exposes an incongruity between the advice to scientists, to align to a linear \emph{supply}-\emph{demand} model of policy-making, and the complex and contested reality of the \CAN{} \SPI. Further, the debate about how scientists should engage with policy has rarely included the voices of scientists who do engage with policy. This means that, not only is the normative debate missing a descriptive counterpart, but the specific tensions for scientists that arise around \CAN{} policy are under-reported. If the advice is to be improved, the debate needs to include an understanding of the influences at the \CAN{} \SPI. This study develops such an understanding by analysing the real-world experiences described by \CAN{} scientists about their engagements with policy.

%4 Your findings and contribution: What using this lens has helped us to understand about the issue that we would not have seen before. Provide a brief overview of your key findings and how these relate to existing studies
This descriptive-analytic study of the real-world influences on scientists who engage at the \CAN{} \SPI, complements the largely normative-prescriptive advice literature. By performing a behavioural analysis of transcripts from semi-structured interviews with scientists, a range of influences on the scientists are revealed. These influences are found to be from three systems: \inte, \know{} and \scip. Additionally, whilst many of the roles and practices described in the advice literature bear out \emph{in vivo}, some scientists are found to experience tensions in their roles. Further, practices used by scientists indicate how they adapt to, mitigate or capitalise on the influences and tensions that they experience. Finally, the insights indicate potential directions for improving the engagements of scientists at the \CAN{} \SPI.

%5 Thesis structure outline: Provide a very brief outline of the next sections (e.g. “Section two provides a review of the relevant literature on…”)
This thesis is outlined as follows: Section~\ref{ch:lit} reviews the relevant literature; Section~\ref{ch:methods} describes the motivation and method of this study; Section~\ref{ch:results} reports on the results of the behavioural analysis to extract a range of influencing factors; Section~\ref{ch:discussion} synthesises the results in comparison to existing literature and suggests some potential directions for further work; Section~\ref{ch:conclusions} summarises and concludes the study.

\section{Positionality statement}\label{sec:metpositionality}

%It is pertinent to include a note on the researcher's positionality (\cite{CreswellP2017}). 
I have over 2 decades of experience working in technology in an arm's-length government organisation and have been been a \href{https://sciencecouncil.org/scientists-science-technicians/which-professional-award-is-right-for-me/csci/}{Chartered Scientist} for nearly a decade. Over this time I have had very little engagement with public policy. My career developed from studying environmental sciences followed by research into the landscape and ocean measurement. All of these have been facilitated by the cultural and social advantages of a lower middle class white British background.

Over recent years, my heightened concern about crises of social injustice, climate destabilisation and habitat degradation have led me to occasional campaigning and the informal and formal study of organisational and public decision making. I am connected via social media with scientists in a range of \CAN-related disciplines. This study arose from an observation of an incongruity between the literature about practices of engagement with policy settings and frustrations expressed on social media by experienced scientists about such policy engagements.


%\info{Overview of the paper: Problem / Challenge/issue / How analysed/solved / Transition to background}
%There is wide acknowledgement and a broad literature reporting a widening gap between science and its relevant policy (e.g. \cite{Nau2009,EdlerKB2022}\improvement{I have better refs for this}). Possibly the most stark example of this gap is between climate and nature (CAN) science and policy; Despite decades of scientists' warnings of catastrophic outcomes of climate change and biodiversity loss, the gap - for example, the difference between levels of atmospheric greenhouse gases (GHGs) and policy to reduce those GHGs\footnote{Measures of GHG volumes are useful for monitoring human impact on climate but, as \textcite{MorenoSF2016} argue, this is a narrow measure of our impact on the natural world and risks perpetuating and exacerbating societal and global injustices.} - continues to widen (\cite{StoddardEtAl2021,IPBES2022,IPCC2023}). This gap between settled science and policy ambition is wide enough to threaten the survival of ecosystems (\cite{DiazEtAl2019,IPBES2022}), global security (\cite{WEF2024}), human civilisation (\cite{TschakertEAKO2019}) and, potentially, the future of humanity (\cite{McKayEtAl2022}). With so much at stake, it is essential that all parties at the CAN policy interface are effective in their roles. Thus, scholars have developed theory around the roles for scientists at the interface with policy, and recommended how they may improve their engagement. These theories are founded studies of the needs of policy which consider how knowledge flows from science into decision-making and where these knowledge flows may be interrupted. 

%\info{Background - Origins of problem - Economic, social, political, and/or technical factors at play - What previous analyses tell us about problem - transition to current}
%The traditional ``linear'' model of policy-making is that ``knowledge shapes policy'' that is, knowledge is produced by science and this informs policy (\cite{Pielke2007}, p12-3; \cite{BoswellS2017}). In reality, policy-making can be much messier as a consequence of the complexity of policy problems (\cite{Cairney2016}), the contrasting ``cultures'' of science and policy (\cite{Dale2002,Obermeister2022}), and the competing influences of political and commercial interests (\cite{StoddardEtAl2021}). Recognising this messier reality, and the frustrations of scientists and policymakers at the interface, \textcite{Pielke2007} defined idealised roles for scientists at the interface, which have been further developed over recent years (e.g. \cite{RapleyD2014,GluckmanBK2021,GregoryBW2024}). Others, having studied the needs of policy the interface, have made recommendations for scientists, including improving their communication, their knowledge of policy processes, the relevance of their research, and the synthesis of evidence (\cite{KennyRHTB2017,LubchencoR2020,GluckmanBK2021,Bisbal2024}). To remain credible\improvement{Cash et al. 2003 CRELE is relevant here}, scientists are also held to high standards of transparency, integrity and legitimacy (\cite{KennyRHTB2017,ColognaKMBMO2024,GregoryBW2024}). There is great emphasis on the actions and roles that scientists \emph{should} assume, but this neglects to understand the actions and roles that scientists already \emph{do} assume and their real-world experiences at the CAN policy interface.

%\info{Challenge and Issues: - Current challenge/issue being addressed - Why different from background - What makes this challenge/issue analysis different - Transition from what contributes to analysis to how conducted}
%Little work exists to understand what are the experiences of scientists engaged in policy (\cite{KennyRHTB2017}) or what they have learned (with the exception of \cite{Obermeister2022}). The roles played by scientists when engaging with policy are tempered by the contexts, such as the field of science and nature of the policy, within which they engage (\cite{EdlerKB2022}). With CAN-related policy, this context is mutlifaceted, involving a range of scientific fields, approaches and outcomes; policy issues and policy settings; and wider factors such as political and commercial perspectives. However, current analyses tend to assume homogeneity across all science-policy contexts. It is perhaps unreasonable to prescribe the actions and roles that scientists should assume without first understanding what are their experiences of the CAN policy interface. Moreover, there is potentially a great deal to learn from scientists who have engaged with policy both from their perceived successes and their frustrations. Therefore, this thesis takes a step back from current works that place the onus of policy impact and influence on scientists. Instead, it uses the insights of scientists themselves to understand how the CAN interface can be enhanced by scientists and by policymakers\unsure{hopefully not too ambitious}.

%\info{How Analysis Was Conducted and Summary of Results - How was the analysis conducted (integration of framework, cases, and/or conceptual framework) - Summarize interesting points in the analysis - Summarize conclusions - Transition to how we got here (Roadmap)}
%In this study, scientists are interviewed about their real-world experiences at the CAN policy interface to discover the actions they take to ease the flow of knowledge into policy. Using the Individual-Social-Material framework of \textcite{DarntonH2013}, it reflects on the factors that influence the behaviours of scientists when they have engaged with CAN policy. It further considers how these behaviours match with the roles prescribed for scientists at the policy interface and inductively contextualises these engagements to understand how and why roles and experiences differ from from prescribed roles and from each other. This study finds [\#\info{\# - indicates text to be completed later}\emph{some interesting points}\#] and draws [\#\emph{some conclusions}\#]. 

%This study brings some balance to the discussion of roles at the policy interface by spotlighting the real-world experience and learning of scientists in a particularly challenging arena - that of CAN policy. It provides descriptive insights to a largely normative literature about scientists' roles in policy-making. In so doing it offers [\#advice\# / \#reflection\# / \#solace\#] to scientists and [\#insights\# / \#recommendations\# / \#appeals\#] to policymakers, which are intended to support the shared aim of science and policy to enhance how knowledge flows through the CAN policy interface.


%from \cite{BA2024trust} ``Policymakers refers to those directly involved in formulating policy. Within this group,  we can distinguish between elected representatives (in the legislature and in government), whom we refer to as ‘politicians’; and civil servants and senior advisors (scientific or otherwise), whom we refer to as ‘officials’. Each of these groupings will use evidence, but they may invoke  it or consider it at different stages or in different ways, and so attention to this nuance in who  is using evidence in the science-for-policymaking system, and when, is important. We try  to carefully distinguish which type of policymaker we are referring to throughout this report.''
