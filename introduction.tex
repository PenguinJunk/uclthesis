\chapter{Introduction}\label{ch:intro}


`` Academics have an important role in society as trusted figures who speak truthfully about subject matters in which they are recognised as experts (British Academy, 2022)'' quote from DykeM2024

gap has gone too far = \cite{Nau2009}



\info{Overview of the paper: Problem / Challenge/issue / How analyzed/solved / Transition to background}
Policy requires knowledge inputs from experts such as scientists providing scientific expertise\change{sound urgh}. Yet, there is wide acknowledgement and a broad literature on the gap between knowledge or science and its relevant policy, indicating that the interface between science and policy is flawed [e.g. refs]. Climate science and climate policy is probably the most stark example of such a gap, whereby many climate science experts have been reporting extreme outcomes of climate change for years (in some cases decades) [refs] and yet even the current most ambitious public policy does not go far enough to limit the impact below catastrophic levels [refs]. In this case, the gap between the settled science and the policy ambition is wide enough to threaten the survival of ecosystems [ref], cultures [ref] and potentially millions of humans [refs].\info{see \cite{GregoryBW2024} for summary of gap in climate and nature policy}

With so much at stake, it is essential that all parties play their roles effectively, to rapidly close the climate science-policy gap. Yet, debate [rages] about the roles that scientists should and could play at the interface with policy [refs], particularly climate policy [refs]. This thesis contributes that debate by studying the roles that scientists already play at the interface with policy. It discovers the behaviours that scientists use to ease the flow of interface into policy, as when as the contexts (such as the field of science and nature of the policy) that they have experienced. It builds on existing work in the field of policy studies, which theorises about roles played at the interface, by gaining behavioural and contextual insights from the scientists themselves.


\info{Background - Origins of problem}
The traditional ``linear'' model of science-related policy-making is that knowledge is produced by science and this is then absorbed into policy. Whilst this still largely dominates the expectations [refs - what is taught and what is expected], the reality is that the flow of knowledge tends to be messier (\cite{BoswellS2017}) meaning that it can be difficult to identify the appropriate role to play at the interface. [GluckmanBK2021: Cairney 2016 and Jasanoff 1994 identify that linear conceptualisation is not reflected in the real world ] \textcite{BoswellS2017} states that this linear ``knowledge shapes policy'' interface is ``largely debunked''\unsure{check quotes}. They instead offer 3 alternative models by which policy and science interact (or fail to interact) in non-linear directions. If the nature of the interface between science and policy does not meet expectations 

\info{Background - Economic, social, political, and/or technical factors at play}
These messier interactions result, at least in part, from the contrasting ``cultures'' of science and policy \cite{Obermeister2022}. Whereas as science values objectivity most highly, policy values decisiveness\unsure{policy - decisiveness?}. Where science sets a very high bar for what is considered ``evidence'', ``Policy is founded on a plurality of knowledge''\unsure{check quotes} (\cite{GluckmanBK2021}, e.g. \cite{PiddingtonMD2024}). Science (particularly the environmental sciences) prefers to only infer the most likely scenarios, policy (founded on economic projections) is comfortable considering worst-cases ([Sharpe, Pearson reference in GregoryBW2024: crying wolf, ReadO2017]). There are linguistic differences between the two cultures whereby the same word (such as ``uncertainty'', ``risk'' \cite{Makin2024} and [proof]\change{find linguistic differences}) have different meanings [De Meyer]. \cite{MountfordD2023} - Scientists are much more likely to tell you what they don't know than what they do.. Also, the incentives structures can be different. These different cultures serve the purposes of pure science and policymaking well, but create barriers to effective conveyance of knowledge into policy and decision-making.

Work over the last two decade has considered the possible roles that scientists can play to overcome such barriers. Much of this work builds on the four idealised roles for scientists set out by Roger Pielke Jr.: \emph{pure scientist} (unconcerned by policy), \emph{science arbiter} (answers questions posed by policy), \emph{issue advocate} (engages policy to promote a particular decision) and \emph{honest broker of policy alternatives} (presents all the relevant alternatives to policy derived by synthesising scientific knowledge) (\cite{Pielke2007}). \textcite{RapleyD2014} add to this list \emph{science communicator}, who engage society in the conversation. Acknowledging the existential risk posed by the climate and nature crisis, \textcite{GregoryBW2024} propose the \emph{honest advocate} (advocates for a particular policy outcome whilst keep to strict criteria of honesty and transparency).  Pielke's roles are derived from the literature on Science and Technology Studies (\cite{Pielke2007}, p8). However, within Policy Studies, other roles are envisaged, particularly the \emph{policy entrepreneur} (advocates for particular proposals using a range of strategies) (\cite{Kingdon1993,Cairney2018}) and \emph{problem broker} (frames and advocates for a particular policy problem) (\cite{Knaggard2015}). Whilst these latter roles are often discussed uncritically, sometimes even [praised] for their ability to `get policy over the line', they lack legitimacy, accountability and transparency (\cite{vonMalmborg2024strategies}) and are thus [anathemic] to scientific [practice]. Thus, it may be surmised that the ability of scientists to influence policy is somewhat [hamstrung] by the constraints of [acceptable] scientific roles, constraints that do not apply to other other roles being played at the interface. [Crouzat et al. 2018 reference in BalvaneraJNOBCDGGKKMPSSW2020) ][co producing solutions by building bridges between a range of knowledges NorstromEtAl2020 in BalvaneraJNOBCDGGKKMPSSW2020, MatukBSAHT2020]

\info{Background - What previous analyses tell us about problem}

Focus on the roles of scientists in conveying knowledge into policy have suggested that scientists consider the clarity of their communication [e.g. using language that is not understood or understood differently outside of their field, refs], and to support decision makers by providing possible options [e.g. stating only the issues, not the possible means of addressing them, refs]. There is a conflict between these theoretical approaches to the communication of knowledge and the culture [e.g. objectivity, refs] and some works have attempted to describe how scientists can communicate their knowledge to better influence citizens and decision-makers [refs]. There is evidence that scientists themselves are reframing their knowledge under different contexts, perhaps with the aim of better influencing [self-censorship refs].

scientists instead turn to activism \cite{Pivovarchuk2024,GregoryBW2024}

scientist learns to answer the questions that policy asks \cite{Gerber2023}

Of course, the policy maker plays their role be selecting the knowledge to be used. W  Decisions makers can choose to ignore evidence (e.g. \cite{TennoyHLN2016}), rarely needing to justify what they include and what they leave out\footnote{not the case with EU Climate Change Advisory Board \cite{WardmanE2023}}.

\cite{EdlerKB2022} - `` critique the current literature's emphasis on the efforts of scientists to generate policy impact, because it neglects the role of `user' policymaking organisations''

Also, less well documented\unsure{is it?} are the power influences at play, that policy, even policy based on scientific understanding, such as the nature of climate change, is [pertinent to] other factors such as commercial interests, concerns about social acceptance of transition [activities] [etc]. [supranational nature of climate governance, colonialism and imperialism in geopolitics, power relations in national governance and decision-making, lobbying by vested interests, dominance of economics and financial interests in governance, etc.]

[Thus, to date, whilst UK can demonstrate some strong statitics c.f. other countries in terms of decarbonisation of electricity grid [refs], these transitions have been largely down to technological changes and have required little individual or societal change. Further decarbonisation of UK economy will mostly now require changes that affect individuals and their mobility, homes, and leisure habits.]




practice - Obermeister, needs more

 - Transition from previous analyses to current

\begin{quote}
Challenge and Issues:    
\end{quote}

Yet, there remain questions about the degree to which scientists can influence policy, or indeed the degree to which decision-makers can be influenced by science when they also need to balance demands from many other directions [refs]. Further, the nature of “influence” is highly ephemeral, particularly within the policy/political domain where apparent policy impact can be overturned by a change in leadership [refs].

Much work on how science can inform policy sit within the policy science arena arguably not particularly accessible to scientists. Further, its is somewhat theoretical - we do not know much about the scientists real-world experiences, and whether the theoretical roles can be performed in practice. Further, the work tends to group all policy contexts together. However, evidence that certain contexts (sciences, policy goals) are valued more highly, or at least have greater propensity to attract the attention of policymakers. What are the interactions of scientists behaviours at the interface and the contextual natures of that interface...? 

- Current challenge/issue being addressed

What can ordinary scientists do, what roles can they play, how can they behave, what is the impact of science and policy contexts?

- Why different from background

There is a considerable body of work determining and developing the demand (policy) side of the interface, but less on the supply (science) side.

- What makes this challenge/issue analysis different (cases, model, conceptual framework, some combination of these)

- Transition from what contributes to analysis to how conducted

This project aims to discover the behaviours and contexts in which scientists have worked to influence policy and from this, derive a deeper understanding of their experience at the science-policy interface. Confronting directly the question of the degree of influence that scientists have on policy, this research builds on a conjecture that certain behaviours are employed by scientists at the science-policy interface, but that the context of the decision process is at least as influential. 

Specifically focussing on climate science, I will thus seek to identify the conditions (behaviours and contexts) in which policy was influenced and, equally, when this was frustrated. Where possible, I will draw conclusions about desirable conditions and common impediments with a view to offering counsel (in the form of both advice and reflection) to scientists undertaking research relevant to climate policy. 

\begin{quote}
How Analysis Was Conducted and Summary of Results: How was the analysis conducted (integration of framework, cases, and/or conceptual framework) / Summarize interesting points in the analysis / Summarize conclusions / Transition to how we got here (Roadmap)
\end{quote}

The science related to climate and nature policy is extremely varied. Policy can require knowledge about the measurement, mitigation of and adaptation to the changes that humans are causing to climate and nature. Relevant science includes the methods for measurement and ongoing monitoring of the natural world and human infrastructure, technologies for ghg emissions reduction and drawdown, technologies to reduce impacts from climate destabilisation and nature depletion, and behavioural and social science to transform individual and societal activities to reduce ghg emissions and adapt to a changing environment. technology, physical, social science

The main source of data will be semi-structured interviews with scientists about their experiences at the climate science-policy interface, working a specific knowledge discovery and transfer case. These interviews will be analysed to understand the behaviours and the contexts of the specific case using both inductive coding and existing frameworks from Policy Studies [e.g. multiple streams approach, Kingdon, Cairney], behaviour science [e.g. the ISM model], science communication [e.g. boundary roles, \cite{RapleyD2014, GluckmanBK2021}] and, in anticipation that experts do not always have agency in their roles, other key influences on climate mitigation [e.g. \cite{StoddardEtAl2021}].

Participants will be identified using several sources: REF and UKRI databases summarising research impact and direct involvement with UK climate policy. The 2014 and 2021 REF (Research Excellence Framework) provide self-reported summaries of impact from UK universities and contain a number of references to influencing climate-related policy. Similarly, UKRI Gateway to Research offers [when the database is fixed] a means to identify research that has policy-influence. From these, I will identify specific projects, and thus scientists, who have worked to influence UK public policy on climate mitigation. I will extend this list of scientists by identifying those who have specifically worked with the UK Climate Change Committee, for instance in developing Integrated Assessment Models (IAMs) for the UK’s carbon budgets.

This study will bring some balance to the discussion of action at the science-policy interface by spotlighting the experience of scientists in a particularly challenging arena - that of climate science-policy. It will characterise the behaviours exhibited by scientists to influence policy, and the contexts within which these activities occur. Where possible, advice on behaviours and contexts that are desirable for climate scientists to influence climate policy shall be derived from these characteristics. This study will also offer a reflection of the experience of scientists at the climate science-policy interface to engender deeper understanding of this experience for scientists, policy-makers and the wider public. Finally, it will offer insights into the degree to which science can realistically aspire to influence policy within the wicked context of climate change.

\begin{quote}
    Roadmap: How does this paper develop this analysis and these conclusions? / Table of contents paragraph 
\end{quote}
