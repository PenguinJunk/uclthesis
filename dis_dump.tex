
\iffalse
\section{braindump}

patient science - good quality science is essential for policy and can take a long time to accumulate - requires long term investment (no expectation of returning quick impact). This may be difficult with [highly specified funding requirements]

\subsection{To institutions}
incentives are not fully there for frank policy engagement - e.g. \cite{ElsensohnACDGGKPRS2019} ``biases and limitations at scientific institutions, including but not limited to, a lack of incentive structures, institutional guidelines, and employment limitations'' - example of programme of training in science advice, maybe advocacy \cite{RussellWC2008}

\subsection{science versus evidence}
science versus evidence p09
evidence review p10, synthesising p01..., hub p06/3 p06/29

\subsection{Frames and windows}
meanings - all about framing
policy institutions - frame the topic and the times very tightly - windows of opportunity
this runs counter to science, which may attempt to transgress disciplinary boundaries and has always proved difficult to timeframe
frames seen being applied more in science - what will be funded, what is easier to publish
such framings, each being concluded with a decision, structure the policy process
- some see this as a constraint, meaning some topics are always off the table but it may also open up policy to prepared researchers who can prepare in advance for forthcoming framings [KEU advice]

\subsection{citizen engagement}
citizens as stakeholders and subjects of CAN policy decisions were mentioned.... The adverse impacts of [the waterfall of policy] at the community level were described in detail by one p11 ... for this reason, others expressed a belief that people should be involved in changes that will affect them (such as through CAs) p03  p01 aware of impact of their technologies on people, and considering researching how to address this, as well as being very open and keen to demonstrate to anyone who is willing to ask

\subsection{intimate understanding}
of policymakers and policy contexts p01, ...
of people and society ..., of the science and projections of CAN for so many years [p02, ... , p08]
\subsection{long time} {p01}{115} {p02}{61} {p03}{102} {p05}{88} {p07}{18} {p08}{55} {p09}{51} {p10}{61} {p11}{7} {p13}{09} {p14}{78}{112}

\subsection{types of science}
Differences between physical, technology and social science contexts?
Considering WesselinkH2020 typology of problems, at first glance climate change appears to be moderately structured since much of the knowledge now has a high certainty. However, it is in the solutions that the science becomes uncertain. Technology claims with confidence that is contested [refs to comparisons on NET and GGR papers laying out uncertainty of these]. The social science, in terms of the responses of individuals and society to impacts of the CAN crisis and to policies addressing it, is very uncertain. This is less likely to be expressed with certainty and does this thus explain the lower engagement of social scientists with policymaking?

Knowledge that is digestible policy - but also knowledge that is needed by policy. One of the major bottlenecks is surely around society and behaviour - these are the questions that still need answering. Therefore, presenting knowledge on social and behavioural consequences of current and potential policy. Such as what messages are out current emphasis of technological solutions (EVs, CSS, SAFs) implying for individuals and organisations.

\subsection{Academic impact on policy}
One of the main means of incentivising and rewarding the engagement of academia with [public decision making] has been to include policy impact in the REF, particularly the most recent REF [ref]. [very short summary of how impact case studies are selected by institutions]. An observation that came up several times was that the work that the participants had been involved in was selected for inclusion in their institution's REF submissions but they had never designed the work to have such an outcome. Moreover, there was [often/always] [an expression of surprise] that their work had ended up in the direction of impacting policy. For instance, In one case the work was actually designed to inform citizens. 

REF ``builds largely on linear models of the policy process'' \cite{CairneyO2020}

The serendipity and chance nature of policy engagement means that the highest impact research may never be designed to have such impact. And yet the nature of incentivisations such as REF and [UKRI paths to impact] may result in rather na\"ive attempts to impact policy which could be a distraction and hindrance rather than a benefit. [More scientists sending their outputs into policy individually] [needs to be coordinated?]

\subsection{what works}
There are some activities that seem to be more successful, at least the few participants who were using those strategies seemed to rate highly their value. 

It is tempting to dismiss the myriad other strategies that scientists mentioned in the course of this research - e.g.s - which had inconclusive outcomes. However, it is also possible that the diversity of strategies has been valuable, creating a range of events, venues and media through which the richness of the science is conveyed. Whilst the more targetted strategies ultimately show signs of [hitting home], this may be due to these previous ``pre-softening'' activities (\cite{Cairney2018}).

Forget the deficit model, build networks and relationships across divides, collaborate on topics, create meaning, \tquote{we have to write in to our bids, given that we know that information by itself is going to be limited, it's going to be relational, it's going to be messy, it's going to be long term, all these other things we're going to create mechanisms that hopefully align with that as far as we can. So we'll have placements and we will have stakeholder coalition workshops where we bring together from lots of different departments and sectors and things and try and break down silos and we'll have some training and master classes so that we're actually building capacity and all these things. This isn't just a tick box, we didn't just have journal papers, we also had a couple of briefing papers, a whole other gigantic set of activities to build capacity and relationships}{p05}{143}

but as \textcite{CairneyO2020} observe, success has more to do with context and entrepreneurship

By nature, participants are people who respond to requests to provide information

Passion for their subject evident across participants

Keen for policy to use best knowledge and feel they have that knowledge

Evidence of awareness of policy cycle? P5

Comments on political context and/or change of government

Involvement of other organisations - industry, NGOs, civil society, citizens?
\subsection{Measures of success}
Participants had been selected based on some external statement of ``success'' of their engagement, such as the engagement becoming a REF impact case study, being reported in a blog post or mentioned by a contact. However, the meaning of success emerged, and more latterly was enquired about within the interviewing process (with questions similar to ``what would you like to have seen arise from [that action]?'').

Being more readily rewarded by the process, rather than seeking an outcome, may feel more successful (NOT! Section~\ref{sec:resskifram} also p08 v p03). In complex issues, outcomes are unpredictable and emergent (maybe \cite{SnowdenB2007} and it is well know that it is difficult to trace input to outcomes \cite{BednarekSHG2015}, perhaps something in the public participation literature on this e.g. \cite{Sprain2016}) and so 

Those who are less concerned about standard academic measures of success such as publications and REF impact case studies are possibly able to draw on other means of demonstrating success - those working in technology for instance are able to demonstrate prototypes p01. Several participants [p09, p13] who work closely with policy identified that they derived a sense of success from knowing that good science had been made available to decision makers within the policymaking process.

Those who tended to indicate more frustration, and less of a feeling of being successful [p03, p11], were [perhaps identifying frustrations more with the system of decisionmaking]. Indeed, participants what were experiencing success, were working, often very strategically, within the existing system. Their successes were in relation to an existing system, and even those working towards changing that system (such as by proposing the economic foundations of decisionmaking) were careful to work within current implicit and explicit rules. This is a pragmatic approach, not only from the perspective that some change that benefits CAN is better than none, but also for maintaining psychological resilience. Yet, [difficult to ignore the perspective of those frustrated by the marginalisation of more radical perspectives on the evidence that is used and how that evidence is ingested into decisionmaking]  

And yet, means to measure the impact of science, such as the REF, rather assume the linear / deficit models are dominant [ref]. 
\subsection{What place more diverse knowledge?}

Knowledge is an institution [Leviathan and the Air Pump], most of us are unaware of the particular perspective our knowledge creates, and this idea wasn't generally picked up in the interviews

\cite{Bendell2024} most scientists are limited by their own privilege, institutional contexts, and position ``within the system'', and thus fail to advocate for the changes that are required for truly just transition  
\cite{StoddardEtAl2021} problems arise due to powerful interests and resulting institutions leading to a pervasive failure to [question many of the core tenets of modern, industrialised societies]. Whether or not this is true, the influence of vested interests, alongside lack of political will, it is given by climate scientists as the main reasons for inadequate action on climate change \cite{Carrington2024} 
\cite{TurnhoutMWKL2020} - power and politics in shaping processes and outcomes
\fi