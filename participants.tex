\section{Participant profiles}\label{sec:profiles}
These are provided to aid thought and discussion and early analysis and probably won't be retained in the final version.

\subsection{Participant 1}\label{sec:p1}
[developing and managing new technologies, policy-experienced, academia-based] Scientist got into field out of very personal interest with no thought of climate relevance. Had been invited to engage in the past and this has developed over the years with ongoing conversations and understanding about what is of value to policymakers. Main driver seems to be to supply public value, in return for the long term funding and because this is the civic thing to do. Also, where previously they'd solely focussed on working with industry to have impact, they've realised that working with policy is as important. Continues to work in science but now with a clear view of how this can best serve policy. Sees others' frustrations with policy but does not find it frustrating because they have a strong insight into the processes involved in making policy.

\subsection{Participant 2}\label{sec:p2}
[understanding and managing the physical environment, policy-inexperienced, academia-based] Scientist has been working the field for a couple of decades with no involvement in policy. Their field has recently become much higher on the policymaking agenda and was invited to present evidence to a committee. This came about due to connections to others who are more established in giving evidence. Process was enjoyable and they would be interested in being involved more. Now thinking about policy relevance of research including in publications as keen to add value to policy with their work.

\subsection{Participant 3}\label{sec:p3}
[understanding and managing the people and society, policy-experienced, academia-based] Scientist who is highly motivated to support transformational policy regarding climate. As such they have contributed in many ways to calls for information, directly their output in ways that they believe are palatable by policy. Yet their experience is largely frustrating because they have not seen a great deal of traction regarding the inclusion of their scientific evidence in policy making.

\subsection{Participant 4}\label{sec:p4}
[understanding and managing the physical environment, policy-experienced, public servant] scientist, public servant, who has provided scientific information in response to requests. Enjoyed the recognition of some of those engagements but has also experienced the cultural gap between science and policy. Frustration arises from not understanding what more information decision makers need to close the gap between science and policy.

\subsection{Participant 5}\label{sec:p5}
[understanding and managing the people and society, policy-experienced, academia-based] scientist with a very keen interest in influencing policy. Has consciously engaged decision makers since very early in research career and developed a range of strategies to attempt to influence policy. So much so that they feel they spend more energy on policy than science and can also feel uncomfortable with the ``sales'' nature of trying to influence policymakers. Has had successes particularly with local and devolved governments and so enjoyed being involved with committee for House of Lords.

\subsection{Participant 6}\label{sec:p6}
[understanding and managing the people and society, policy-experienced, academia-based] scientist with direct experience of UK government. Has actively sought out positions to be at the heart of climate-related decision making. Currently running a programme to synthesise existing knowledge into forms that are directly relevant for decision makers throughout government and the state.

\subsection{Participant 7}\label{sec:p7}
[chemist, policy-experienced, public servant] who left science quite early and played a number of roles in European and UK research co-ordination and policy. Now performs a policy and public engagement role for a science research group. Has been focusing on building and maintaining relationships with policy officials in relevant departments.

\subsection{Participant 8}\label{sec:p8}
[understanding and managing the physical environment, policy-experienced, academia-based] scientist who published ideas that gained traction and has built on their reputation by engaging deeply with governments at many levels. They are driven by the  CAN issue as they have seen it through their own research and honed skills in economics, policy and negotiation in order to gain a seat at the table in a number of different settings.

\subsection{Participant 9}\label{sec:p9}
[multi/trans/a-disciplinary, policy-experienced, academia-based] scientist with direct experience of policymaking from early on their scientist career. Very thoughtful and knowledgeable about how, and indeed whether, scientists should engage with policy. Derives reward from knowing that good science has been presented well in policy settings enabling constructive conversations from which all parties learn.

\subsection{Participant 10}\label{sec:p10}


\subsection{Participant 11}\label{sec:p11}
[understanding and managing the physical environment, policy-experienced, academia-based] scientist who has ``always'' been engaged in policy both at `high' level, disseminating policy, and at lower level managing the impacts of policy in the field, thus much less trying to influence policy as responding to it. Strongly motivated by the injustices in the disconnection between the two worlds (decision making and action taking) and impassioned by a sense of the loss of influence by scientific knowledge. Much of their work now is in modelling and mitigating the negative effects of policy. 

\subsection{Participant 12}\label{sec:p12}
[understanding and managing the people and society, policy-experienced, academia-based] scientist with experience in government settings. Research is long term, designed to understand social transition. They are very aware that engagements with policy develop via social networks and they are careful to develop these alongside engaging with policy where possible. 

\subsection{Participant 13}\label{sec:p13}
[understanding and managing the physical environment, policy-experienced, public servant] scientist who was always aware of the policy implications of their work. Having set up a research programme for deeper understanding key aspects of the biohphysical environment that were to become highly relevant to policy, and after experiencing policy through writing independent scientific reports for specific departments, they left academia to work in a government department. Continues to have a strong conviction that science is vital to decision-making and that science should be of the highest quality and evidence should be presented to policymakers with balance. 

\subsection{Participant 14}\label{sec:p14}
[understanding and managing the physical environment, policy-experienced, ex-public servant] scientist with a great deal of experience, and well-established relationships, with government departments. Despite this, is well aware of a need to continue cultivating relationships in orfer to have impact on policymaking. 