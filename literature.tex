\chapter{Literature Review}\label{ch:lit}
\information{what do we mean by science and scientists?}
\information{what do we mean by policy and policymakers?}

\section{Gaps between climate science and climate policy}\label{sec:gaps}
\emph{foundation of this thesis is the belief that this gap is a problem} \\
the nature of the gap(s) and their consequences

(Suggested causes of the gap(s))
\emph{overview of work giving reasons for the gap - not too detailed but these concepts are useful for the rest of the literature review} \\
use same order/topics as in introduction\\
science and policy, two very different philosophies\\
communication of science to non-scientists\\
selection of science by policy-makers\\
post-normal science\\
external influences

\subsection{A wicked problem}
Wicked problems \cite{RittelW1973}
\cite{WesselinkH2020} - unstructured problem - low certainty of knowledge (social science) low agreement on goals

, particularly given the `wicked' nature of the issues and their possible solutions (\cite{Cairney2016}, p94). 


[Thus, to date, whilst UK can demonstrate some strong statitics c.f. other countries in terms of decarbonisation of electricity grid [refs], these transitions have been largely down to technological changes and have required little individual or societal change. Further decarbonisation of UK economy will mostly now require changes that affect individuals and their mobility, homes, and leisure habits.]

\subsubsection{What does ``effective'' or ``successful'' mean in this context?}
When trying to describe the gap between science and policy it is valuable to be able to articulate what close that gap would look like\check{what do writings about the science policy gap want to see improved?}
In the arena of CAN policy, the clearest measure would be if policy ambition met with physical modelling of the need to close the gap between budget and emissions. Other measures would be to reach zero habitat loss? 

\cite{BednarekSHG2015} - section 4.4 Measuring impact is difficult in any complex setting 
\cite{JagannathanEtAl2023} - section 4.1.1 identify that ``How do we define and evaluate success in producing actionable knowledge?'' needs deeper research as the definition and methodologies are very context-dependant
\cite{KEU2021impact} - its difficult to define impact, for REF 2021 it comes down to reach and significance

\subsection{Two cultures}
These messier interactions result, at least in part, from the contrasting ``cultures'' of science and policy \cite{Obermeister2022}. Whereas as science values objectivity most highly, policy values decisiveness\unsure{policy - decisiveness?}. Where science sets a very high bar for what is considered ``evidence'', ``Policy is founded on a plurality of knowledge''\unsure{check quotes} (\cite{GluckmanBK2021}, e.g. \cite{PiddingtonMD2024}). Science (particularly the environmental sciences) prefers to only infer the most likely scenarios, policy (founded on economic projections) is comfortable considering worst-cases ([PoeS2023, Pearson reference in GregoryBW2024: crying wolf, ReadO2017]). There are linguistic differences between the two cultures whereby the same word (such as ``uncertainty'', ``risk''  and ``error'') have different meanings (\cite{SomervilleH2011,Makin2024}[De Meyer,]). Scientists can even be reluctant to make confident statements about what they know, preferring instead to talk about what they don't know (\cite{MountfordD2023}). These different cultures serve the purposes of pure science and policymaking well, but create barriers to effective conveyance of knowledge into policy and decision-making.


\section{The science-policy interface}\label{sec:interface}
\emph{normative and descriptive literature on the how science and policy interact - again for foundation and to provide framework for the contexts that should arise from the study}\\
models of science-policy relations - the standard model of evidence influencing policy and work that challenges this\\
examples of science-policy interfaces in climate policy, e.g. institutions, projects

look up OECD scientific advice to policymaking

\cite{StrassheimK2014} tensions due to selectivity and power - ``intensive and complex struggle for political and epistemic authority on both sides; science as well as policy''

\cite{McNie2007} - nice review of work talking about the linear model section 2.2.1. Policy
\cite{HaynesDCRHGS2011} - ``Linear, rational depictions of the 'the policy cycle' are increasingly viewed as idealised normative models that poorly describe a far messier social process (Greenhalgh, 2006, Hanney et al., 2003, Lewis, 2006).''
\textcite{BoswellS2017} states that this linear  interface is ``widely debunked''. They instead offer 3 alternative models by which policy and science interact (or fail to interact) in non-linear directions. If the nature of the interface between science and policy does not meet expectations 

Owing to the complexities in the flow of information at the interface, boundary organisations of a range of design have been established worldwide. These include IPCC and IPBES, international organisations who are [trying really hard] to establish the policy-relevance science as well as co-produce the policy advice. These organisations provide essential knowledge for policymakers, but the policy itself is nationally- and locally-[made]. In the UK, CCC [acts as a science-policy boundary organisation?\unsure{is CCC a boundary organisation?}]... Boundary organisations are [doing an amazing job], indeed in issues other than CAN they may be doing a good enough job. However, when it comes to the gap between CAN science and policy decisions are currently deeply - and dangerously - inadequate. \unsure{is there any evidence that boundary organisations have helped in CAN at all?}

\cite{OlejniczakBDP2019} describe the s-p gap and reasons for it

Whilst SPIs are often suggested, many scientists engage directly with policy makers through... calls for evidence, committees, ... less formally

\section{Roles at the interface}\label{sec:roles}
\emph{theory and observation of roles played between science and policy - this is the core of the literature because I am aiming to understand what are the roles that could be, should be and are played, their benefits and trade-offs, and how scientists specifically inhabit these roles}\\
key theories from policy studies (e.g. Policy Entrepreneur, Problem Broker, Pielke’s roles)\\
characteristics exhibited in each role\\
conflicts between philosophical position of scientists and roles proposed at interface\\
studies identifying the roles played by scientists at the interface

, \textcite{Pielke2007}, defined four idealised roles for scientists at the interface: \emph{pure scientist} - unconcerned by policy; \emph{science arbiter} - answers questions posed by policy; \emph{issue advocate} - engages policy to promote a particular decision; and \emph{honest broker of policy alternatives} - presents all the relevant alternatives to policy derived by synthesising scientific knowledge. \textcite{RapleyD2014} add to this list \emph{science communicator}, who engage society in the conversation. Acknowledging the existential risk posed by the climate and nature crisis, \textcite{GregoryBW2024} propose the \emph{honest advocate} (advocates for a particular policy outcome whilst keep to strict criteria of honesty and transparency).  Pielke's roles are derived from the literature on Science and Technology Studies (\cite{Pielke2007}, p8). However, within Policy Studies, other roles are envisaged, particularly the \emph{policy entrepreneur} (advocates for particular proposals using a range of strategies) (\cite{Kingdon1993,Cairney2018}) and \emph{problem broker} (frames and advocates for a particular policy problem) (\cite{Knaggard2015}). Whilst these latter roles are often discussed uncritically, sometimes even [praised] for their ability to `get policy over the line', they lack legitimacy, accountability and transparency (\cite{vonMalmborg2024strategies}) and are thus [anathemic] to scientific [practice]. Thus, it may be surmised that the ability of scientists to influence policy is somewhat [hamstrung] by the constraints of [acceptable] scientific roles, constraints that do not apply to other other roles being played at the interface. [Crouzat et al. 2018 reference in BalvaneraJNOBCDGGKKMPSSW2020) ][co producing solutions by building bridges between a range of knowledges NorstromEtAl2020 in BalvaneraJNOBCDGGKKMPSSW2020, MatukBSAHT2020]
credibility balanced with usefulness WesselinkH2020
constraint of credibility does not apply to other advocacy roles that aim to influence policy makers (\cite{Kingdon1993,Knaggard2015,Cairney2018,vonMalmborg2024strategies})

\cite{SteelLLS2004,SinghTKMMC2014} - 5 roles: reporting, interpreting, integrating, taking a position, decision making

Levien 1979 identified 3 roles for scientists - clear description including uncertainties, options, contributing to problem solving (get citiation from \cite{SteelLLS2004})

\cite{ColognaKMBMO2024} scientists' advocacy for greater action may not affect credibility but advocating for a specific policy may

\cite{ColognaKMBMO2024} - important for scientists to demonstrate competence, benevolence, integrity and openness

\cite{Horton2022} - classic example of Pure Science approach at the policy interface - ``the scientists made no political recommendations, as they were there simply to present the science'' - very little, possible nothing, came of it.

\cite{FuntowiczR1993,Jasanoff2003} - in our era of \emph{post-normal science} scientists should form \emph{extended peer communities} with other affected citizens to consider impacts of their science and of policy interventions (also \cite{KalafatisL2019})

\subsection{Strategies at the interface}
\cite{BednarekSHG2015} - section 4.1 nice summary of the kinds of strategies for scientists that are found literature and examples of skills that are useful in the wider team to support the conveyance of knowledge.

\section{science policy contexts}

understood now that context of policymaking is relevant \cite{CairneyW2017} and models of these contexts have been devised \cite{WeyrauchES2016}

The science related to CAN policy is extremely varied. Policy can require knowledge about the measurement, mitigation of and adaptation to the changes that humans are causing to climate and nature. Relevant science includes the methods for measurement and ongoing monitoring of the natural world and human infrastructure, technologies for ghg emissions reduction and drawdown, technologies to reduce impacts from climate destabilisation and nature depletion, and behavioural and social science to transform individual and societal activities to reduce ghg emissions and adapt to a changing environment. technology, physical, social science

\cite{DanfordDR2009} context of government body scientists

\cite{BalvaneraJNOBCDGGKKMPSSW2020} - Power imbalances between academic disciplines

\section{Scientists’ experiences of policy}\label{sec:experiences}
\emph{review of literature describing the experience of scientists - I believe my study is one of very few looking at science-policy specifically from the scientists’ perspective}
\cite{OjanenBKP2021} deep investigation of experiences of scientists - include more detail from this

\cite{DanfordDR2009} - experiences of working on government science from the perspective of management, nature of work, ... but not directly involved in policy

\cite{KothariME2009} - study of health researchers' experiences with public policy

\section{Methods literature}\label{sec:methodlit}
\emph{the literature relevant to how I will discover scientists’ experiences}\\
discovering contexts using inductive coding - the contexts in which scientists engage with policy have many dimensions (e.g. actors, environments, impact type, stage in policy cycle, etc.). Within my small study, and possibly across the arena of science for climate policy, it is probable that engagement occurs in only a few subcontexts (e.g. providing evidence (impact type) to parliamentary committee members (actors) as part of the agenda setting process (policy cycle))\\
discovering roles using behavioural analysis - given the conflicts between science and policy, which can be stronger even than other areas of expertise (economics, medicine,) I will not assume specific theoretical roles are played and instead perform a comprehensive analysis of the behaviours exhibited by scientists

\cite{AtkinsFIOPIFDCGLM2017} - theoretical domains that relate to behaviour change

\section{text dump}
\cite{LubchencoR2020} - in 1998 Lubchenco called for greater effort by scientists to make science accessible, understandable, relevant and credible - whilst much has been achieved in 2 decades, still barriers, largely due to the nature of the institutions of science
\cite{Gerber2023} - scientists should answer the questions posed by policy but what if policy is not asking the right questions? e.g. Emphasis of policy on NET ValiverronenS2021, CalverleyA2022, Carton2021, CartonHML2023 rather than behaviour and society
\cite{Bisbal2024} - expresses the frustration of the policymaker unable to gain the knowledge they require and suggests actions that scientists can take 


the issues of credibility, objectivity, etc. To an extent disadvantage scientists. Voluntary sector outperforms high education KennyRHTB2017. CernaTHTTS2020 said that ``Greta Thunberg has been more successful in communicating the globally agreed scientific facts concerning climate change that the thousands of scientists actively involved in the International Panel on Climate Change'' yet Greta was not bound by the restrictions of scientific practice. Yet, policymakers are also influenced by actors playing issue and policy and advocacy roles that may have much less legitimacy, accountability and transparency (\cite{Kingdon1993,Knaggard2015,Cairney2018,vonMalmborg2024strategies}) , potentially putting scientists at an influential disadvantage. Voluntary sector outperforms higher education KennyRHTB2017. studies looking mainly at the experience of the demand-side / policy
LubchencoR2020 - in 1998 identified that science is not delivering on its social contract it is not accessible, understandable, or seen as relevant, or credible and thus not 
POST study found underperformance of academia in contribution to uk parliament. Areas for academia to improve: communication, understanding of parliamentary processes, research relevance, credibility KennyRHTB2017  
communicate the impacts that are meaningful Sharpe2019
GluckmanBK2021 make 8 recommendations for effective knowledge brokerage
SomervilleH2011, Makin2024 - scientists use language that had a different meaning in society 
roles played to advocate for CAN policy - FOE as PE in CarterC2018 - NGOs as PE in Braun2009

Of course, the policy maker plays their role be selecting the knowledge to be used. W  Decisions makers can choose to ignore evidence (e.g. \cite{TennoyHLN2016}), rarely needing to justify what they include and what they leave out\footnote{not the case with EU Climate Change Advisory Board \cite{WardmanE2023}}.


Also, less well documented\unsure{is it?} are the power influences at play, that policy, even policy based on scientific understanding, such as the nature of climate change, is [pertinent to] other factors such as commercial interests, concerns about social acceptance of transition [activities] [etc]. [supranational nature of climate governance, colonialism and imperialism in geopolitics, power relations in national governance and decision-making, lobbying by vested interests, dominance of economics and financial interests in governance, etc.]

scientists instead turn to activism \cite{Pivovarchuk2024,GregoryBW2024}

scientist learns to answer the questions that policy asks \cite{Gerber2023}

