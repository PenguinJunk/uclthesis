\chapter{Literature Review}\label{ch:lit}

\section{The science-policy interface}\label{sec:litspi}
\subsection{SPI - general}
\cite{JagannathanEtAl2023} - ``Among these numerous frameworks, science-policy interface (SPI) (van Ravenswaay et al., 1983) is one of the earliest and most popular, that describes the exchange between science and decision-making''
place that knowledge exchange between science and policy occurs
look up OECD scientific advice to policymaking
\emph{normative and descriptive literature on the how science and policy interact - again for foundation and to provide framework for the contexts that should arise from the study}\\
models of science-policy relations - the standard model of evidence influencing policy and work that challenges this\\
examples of science-policy interfaces in climate policy, e.g. institutions, projects

\subsection{SPI - tensions}

\subsubsection{rational model}
\cite{BoswellS2017} - linear model ``widely debunked'', presents 4 models of research-policy relations
\cite{BednarekSHG2015} - ``complicated and erratic flow of information''
\cite{McNie2007} - nice review of work talking about the linear model section 2.2.1. Policy
\cite{HaynesDCRHGS2011} - ``Linear, rational depictions of the 'the policy cycle' are increasingly viewed as idealised normative models that poorly describe a far messier social process (Greenhalgh, 2006, Hanney et al., 2003, Lewis, 2006).''
\textcite{BoswellS2017} states that this linear  interface is ``widely debunked''. They instead offer 3 alternative models by which policy and science interact (or fail to interact) in non-linear directions. If the nature of the interface between science and policy does not meet expectations 
\cite{Cairney2018} - ``rational policymaking during a policy cycle'' - ``the biggest work of fiction in policy studies'
\cite{StrassheimK2014} - argues that the other side of evidence-based policy may be policy-based evidence ``evidence-based policy immunising itself against criticism and thus turning into policy-based evidence''

\subsubsection{science-policy gap}
Such study and discussion of the nature of the interface between science and policy is catalysed by the perceived ``gap'' between science knowledge and policy goals (e.g. \cite{RapleyD2014,KarlssonG2020,CairneyTS2023,Pearce2024})
\cite{KarlssonG2020} - characterise the gap as consquence of both science delay and policy delay
\cite{KennyRHTB2017} - In a 2017 study interviewing MPs and staff in UK parliament `` academic research is not cutting through; for example, the voluntary sector outperforms the higher education sector in terms of written and oral evidence submissions to committees in Parliament.''
\cite{KEU2021perceptions} - A study at the same time (2017) found majority of academics had low confidence about engaging with parliament

\subsubsection{power}
\cite{StrassheimK2014} tensions due to selectivity and power - ``intensive and complex struggle for political and epistemic authority on both sides; science as well as policy''
\cite{OlejniczakBDP2019} describe the s-p gap and reasons for it
\cite{CairneyO2020} - power, competition with other sources of `evidence'
\cite{CairneyO2020} - power, institutions, rules, norms
\cite{CairneyO2020} - power, networks and subsystems
\cite{CairneyO2020} - power, dominant beliefs and paradigms
\cite{CairneyO2020} - common dilemmas: engaging to provide advice versus recommendations - the divide between scientist and policymaker is necessarily blurred; coproduction versus independence; engage for influence versus engage to learn 
\cite{CairneyO2020} - inequality within engagement - gender, race, social
\cite{Carton2021} - the dominant discourse creates constraints on science: p38 knowledge production reflects existing power relations
\cite{MoallemiZHSMZHKHMGLB2023} - a feature of decision-making: power
\cite{MoallemiZHSMZHKHMGLB2023} - opportunities include: Understand and manage power dynamics with all actors
\cite{OjanenBKP2021} - forest researchers tensions and frustration include: being asked to prescribe policy by actors with more power
\cite{StoddardEtAl2021} - finds power is a thread in all 9 perspectives on why GHG emissions have risen despite wide recognition of climate change
\cite{StrassheimK2014} - ``policy-relevant facts are the result of an intensive and complex struggle for political and epistemic authority on both sides; science as well as policy ... what counts as evidence is defined by institutionally and discursively''

\cite{TurnhoutMWKL2020} - power of elite actors, including scientists, with and western biases shapes processes and citizens, communities have unequal access to process, frames already defined - voices unheard 

\subsubsection{other epistemologies}
\cite{PascualEtAl2018} - means to bridge different worldviews and values
Policymakers and scientists have different worldviews. What constitutes evidence to a policymaker can be varied from rigorously-derived evidence to individual stories (\cite{PiddingtonMD2024}).
\cite{MatukBSAHT2020} - including diverse knowledges, worldviews and values in process at IPBES
\cite{IbarraJOBCIMRS2022} - science knowledge is not neutral, tends to be situated, Western and decisions need to embrace other knowledges, especially country's civic epistemology



\subsection{SPI - organisations}
Owing to the complexities in the flow of information at the interface, boundary organisations of a range of design have been established worldwide. These include IPCC and IPBES, international organisations who are [trying really hard] to establish the policy-relevance science as well as co-produce the policy advice. These organisations provide essential knowledge for policymakers, but the policy itself is nationally- and locally-[made]. In the UK, CCC, Tyndall Centre, Hadley Centre (\cite{WesselinkH2020}) Boundary organisations are [doing an amazing job], indeed in issues other than CAN they may be doing a good enough job. However, when it comes to the gap between CAN science and policy decisions are currently deeply - and dangerously - inadequate. \unsure{is there any evidence that boundary organisations have helped in CAN at all?}
\cite{MacKillopCDD2023} - knowledge-brokerage organisations
\cite{OlejniczakBDP2019} - policy labs
\cite{VelanderD2024} - UN convention SPI

\subsection{SPI - CAN specifics}
\subsubsection{CAN issues}
\cite{IPBES2022} - Nature is deteriorating and this is happening more rapidly
\cite{IIPCC2022} - ``rapidly closing window of opportunity to secure a liveable and sustainable future for all''
\cite{TschakertEAKO2019} - so many intangible climate-related harms
\cite{McKayEtAl2022} - prospect for even a slight increase in global temperature has ``serious implications for humanity''  - urgent action needed  
\cite{WEF2024} - WEF Global Risks Report finds climate and nature risks increasing in severity to making up half of the top 10 impacts over the coming decade. 
\cite{LaybournTS2023} - transformation of societies is necessary to meet climate goals
\cite{Carrington2024} - lead authors and review editors on IPCC reports are largely pessimistic about the future
\cite{Killick2023} -  politicians have changed their perspectives a great deal, but environment is still ``tacked onto how they always thought about the economy'' this may not be so for future generations of politicians

\cite{WesselinkH2020} - with a low certainty of knowledge and poor agreement on goals, environmental crises are ``unstructured problems'' - they are characterised as agenda-changing politics, crisis management; deliberation and learning in emergent network(s) (akin to \cite{RittelW1973} wicked problems). The failure to solve them is due to not recognising this nature.
\cite[p739]{FuntowiczR1993} - post-normal science - ``remedy the pathologies of the global industrial system of which it forms the basis''

\subsubsection{imperative to engage}

\cite{CairneyTS2023} find that researchers in climate change, policy, justice and equity seek (a) transformation of systems of governance (b) mainstreaming of justice / policy integration (c) organise to overcome neoliberalism (d) design effective policy tools

\section{Roles at the science-policy interface}\label{sec:litroles}

in their study on the claims about SPI, \cite{JagannathanEtAl2023} found a majority of the researchers in the agreed with ``SPIs need intermediaries (i.e. boundary spanners) that bridge the functional divide between scientists and policy-makers''
ease the exchange of knowledge

Whilst SPIs are often suggested, many scientists engage directly with policy makers through... calls for evidence, committees, ... less formally
\info{what do we mean by scientists?}
\info{what do we mean by policymakers?}

\subsection{pure scientist}
\cite{Pielke2007} - ``no interest in the visitor’s decision-making process and simply want to share some fundamental information''
\cite{BalvaneraJNOBCDGGKKMPSSW2020} - knowledge generator
\cite{RapleyD2014} - Pure scientist - generating facts with no consideration of their use
\cite{SteelLLS2004,SinghTKMMC2014} - Reporting only - reporting science, which includes publishing their findings, presenting their work at conferences, etc.

\subsection{science communicator}
\cite{RapleyD2014} - Science communicator - engages with society to present results, offer expert interpretation and draw attention to implications
\cite{SteelLLS2004,SinghTKMMC2014} - Interpreting - interpreting science for policy makers and the public without taking a policy position (eg publishing an article in the popular media, being interviewed by a journalist, giving a public seminar).

\subsection{science adviser}
\cite{GluckmanBK2021}

\subsection{science arbiter}
\cite{Pielke2007} - ``serves as a resource for the decision-maker, standing ready to answer factual questions that the decision-maker thinks are relevant. The Science Arbiter does not tell the decision-maker what he or she ought to prefer''
\cite{RapleyD2014} - Science arbiter - answers questions posed by decision-makers
\cite{GluckmanBK2021} - Pielke describes as having a narrow technocratic focus on questions that science can answer explicitly, and does not seek to influence the direction of policy
\cite{BalvaneraJNOBCDGGKKMPSSW2020} - arbiter of knowledge generation
\cite{BednarekSHG2015} - science-policy intermediaries due to need for time and expertise to engage both science and policy, thus boundary organisation ``refraining from advocating for specific policy positions''
\cite{GregoryBW2024} - Pielke ``seeks to engage and answer factual questions and queries posed by policymakers to remove any ambiguity around the science and evidence, and therefore indirectly inform decision making, while removing their own agency from the discussion''


\subsection{honest broker}
\cite{Pielke2007} - ``expand (or at least clarify) the scope of choice for decision-making in a way that allows for the decision-maker to reduce choice'' based on preferences and values
\cite{GluckmanBK2021} - ``described by Pielke (2007) as a person or group of persons who put personal biases and values aside in order to assist policymakers in making choices between options, generally by providing clarity on the evidence ... must be trusted as neutral'' - knowledge synthesis, aligned to policy, unbiased, advice in form of options, sensitive to all forms of knowledge
\cite{GogginEtAl2015} - identified knowledge brokerage in attributes of successful practitioners - communicates across disciplines/ translates technical information so it's easy to understand/ synthesises and communicate work
\cite{RapleyD2014} - Honest broker of policy alternatives - contributes expertise to decision-making (in co-production)
\cite{SteelLLS2004,SinghTKMMC2014} - Integrating - integrating science into decision making through collaborations with policy makers without taking a policy position (eg providing expert advice, writing science briefs or white papers).

\subsection{issue advocate}
\cite{Pielke2007} - Issue Advocate - makes the case for one alternative over others
\cite{GregoryBW2024} - Pielke ``comes armed from their own analysis and value set with a preferred view and focuses on the implementation of that specific policy option, rather than offering a range of different options. They act to limit choices''
\cite{GluckmanBK2021} - issue advocate ``not generally appropriate as an institutionalized boundary function ... pursue a special agenda''
\cite{BalvaneraJNOBCDGGKKMPSSW2020} - advocating for a particular cause
\cite{ElsensohnACDGGKPRS2019} - argues for scientists to ``highlight the importance of governmental funding for basic, translational, and applied research''
\cite{KalafatisL2019} - in a graphical survey of scientists, identified 5 models of how scientists perceived their potential roles in science-policy diffusion process. Two of these (the Beacon Model and Outcast Model) could be translated as successful and unsuccessful issue advocates, highlighting .
\cite{RapleyD2014} - Issue advocate - engages with decision makers and public to promote a particular course of action, based on expert knowledge - Some see impartiality compromised by issue advocate role, others ``argue that a climate scientist's specialist knowledge, acquired at the taxpayers' expense, constitutes an obligation to speak out''

\subsection{policy advocate}
\cite{ScottRLPAFSRSS2007} - policy advocacy (``support of a particular policy or class of policies'') is present in almost all of 270 papers reviewed from journals that emphasized application of science to conservation and management - majority of survey respondents also felt papers advocated for policy preferences, and most considered that this should be included
\cite{SteelLLS2004,SinghTKMMC2014} - Taking a position - actively supporting a position on a particular issue based on scientific results (eg participating in an organized campaign, writing op-eds, contacting politicians to voice opinions, speaking at rallies or protests).
\cite{SteelLLS2004} - emerging ``integrative'' model of the role of science and scientists in the policy process (also called ``post-normal science'') ``suggests that scientists should not hesitate to make judgments that favor certain management alternatives, if the preponderance of evidence and their own experience and judgment moves them in certain practical directions. They are, after all, in the best position to interpret the scientific data and findings and thus are in a special position to advocate for specific management policies and alternatives''

\subsection{honest advocate version of policy advocate}
\cite{GregoryBW2024} - question whether it is enough for scientists to be detached `honest brokers' in policy formation and suggest honest advocate

\subsection{policy entrepreneur}
\cite{vonMalmborg2024strategies} - PE role ``introduced by Robert A. Dahl in the 1960s and popularized by John W. Kingdon in the 1980s''
\cite{Cairney2018} - policy entrepreneur
\cite{AukesLB2018} - interpretive policy entrepreneur (IPE)
\cite{CarterC2018} - FOE as PE

\cite{MacKillopCDD2023} - in KBOs: ``PEs, and to some extent PBs, assist decision-makers by helping them privilege particular ('credible') framings of policy problems and solutions''
\cite{vonMalmborg2024transport} - discover two policy entrepreneurs in EU negotiations around decarbonisation of maritime transport
\cite{vonMalmborg2024strategies} - ``policy entrepreneurs do not only impact agenda-setting, policy preferences, and policies and their outcomes, but also procedures and normative principles of liberal and deliberative democratic governance---positively or negatively, intentionally or unintentionally.''

\emph{theory and observation of roles played between science and policy - this is the core of the literature because I am aiming to understand what are the roles that could be, should be and are played, their benefits and trade-offs, and how scientists specifically inhabit these roles}\\
key theories from policy studies (e.g. Policy Entrepreneur, Problem Broker, Pielke’s roles)\\
characteristics exhibited in each role\\
conflicts between philosophical position of scientists and roles proposed at interface\\
studies identifying the roles played by scientists at the interface

, \textcite{Pielke2007}, defined four idealised roles for scientists at the interface: \emph{pure scientist} - unconcerned by policy; \emph{science arbiter} - answers questions posed by policy; \emph{issue advocate} - engages policy to promote a particular decision; and \emph{honest broker of policy alternatives} - presents all the relevant alternatives to policy derived by synthesising scientific knowledge. \textcite{RapleyD2014} add to this list \emph{science communicator}, who engage society in the conversation. Acknowledging the existential threat posed by the climate and nature crisis, \textcite{GregoryBW2024} propose the \emph{honest advocate} (advocates for a particular policy outcome whilst keep to strict criteria of honesty and transparency).  Pielke's roles are derived from the literature on Science and Technology Studies (\cite{Pielke2007}, p8). However, within Policy Studies, other roles are envisaged, particularly the \emph{policy entrepreneur} (advocates for particular proposals using a range of strategies) (\cite{Kingdon1993,Cairney2018}) and \emph{problem broker} (frames and advocates for a particular policy problem) (\cite{Knaggard2015}). Whilst these latter roles are often discussed uncritically, sometimes even [praised] for their ability to `get policy over the line', they lack legitimacy, accountability and transparency (\cite{vonMalmborg2024strategies}) and are thus [anathemic] to scientific [practice]. Thus, it may be surmised that the ability of scientists to influence policy is somewhat [hamstrung] by the constraints of [acceptable] scientific roles, constraints that do not apply to other other roles being played at the interface. [Crouzat et al. 2018 reference in \cite{BalvaneraJNOBCDGGKKMPSSW2020}) ][co producing solutions by building bridges between a range of knowledges \cite{NorstromEtAl2020} in \cite{BalvaneraJNOBCDGGKKMPSSW2020}, \cite{MatukBSAHT2020}]
credibility balanced with usefulness \cite{WesselinkH2020}
constraint of credibility does not apply to other advocacy roles that aim to influence policy makers (\cite{Kingdon1993,Knaggard2015,Cairney2018,vonMalmborg2024strategies})

\cite{SteelLLS2004,SinghTKMMC2014} - 5 roles: reporting, interpreting, integrating, taking a position, decision making

Levien 1979 identified 3 roles for scientists - clear description including uncertainties, options, contributing to problem solving (get citiation from \cite{SteelLLS2004})

\cite{ColognaKMBMO2024} scientists' advocacy for greater action may not affect credibility but advocating for a specific policy may

\cite{ColognaKMBMO2024} - important for scientists to demonstrate competence, benevolence, integrity and openness

\cite{Horton2022} - classic example of Pure Science approach at the policy interface - ``the scientists made no political recommendations, as they were there simply to present the science'' - very little, possible nothing, came of it.

\cite{FuntowiczR1993,Jasanoff2003} - in our era of \emph{post-normal science} scientists should form \emph{extended peer communities} with other affected citizens to consider impacts of their science and of policy interventions (also \cite{KalafatisL2019})

\subsection{role tensions}
For some time been recognised that a tension exists between the values and rigour espoused by scholars and the moral and political framework into which they submit their knowledge (\cite{Nau2009}, quoting \cite[p263]{Bull1972} ))
\cite{ColognaKMBMO2024} - may be nuance in advocacy``Trust in scientists not affected by their respectful advocacy for greater climate action but their credibility may be affected by advocating for a specific policy''
\cite{DablanderSCSBGGBAH2024} - ``research suggests that loss of credibility due to advocacy depends on the advocated policy''
Some have taken to the media, they feel so strongly, e.g. ``scientists misuse their authority [and undermine the credibility of climate science] if they publicise their preferred policy options'' \cite{Edwards2013}
\cite{Gerber2023} - ``contributing to policy does not have the same prestige on a tenure application as publishing high-profile research papers''

\cite{MacKillopCDD2023} - tensions in knowledge brokerage organisations: ``objectivity versus finding common ground, making recommendations, influencing and steering''

\cite{WesselinkH2020} - contrast between the sacred objectivity versus proface co-production of solutions - balancing credibility and usefulness
\cite{GregoryBW2024} - ``environmental scientists and key scientific roles in society could be best described as a Science Arbiter or Honest Broker''
\cite{GregoryBW2024} - ``a small but growing number of high-profile academics also act vocally as Issue Advocates on both climate and nature''
\cite{GregoryBW2024} - honest advocate has potential to have policy influence also potential risk to scientific credibility
\cite{GregoryBW2024} - ``the risk to scientific credibility might be equally high if scientists failed to engage actively in advocacy too, especially when they judge threats to be extreme''

\cite{JagannathanEtAl2023} - there is disagreement with ``Scientists are apolitical actors in the SPI''

\cite{Makin2024} - different languages used by climate scientists and economists, even most basic terms like risk and tolerance of uncertainty

\cite{MacKillopCDD2023} - in KBOs ``asked to play a political role by doing what they referred to as `policy-based evidence' rather than evidence-informed policymaking''

\cite{vonMalmborg2024strategies} - questions the role of PEs and how well they align to democrat norms and principles: legitimacy - input, throughput and output legitimacy; accountability; transparency and publicity; openness and equality; impartiality and fairness; justice - procedural and distributive; suggesting they are ``illiberal and antidemocratic actors using democratic institutions and functions to reach their Ultimate Goal''

Perceptions of scientists e.g. \cite{McNiePS2017} consider several science values as ``narrow'' and ``constrained''

\cite{Buntgen2024} - Argues that scientists should not be activists because they should not have prior interest in the outcome of their studies

\section{Strategies at the science-policy interface}\label{sec:litstrat}
Goal is to achieve policy impact, what that means depends on you perspective
influence policy or change behaviour

\subsection{Economics and political science models of policy engagement}
much of SPI roles work is based on these
Deficit model
MSA

Cash CRELE
\cite{Bisbal2024} - a list of needs and wants put together by a decision maker

\subsection{Psychology and behavioural science models of behaviour change}
\cite{CairneyW2017} - advises that policy science should learns from psychology and cognitive sciences ``to improve our model of choice''
\cite{AtkinsFIOPIFDCGLM2017} - Theoretical Domains Framework to understand how to intervene to generate behaviour change
\cite{HamptonW2023} - intrinsic influences are on individual, expands extrinsic influences to identify 3 further areas of influence: social, physical and political (they also reference ISM and their framework is very similar)
\cite{MoallemiZHSMZHKHMGLB2023} - opportunities include: Create trusted knowledge from multiple sources

\subsection{Specific strategies}
derived from a mix of normative and empirical studies

\subsubsection{credibility}
\cite{GluckmanBK2021} - identify constraints on scientific claims - inferential gap between knowledge and conclusions
\cite{GluckmanBK2021} - assess the evidence base (quantity and quality of available evidence)
\cite{GluckmanBK2021} - evaluate the level of `consensus'
\cite{ElsensohnACDGGKPRS2019} - ``provide clear and accurate information''
\cite{Horton2022} - report on the presentation of climate science to parliament ``the scientists made no political recommendations, as they were there simply to present the science''

\subsubsection{relevance}
\cite{Gerber2023} - ``compelling scientific results with an agency decision-maker who looked at me and said,You're answering a question I'm not asking you''
\cite{GluckmanBK2021} - consider the demand (policy) side and policy dynamics of the issue [Bisbal2024 cites McNie2007 for this]
\cite{GluckmanBK2021} - recognise the policy question, purpose and evidence need
\cite{LubchencoR2020} - increase in recent years in use-inspired science
\cite{OjanenBKP2021} - suggest that forest researchers: ensuring policy-relevance

\subsubsection{legitimacy}

\subsubsection{openness}
\cite{ElsensohnACDGGKPRS2019} - science advocates should remain issue-focussed, provide clear and accurate information, be honest about uncertainty, be aware of own and audiences values, expertise, biases and needs, and be aware of role of language
\cite{GluckmanBK2021} - communicate the uncertainties, caveats and reliability of evidence
\cite{GogginEtAl2015} - attribute of successful practitioners included Accessible/ flexible/ open/ willing to come to us
\cite{ElsensohnACDGGKPRS2019} - ``be honest about uncertainty''
\cite{OjanenBKP2021} - suggest that forest researchers: high-lighting rigorousness and transparency of research

\subsubsection{framing}
\cite{AukesLB2018} - establishing meaning through framing mechanisms
\cite{Cairney2018} - (PE) tell a good story about the policy problem
\cite{CairneyO2020} - one of the topics in the engagement literature: framing
\cite{Mintrom2019} - (PE) problem framing - Meaning is constructed through discussion with others in order to establish persuasive frames
\cite{GluckmanBK2021} - think about the framing - is the right question being asked?
\cite{ElsensohnACDGGKPRS2019} - ``be aware of role of language''
\cite{vonMalmborg2024strategies} - PEs do: ``Attention and support seeking strategies - Problem framing; idea generation. Strategic dissemination of information. Lead by example; use demonstration projects. Rhetorical persuasion; media attention. Exploitation of focusing event(s)''
\cite{MoallemiZHSMZHKHMGLB2023} - a feature of decision-making: framing
\cite{NabiGJ2018} - research into the response of individuals to different framings of climate change
\cite{OjanenBKP2021} - suggest that forest researchers: different framings of problems
\cite{DeMeyer2019} - produce communications in risk currency of policymaker
\cite{Sharpe2019} - risk assessment of non-arbitrary thresholds of impact plotted as probability over time
\cite{SharpeMVIGPKN2021} - risk-opportunity analysis instead of cost-benefit analysis

\subsubsection{self-censorship}
\cite{Pielke2007},p13-4 Member on congress suggesting ``scientists should consider self-censoring their views based on how they might be received in the political arena'' p11 post WWII era linear model of policymaking p12-3 but \cite{OjanenBKP2021} - forest researchers - self-censorship represents a conflict of interest
\cite{SimmsA2020,Carton2021,Bendell2024} - some scientists calling for a more frank, uncensored ``framing'' of climate science because ``collectively and individually, we have serially underplayed the implications of our research findings in communicating them to policymakers, the wider public and sometimes ourselves'' (\cite{CalverleyA2022})
\cite{Pearce2024} - ``scientists have not wanted to confuse or naysay policymakers looking to build public support for climate action''
\cite{ValiverronenS2021} - find self-censorship by Finnish academic scientists owing to a range of controls (political, economic, organisational, academic rivalry, by the public) 
\cite{BarTal2017} - antecedents and consequences of self-censorship in Israel-Palestine context
\cite{OjanenBKP2021} - forest researchers tensions and frustration include: requests to change or remove content of papers or talks
\cite{ReadO2017} - ``Scientists are typically concerned above all to avoid false positives, orfalse alarms'' which may explain why many scientists are relatively cautious in their claims and public pronouncements
\cite{GregoryBW2024} - note concern of some scientists (Pearson) about ``crying wolf''
\cite{PoeS2023} - Science culture of risk is very averse to false positives unlike insurance or defence who are very averse to false negatives

\subsubsection{networks and relationships}
networks and relationships at the heart of core policy science understanding such as Advocacy Coalition Framework (\cite{Dowding2018})
\cite{CairneyO2020} - a ``safe solution''
\cite{Mintrom2019} - (PE) using and expanding networks and working with advocacy coalitions
\cite{BollykyP2024} - build relationships and trust before the time of crisis in order to be effective
\cite{ArnoldNG2016} - used text mining
\cite{BoswellS2017} - suggest that collaborations of diverse perspectives are better  
\cite{GogginEtAl2015} - attribute of successful practitioners included Well connected (to universities for students and expert advice, OEH, Local Land Services,practitioners)
\cite{vonMalmborg2024strategies} - PEs do: ``Linking strategies - Coalition and team building with bureaucratic insiders and policy influencers outside of government. Issue linking. Game linking; Relational management strategies - Networking by using social acuity. Trust building.''
\cite{OjanenBKP2021} - suggest that forest researchers: building good relationships with policymakers was essential for policy relevance but also impacted ability to openly discuss critical research
\cite{SaxonbergSL2023} - establishing personal contacts with key players and being collaborative leads to more influence

\subsubsection{skills and knowledge}
\cite{BednarekSHG2015,Mintrom2019} - a range of skills that are needed to be effective at the interface
\cite{Braun2009} - study on specific engagement found that knowledge and information build powerful individuals and networks
\cite{GogginEtAl2015} - attribute of successful practitioners included Expert/ rigorous/ respected scientist/ technical skills/ publishes in scientific literature
\cite{MoallemiZHSMZHKHMGLB2023} - opportunities include: Empower learning and collaboration
\cite{MoallemiZHSMZHKHMGLB2023} - opportunities include: Link knowledge with action for transformative change

\subsubsection{evidence quality}
\cite{CairneyO2020} - a ``safe solution''
\cite{IbarraJOBCIMRS2022} - excellence, relevance, credibility, legitimacy are necessary but not suffiient precursers for taking a seat at the table in climate governance

\subsubsection{timing and venues}
\cite{CairneyW2017} - ``events ... offer opportunities and shocks''
\cite{Cairney2018} - (PE) have a solution ready to chase a problem
\cite{Cairney2018} - (PE) `surf the waves' or try to move the sea (venues)
\cite{GluckmanBK2021} - details the many ways that timing matters when engaging at the science-policy interface
\cite{vonMalmborg2024strategies} - PEs do: ``Attention and support seeking strategies - Exploitation of focusing event(s)''
\cite{vonMalmborg2024strategies} - PEs do: ``Arena strategies - Venue shopping. Timing.''
\cite{CairneyO2020} - two of the topics in the engagement literature: values and events
\cite{MoallemiZHSMZHKHMGLB2023} - a feature of decision-making: timing

\subsubsection{values}
\cite{GeuijenMCRv2017} - suggest a means to hold different perspectives on public value
\cite{GregoryBW2024} - ``acknowledge and communicate values''
\cite{ElsensohnACDGGKPRS2019} - ``be aware of own and audiences values, expertise, biases and needs''
\cite{VoisardW2023} - practical climate ethics, how to decide on way forward
\cite{PascualEtAl2018} - need to recognise different worldviews and values
\cite{RogeljLPLWXXXX} - suggest using principles

\subsubsection{communication}
\cite{LubchencoR2020} - science communication via social media 
\cite{vonMalmborg2024strategies} - PEs do: ``Attention and support seeking strategies - Problem framing; idea generation. Strategic dissemination of information. Lead by example; use demonstration projects. Rhetorical persuasion; media attention. Exploitation of focusing event(s)''

\subsubsection{self-reflection}
\cite{OjanenBKP2021} - suggest that forest researchers: self-reflection, humbleness

\section{Influences at the SPI}

\subsection{principles}

\subsection{norms}

\subsection{rules}

\subsection{beliefs}
\cite{CairneyW2017} - ``ideas or beliefs that dominate the ways in which [policymakers] think about problems and solutions''
\cite{BalvaneraJNOBCDGGKKMPSSW2020} - highlight that scientists' belief in the superiority of scientific knowledge causes a departure from recognised roles for scientists

\subsection{emotions}
\cite{RandallH2019} - climate scientists ``face the disturbing reality of climate change on a regular basis'' may experience strong emotions as a consequence of their own scientific understanding and ``confusion'' about the responses they experience when sharing their knowledge
\cite{Pivovarchuk2024} - al jazeera article about scientists' grief
\cite{Carrington2024} - guardian article about scientists' despair
\cite{Makin2024} - Emotion plays a role in shaping how a concept is represented

\subsection{organisations and institutions}
\cite{CairneyW2017} - ``formal and informal institutions that guide [policymakers'] actions at each level''
\cite{BalvaneraJNOBCDGGKKMPSSW2020} - power imbalances between academic disciplines (social v natural sciences) unable to fully appreciate each other
\cite{GeuijenMCRv2017} - 
\cite{SaxonbergSL2023} - institutional setting affects experiences and outcomes
\cite{StrassheimK2014} - ``what counts as evidence is defined by institutionally and discursively''
\cite{WesselinkH2020} - BO's success depends on: role of the relevant boundary organization and its activities

\subsection{politics and governance}
\cite{CairneyW2017} - ``multiple policymakers and influencers spread across levels and types of government''
\cite{GeuijenMCRv2017} - lack of authorising environment at the global scale but also the global and local civil society can influence national states and international organisations
\cite{MoallemiZHSMZHKHMGLB2023} - a feature of decision-making: politics/governance
\cite{MoallemiZHSMZHKHMGLB2023} - opportunities include: Account for politics and inclusive governance
\cite{SaxonbergSL2023} - political context affects experiences and outcomes
\cite{StrassheimK2014} - ``evidence is neglected and distorted because it conflicts with political values and ideologies (normative selectivity) or it is ignored and misinterpreted as a result of limited politico-administrative perception (cognitive selectivity)''
\cite{WesselinkH2020} - BO's success depends on: (inter)national political culture 

\subsection{effort}
\cite{BednarekSHG2015} - ``efforts required in the enterprise of connecting science and policy can often exceed the skill sets or time constraints of individual scientists

\subsection{funding}
\cite{OjanenBKP2021} - forest researchers tensions and frustration include: impact on research of funding and funders

\subsection{policy area}
\cite{SaxonbergSL2023} - some policy areas are more ideologically charged than others - [CAN is likely to be highly charged] affecting experience and outcomes 
\cite{WesselinkH2020} - framing of the policy problem (defines the need for science)
\cite{WesselinkH2020} - BO's success depends on: culture of the policy domain
\cite{WesselinkH2020} - BO's success depends on: characteristics of the policy problem itself



\section{Measuring impact}
\cite{BednarekSHG2015} - measuring impact is difficult in any complex setting
REF probably the most well know structured approach to measuring policy impact of research.
\cite{KEU2021impact} - for REF 2021 policy impact is a combination of ``reach'' and ``significance'' via direct and indirect contact with policymakers, admits that this is difficult to evidence
\cite{BoswellS2017} - REF assumes linear model (and can adversely affect strategies such as building collaborations)
\cite{Cairney2018} - REF builds on the linear model
\cite{CairneyO2020} - ``more challenging issues ... arise when we consider what it would take to secure real, long term policy impact with evidence''
\cite{JagannathanEtAl2023} - defining and evaluating success in creating actionable knowledge remains an area of active research

When trying to describe the gap between science and policy it is valuable to be able to articulate what close that gap would look like\unsure{what do writings about the science policy gap want to see improved?}
In the arena of CAN policy, the clearest measure would be if policy ambition met with physical modelling of the need to close the gap between budget and emissions. Other measures would be to reach zero habitat loss? 

\cite{BednarekSHG2015} - section 4.4 Measuring impact is difficult in any complex setting 
\cite{JagannathanEtAl2023} - section 4.1.1 identify that ``How do we define and evaluate success in producing actionable knowledge?'' needs deeper research as the definition and methodologies are very context-dependant
\cite{KEU2021impact} - its difficult to define impact, for REF 2021 it comes down to reach and significance

\section{Empirical studies}
\cite{OjanenBKP2021} - forest scientists and the tensions they experience at the interface
\cite{vonMalmborg2024transport} - positions, framings, policy windows present in negotiations around EU transport regulation
\cite{MacKillopCDD2023} - interview staff at knowledge brokering organisations to understand tensions, strategies and characteristics
\cite{KennyRHTB2017} - experience of MPs and parliamentary staff
\cite{KalafatisL2019} - graphical survey of geoscientists of their perceptions of engagement with policy
\cite{IbarraJOBCIMRS2022} - action research with Chilean scientists within a climate governance organisation
\cite{ElsensohnACDGGKPRS2019} - survey of former fellows of ESA science policy initiative to discover their concerns about advocacy
\cite{DesikanC2023} - survey of scientists in US federal government finds that they are still self-censoring - experience not changed much since the Trump administration
\cite{DanfordDR2009} - experiences of a GocSci department (probably met office) to understand their experience of public sector reform
\cite{DablanderSCSBGGBAH2024} - survey of scientists in a range of fields to understand their attitudes and barriers to advocacy and protest
\cite{BednarekSHG2015} - case studies from one science-policy intermediary organisation to understand what they have learned
\cite{SaxonbergSL2023} - editorial for special issue on the experiences of social scientists in government
\cite{SteelLLS2004} - surveyed ecological scientists, natural resource managers, representatives of public interest groups, and the attentive public in the context of the Long Term Ecological Research Programe, in the American West
\cite{VelanderD2024} - interviews with members of UNCCD SPI to derive framework comprising institution design, boundary work and windowns of opportunity

\section{Justification for this work}
Work on how scientists can engage with policy has often been rather normative, describing the needs of policymakers and how scientists must align to these (e.g. \cite{McNie2007,Bisbal2024}). 

\cite{CairneyO2020} - ``reject the idea that political scientists can draw on generally applicable `how to' advice''
\cite{CairneyO2020} - scholars face major dilemmas when engaging with policy
\cite{CairneyTS2023} - find that in the climate justice literature there is little use of policy theory to seek practical lessons such as how to advocate for policy change  - `` theory-informed studies do not solve the problems raised by climate justice scholars, they provide the concepts or language to help identify patterns and mechanics of policy change'' [so whilst not using theory - are they applying it in practice?]
\cite{JagannathanEtAl2023} - boundary spanning ``roles and their associated practices, expectations, institutional supports, and capacities vary greatly ... empirical analyses of such boundary functions are far and few ... also limited examination of the risks they may face while engaging in such a crucial role''

\cite{KennyRHTB2017} - ``examined only the parliamentary demand-side of the research-to-use ecosystem. Equally important are the supply-side researchers and the many intermediaries who translate research for parliamentary use''

There is a great deal more work prescribing the ideal roles and behaviours of scientists engaging with policy than there is of their experiences of doing so. Like the issues underlying the CAN crises themselves, it is not enough to simply provide instructions. First we need to understand experiences and the influences over behaviours. This study aims to initiate this first step...

\section{text dump}
\cite{LubchencoR2020} - in 1998 Lubchenco called for greater effort by scientists to make science accessible, understandable, relevant and credible - whilst much has been achieved in 2 decades, still barriers, largely due to the nature of the institutions of science
\cite{Gerber2023} - scientists should answer the questions posed by policy but what if policy is not asking the right questions? e.g. Emphasis of policy on NET ValiverronenS2021, CalverleyA2022, Carton2021, CartonHML2023 rather than behaviour and society
\cite{Bisbal2024} - expresses the frustration of the policymaker unable to gain the knowledge they require and suggests actions that scientists can take 


the issues of credibility, objectivity, etc. To an extent disadvantage scientists. Voluntary sector outperforms high education KennyRHTB2017. CernaTHTTS2020 said that ``Greta Thunberg has been more successful in communicating the globally agreed scientific facts concerning climate change that the thousands of scientists actively involved in the International Panel on Climate Change'' yet Greta was not bound by the restrictions of scientific practice. Yet, policymakers are also influenced by actors playing issue and policy and advocacy roles that may have much less legitimacy, accountability and transparency (\cite{Kingdon1993,Knaggard2015,Cairney2018,vonMalmborg2024strategies}) , potentially putting scientists at an influential disadvantage. Voluntary sector outperforms higher education KennyRHTB2017. studies looking mainly at the experience of the demand-side / policy
LubchencoR2020 - in 1998 identified that science is not delivering on its social contract it is not accessible, understandable, or seen as relevant, or credible and thus not 
POST study found underperformance of academia in contribution to uk parliament. Areas for academia to improve: communication, understanding of parliamentary processes, research relevance, credibility KennyRHTB2017  
communicate the impacts that are meaningful Sharpe2019
GluckmanBK2021 make 8 recommendations for effective knowledge brokerage
SomervilleH2011, Makin2024 - scientists use language that had a different meaning in society 
roles played to advocate for CAN policy - FOE as PE in CarterC2018 - NGOs as PE in Braun2009

Of course, the policy maker plays their role be selecting the knowledge to be used. W  Decisions makers can choose to ignore evidence (e.g. \cite{TennoyHLN2016}), rarely needing to justify what they include and what they leave out\footnote{not the case with EU Climate Change Advisory Board \cite{WardmanE2023}}.


Also, less well documented\unsure{is it?} are the power influences at play, that policy, even policy based on scientific understanding, such as the nature of climate change, is [pertinent to] other factors such as commercial interests, concerns about social acceptance of transition [activities] [etc]. [supranational nature of climate governance, colonialism and imperialism in geopolitics, power relations in national governance and decision-making, lobbying by vested interests, dominance of economics and financial interests in governance, etc.]

scientists instead turn to activism \cite{Pivovarchuk2024,GregoryBW2024}

scientist learns to answer the questions that policy asks \cite{Gerber2023}

\emph{foundation of this thesis is the belief that this gap is a problem} \\
the nature of the gap(s) and their consequences

(Suggested causes of the gap(s))
\emph{overview of work giving reasons for the gap - not too detailed but these concepts are useful for the rest of the literature review} \\
use same order/topics as in introduction\\
science and policy, two very different philosophies\\
communication of science to non-scientists\\
selection of science by policy-makers\\
post-normal science\\
external influences

\subsection{A wicked problem}
Wicked problems \cite{RittelW1973}
\cite{WesselinkH2020} - unstructured problem - low certainty of knowledge (social science) low agreement on goals

, particularly given the `wicked' nature of the issues and their possible solutions (\cite{Cairney2016}, p94). 

[Thus, to date, whilst UK can demonstrate some strong statitics c.f. other countries in terms of decarbonisation of electricity grid [refs], these transitions have been largely down to technological changes and have required little individual or societal change. Further decarbonisation of UK economy will mostly now require changes that affect individuals and their mobility, homes, and leisure habits.]

\subsection{Two cultures}
These messier interactions result, at least in part, from the contrasting ``cultures'' of science and policy \cite{Obermeister2022}. Whereas as science values objectivity most highly, policy values decisiveness\unsure{policy - decisiveness?}. Where science sets a very high bar for what is considered ``evidence'', ``Policy is founded on a plurality of knowledge''\unsure{check quotes} (\cite{GluckmanBK2021}, e.g. \cite{PiddingtonMD2024}). Science (particularly the environmental sciences) prefers to only infer the most likely scenarios, policy (founded on economic projections) is comfortable considering worst-cases ([PoeS2023, Pearson reference in GregoryBW2024: crying wolf, ReadO2017]). There are linguistic differences between the two cultures whereby the same word (such as ``uncertainty'', ``risk''  and ``error'') have different meanings (\cite{SomervilleH2011,Makin2024}[De Meyer,]). Scientists can even be reluctant to make confident statements about what they know, preferring instead to talk about what they don't know (\cite{MountfordD2023}). These different cultures serve the purposes of pure science and policymaking well, but create barriers to effective conveyance of knowledge into policy and decision-making.


\subsection{science policy contexts}

understood now that context of policymaking is relevant \cite{CairneyW2017} and models of these contexts have been devised \cite{WeyrauchES2016}

The science related to CAN policy is extremely varied. Policy can require knowledge about the measurement, mitigation of and adaptation to the changes that humans are causing to climate and nature. Relevant science includes the methods for measurement and ongoing monitoring of the natural world and human infrastructure, technologies for ghg emissions reduction and drawdown, technologies to reduce impacts from climate destabilisation and nature depletion, and behavioural and social science to transform individual and societal activities to reduce ghg emissions and adapt to a changing environment. technology, physical, social science

\cite{DanfordDR2009} context of government body scientists

\cite{BalvaneraJNOBCDGGKKMPSSW2020} - Power imbalances between academic disciplines

\subsection{Scientists’ experiences of policy}\label{sec:experiences}
\emph{review of literature describing the experience of scientists - I believe my study is one of very few looking at science-policy specifically from the scientists’ perspective}
\cite{OjanenBKP2021} deep investigation of experiences of forest researchers  - include more detail from this

\cite{DanfordDR2009} - experiences of working on government science from the perspective of management, nature of work, ... but not directly involved in policy

\cite{KothariME2009} - study of health researchers' experiences with public policy




