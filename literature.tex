\chapter{Literature Review}\label{ch:lit}

\section{Gaps between climate science and climate policy}\label{sec:gaps}
\emph{foundation of this thesis is the belief that this gap is a problem} \\
the nature of the gap(s) and their consequences

(Suggested causes of the gap(s))
\emph{overview of work giving reasons for the gap - not too detailed but these concepts are useful for the rest of the literature review} \\
science and policy, two very different philosophies\\
communication of science to non-scientists\\
selection of science by policy-makers\\
post-normal science\\
external influences

\section{The science-policy interface}\label{sec:interface}
\emph{normative and descriptive literature on the how science and policy interact - again for foundation and to provide framework for the contexts that should arise from the study}\\
models of science-policy relations - the standard model of evidence influencing policy and work that challenges this\\
examples of science-policy interfaces in climate policy, e.g. institutions, projects

\section{Roles at the interface}\label{sec:roles}
\emph{theory and observation of roles played between science and policy - this is the core of the literature because I am aiming to understand what are the roles that could be, should be and are played, their benefits and trade-offs, and how scientists specifically inhabit these roles}\\
key theories from policy studies (e.g. Policy Entrepreneur, Problem Broker, Pielke’s roles)\\
characteristics exhibited in each role\\
conflicts between philosophical position of scientists and roles proposed at interface\\
studies identifying the roles played by scientists at the interface

\cite{ColognaKMBMO2024} scientists' advocacy for greater action may not affect credibility but advocating for a specific policy may

\cite{ColognaKMBMO2024} - important for scientists to demonstrate competence, benevolence, integrity and openness

\cite{Horton2022} - classic example of Pure Science approach at the policy interface - ``the scientists made no political recommendations, as they were there simply to present the science'' - very little, possible nothing, came of it.

\section{science policy contexts}
Science

\cite{BalvaneraJNOBCDGGKKMPSSW2020} - Power imbalances between academic disciplines

\section{Scientists’ experiences of policy}\label{sec:experiences}
\emph{review of literature describing the experience of scientists - I believe my study is one of very few looking at science-policy specifically from the scientists’ perspective}

\section{Methods literature}\label{sec:methodlit}
\emph{the literature relevant to how I will discover scientists’ experiences}\\
discovering contexts using inductive coding - the contexts in which scientists engage with policy have many dimensions (e.g. actors, environments, impact type, stage in policy cycle, etc.). Within my small study, and possibly across the arena of science for climate policy, it is probable that engagement occurs in only a few subcontexts (e.g. providing evidence (impact type) to parliamentary committee members (actors) as part of the agenda setting process (policy cycle))\\
discovering roles using behavioural analysis - given the conflicts between science and policy, which can be stronger even than other areas of expertise (economics, medicine,) I will not assume specific theoretical roles are played and instead perform a comprehensive analysis of the behaviours exhibited by scientists

\cite{LubchencoR2020} - in 1998 Lubchenco called for greater effort by scientists to make science accessible, understandable, relevant and credible - whilst much has been achieved in 2 decades, still barriers, largely due to the nature of the institutions of science
\cite{Gerber2023} - scientists should answer the questions posed by policy but what if policy is not asking the right questions? e.g. Emphasis of policy on NET ValiverronenS2021, CalverleyA2022, Carton2021, CartonHML2023 rather than behaviour and society
\cite{Bisbal2024} - expresses the frustration of the policymaker unable to gain the knowledge they require and suggests actions that scientists can take 