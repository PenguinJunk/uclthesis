\chapter{Literature}\label{ch:lit}

\section{The Climate and Nature Science-Policy Interface}\label{sec:litspi}

The intersecting crises of a destabilising climate (\cite{IIPCC2022}) and deteriorating nature (\cite{IPBES2022}) are already destroying people's lives and livelihoods (\cite{TschakertEAKO2019,SpaiserEtAl2024}) and have dire implications for the future of human civilisation (\cite{McKayEtAl2022,WEF2024,SpaiserEtAl2024}). Given the gravity of these crises, scientists researching \CAN{} may feel an imperative to engage with public policy, hoping to influence decision-making. Available to these scientists is advice on the roles (Section~\ref{sec:litroles}) and practices (Section~\ref{sec:litpractices}) that can they can use to engage with policy, as well as some discussion of how to influence decision-making (Section~\ref{sec:litinfluencing}). %This advice and the underlying research and discussions, are part of a broad effort to close the ``gap'' identified between scientific knowledge and policy goals (e.g. \cite{RapleyD2014,KarlssonG2020,CairneyTS2023}).

Science and policy exchange knowledge through institutions and mechanisms that are often conceptualised as the \SPI{} (\cite{JagannathanEtAl2023}). 
%processes of scientific enquiry and public policy decision making. 
Traditionally, this interface enables %In a traditional  form, %\emph{the rational model}, 
evidence derived from scientific enquiry to inform policy decisions. Thus science is often considered to \emph{supply} knowledge in response to \emph{demand} from policy. 
In reality, the \SPI{} comprises an ``erratic flow of information'' (\cite{BednarekSHG2015}) that is constantly evolving with changing values, demands and understanding of science, policy and society (\cite{Obermeister2020}). 
This complexity is compounded by the contestations that specifically arise around \CAN{} issues.  
%Such guidances, and their underlying research and discussion, are well-intended efforts to close the ``gap'' that is widely recognised between science knowledge and policy goals (e.g. \cite{RapleyD2014,KarlssonG2020,CairneyTS2023}). %Some of the greatest consequences of this gap is the delay in policy, and thus action, related to the worsening \CAN{} crises. These intersecting crises of a destabilising climate (\cite{IIPCC2022}) and deteriorating nature (\cite{IPBES2022}) and are already threatening and undermining the lives and livelihoods of people in a myriad of guises (\cite{TschakertEAKO2019,SpaiserEtAl2024}) and the implications for humanity as a whole are dire (\cite{McKayEtAl2022,WEF2024,SpaiserEtAl2024}). 
The disputes and uncertainties of \CAN{} science, coupled with the urgency and high stakes of \CAN{} policy have been characterised as \PNS, requiring new approaches to decision-making that integrate a diversity of knowledges, sometimes known as \emph{extended peer review} (\cite{FuntowiczR1993,Ravetz1999,Jasanoff2003,Hewitt2024}). 
Yet, while policymakers have changed their perspectives on \CAN{} a great deal (\cite[p118]{Killick2023}), there has been a failure to recognise and respond to the unstructured and emergent nature of \CAN{} crises (\cite{FuntowiczR1993,WesselinkH2020}), which instead are ``tacked onto how they always thought about the economy'' (\cite[p118]{Killick2023}). 

The literature about engaging with policy is at odds with that conceptualising the \CAN{} \SPI. Whilst, the \emph{supply}-\emph{demand} model of the \SPI{} has been ``widely debunked'' (\cite{BoswellS2017}), it is nevertheless implied in advice for scientists to align their actions to the needs of policy (e.g. \cite{McNie2007,GeddesDP2018,BlessenohlS2022,Bisbal2024}). When it comes to \CAN{} science, the implication of \PNS{} is not so much the need to align science to policy, but to align the \SPI{} to the needs of the \CAN{} crises (\cite{BalvaneraJNOBCDGGKKMPSSW2020}). These tensions raise questions about the experiences of scientists who engage at the \CAN{} \SPI{} and how those experiences compare to the advice literature. To understand better what is currently expected of scientists engaging with policy, the following sections review the advice literature regarding the roles that scientists can play and the practices that they can use, as well as literature describing how to influence and create change.

%\textcite{vonSchneidemesserMS2020} reframe it as a system


%This has not gone unnoticed, \SPI{} boundary organisations have been established worldwide within which scientists engage directly on science-related policy (\cite{WesselinkH2020}), including international bodies such as the Intergovernmental Panel on Climate Change, Intergovernmental Science-Policy Platform on Biodiversity and Ecosystem Services, Climate Change Committee and Joint Nature Conservation Committee. 
%For instance the \IPCC{} and the \IPBES{} are international organisations tasked with establishing policy-relevant science, drawing on wider knowledges and values (e.g. \cite{PascualEtAl2018,MatukBSAHT2020})  and producing the policy advice. Policy itself is made nationally and locally, requiring the relevant knowledge to be convened in these contexts. In the UK, the \CCC{} is the independent statutory body for advising the UK and devolved governments on climate-related policy and the \JNCC{} is the statutory nature adviser for UK and devolved governments. Outside of boundary organisations, 

\section{Roles at the science-policy interface}\label{sec:litroles}

Many researchers studying the processes, practices and pathways by which knowledge meaningfully informs policy, consider that \SPI s need prescribed intermediaries between scientists and policymakers (\cite{JagannathanEtAl2023}). Thus, a number of boundary-spanning roles are defined within the literature, summarised in Table~\ref{tab:litroles}. These roles embody different values, objectives and practices for knowledge creation and exchange with policymaking, but largely imply a one-way transfer of knowledge from science into policy, as exemplified by \textcite{SteelLLS2004}'s \emph{reporting}, \emph{interpreting}, \emph{integrating} roles. The work of Roger A. Pielke, Jr. has been highly influential in describing an idealised set of roles at the \SPI: \emph{Pure Scientist}, \emph{Science Arbiter}, \emph{Issue Advocate} and \emph{Honest Broker of Policy Alternatives} (\cite{Pielke2007}), with emphasis on the scientist's credibility and the science's legitimacy (\cite{DuncanRE2020}). Several authors have further developed this set of roles over recent years (e.g. \cite{RapleyD2014,DuncanRE2020,GluckmanBK2021,GregoryBW2024}) as outlined in Table~\ref{tab:litroles}. %The table also includes a summary of a further role, the \emph{Policy Entrepreneur}, probably the best known policy intermediary, although not necessarily one that interfaces with science.

\begin{table}[!ht]
\footnotesize
\caption{Roles of relevance to the \SPI}\label{tab:litroles}
\begin{tabular}{L{.48\linewidth}L{.30\linewidth}L{.22\linewidth}}  \hline
\textbf{description} & \textbf{similar roles} & \textbf{empirical examples} \\ \hline \hline
\multicolumn{3}{c}{\textbf{\small Pure Scientist}} \\ 
Generates and shares knowledge without consideration for its use (\cite{Pielke2007,RapleyD2014}). & Knowledge Generator (\cite{BalvaneraJNOBCDGGKKMPSSW2020}) & \textcite{SteelLLS2004,SinghTKMMC2014} \\ \hline
\multicolumn{3}{c}{\textbf{\small Science Communicator}} \\ 
Engages with society offering expert interpretation and drawing attention to the implications (\cite{RapleyD2014}) & Interpreting (\cite{SteelLLS2004}). & \textcite{SteelLLS2004,SinghTKMMC2014} \\ \hline
\multicolumn{3}{c}{\textbf{\small Science Arbiter}} \\ 
Using scientific findings to answer questions that the policymaker thinks are relevant without expressing opinion (\cite{Pielke2007,GluckmanBK2021}). & Synthesiser (\cite{KarkkainenLKK2024}) &  \\ \hline
\multicolumn{3}{c}{\textbf{\small Science Adviser}} \\ 
Scientists called on to synthesise the existing science and interpret it for the policy context. \textcite{Obermeister2020} describes how this role is a process of learning and becoming increasingly expert and thus can be considered the transition between Science Arbiter and Knowledge Broker (\cite{Obermeister2020,GluckmanBK2021}). & Interpreting (\cite{SteelLLS2004}) & \textcite{SteelLLS2004,SinghTKMMC2014,Obermeister2020} \\ \hline
\multicolumn{3}{c}{\textbf{\small Knowledge Broker}} \\ 
Translates knowledge to the given setting and communicates it across disciplines (\cite{GogginEtAl2015}) contributing their own expertise (\cite{RapleyD2014}), including knowledges other than those of science and policy (\cite{Gluckman2014}). & Honest Broker of Policy Alternatives (\cite{Pielke2007}) Commentator (\cite{KarkkainenLKK2024}) Integrating (\cite{SteelLLS2004}) & \textcite{SteelLLS2004,SinghTKMMC2014,BednarekSHG2015} \\ \hline
\multicolumn{3}{c}{\textbf{\small Advocate}} \\ 
Advocates for attention to be paid to a particular cause (\cite{KarkkainenLKK2024}). Scientists with expert knowledge about the cause may use advocacy depending on their values (\cite{RykielEtAl2002,RapleyD2014,ElsensohnACDGGKPRS2019}).  & Issue Advocate (\cite{Pielke2007}) Policy Advocate (\cite{ScottRLPAFSRSS2007}) Taking a Position (\cite{SteelLLS2004}) Honest Advocate (\cite{RoseBOP2018,GregoryBW2024}) %Stealth Advocate (\cite{Pielke2007})
& \textcite{SteelLLS2004,ScottRLPAFSRSS2007,SinghTKMMC2014,RoseBOP2018,DablanderSCSBGGBAH2024} \\ \hline
%\multicolumn{3}{c}{\textbf{\PE}} \\ 
%Focused on influencing policy using a range of means (\cite{Kingdon1993}) that are not necessarily derived from objective evidence, raising questions of legitimacy, accountability, justice (\cite{vonMalmborg2024strategies}). Therefore, this is not recommended for scientists but of relevance to scientists because of their presence at the \SPI. & Problem Broker (\cite{Knaggard2015}) Interpretive Policy Entrepreneur (\cite{AukesLB2018}) & \textcite{CarterC2018} (Friends of the Earth as a \PE) \textcite{MintromL2017} (\PE s in climate policy)  \\
%\hline
\end{tabular}
\end{table}

These roles are not without major dilemmas (\cite{NelsonV2009,CairneyO2020}). As will be seen in the next section, advice to scientists about \SPI{} engagement includes a recommendation to choose between advocacy and advice/brokerage (\cite{OliverC2019}), or to avoid advocacy altogether (\cite{Lackey2007,Gluckman2014}). This is because there remains some debate about the impacts of advocacy on the trust in, and the credibility of, science and scientists (\cite{Edwards2013,GregoryBW2024,ColognaKMBMO2024,DablanderSCSBGGBAH2024,RykielEtAl2002}). Although, there is increasing evidence of the involvement of \CAN{} scientists in a range of advocacy activities (\cite{ColognaKMBMO2024,DablanderSCSBGGBAH2024,ScottRLPAFSRSS2007}). Even when avoiding advocacy, advisers and brokers can experience tensions between the ``sacred'' objectivity of science and the ``profane'' of co-producing solutions (\cite{WesselinkH2020,MacKillopCDD2023}) and reconciling the disputed and uncertain scientific knowledge with the desire of policy for consensus and certainty (\cite{Stirling2010,Hicks2024}). This is not simply frustrating, the reputations of scientists, and science itself, can be damaged when nuance is insufficiently communicated (\cite{Stirling2010,OjanenBKP2021}).

%To some, advocacy, whether for an issue or a particular policy, is seen by some as undermining the credibility of science, or at least the scientist (\cite{Edwards2013}). \citeauthor{Pielke2007} warned about the \emph{Stealth Advocate} who uses their credibility as a scientist to argue for a particular cause, even when the science is not settled (\cite{CardouV2023,GluckmanBK2021}). However, others have argued that the urgency and scale of our \CAN{} challenges are trivialised when scientists do not take a strong position on political action (\cite{GregoryBW2024}). The choice to advocate is often left to individual scientists to make based on their own values (\cite{RykielEtAl2002}). A survey by \textcite{DablanderSCSBGGBAH2024} and a narrative review by \textcite{ColognaKMBMO2024} both find that there are nuances to the affect on trust in science and scientists of their advocacy that depend on what is being advocated for, and the audience perceiving the advocacy. Significantly, both papers note the engagement of thousands of scientists in some aspect of climate campaigning, and a review of 270 conservation articles identified that policy advocacy was present in almost all of them (\cite{ScottRLPAFSRSS2007}).  

%Even choosing advice or brokerage can create tensions. \textcite{WesselinkH2020} contrasts the ``sacred'' objectivity of science with the ``profane'' of co-producing solutions. Empirical evidence points to scientists experiencing pressure to produce evidence that aligns with a particular policy perspective (\cite{MacKillopCDD2023}) and personal accounts from science advisers articulate tensions between the desire of policymakers for certainty and consensus, and the disputed and uncertain nature of scientific knowledge  (\cite{Stirling2010,Hicks2024}). Such pluralities of perspective may be difficult to reconcile in policy settings that demand communications in a condensed form. This is not simply frustrating, the reputations of scientists, and science itself, can be damaged when nuance is insufficiently communicated (\cite{Stirling2010,OjanenBKP2021}). The work of producing scientific evidence is held to a rigorous set of values that are in tension with the ``moral and political framework'' into which this evidence is submitted (\cite{Nau2009}, quoting \cite[p263]{Bull1972}).

\section{Practices at the science-policy interface}\label{sec:litpractices}

The evidence is that scientists are not confident about engaging with policy (e.g. \cite{KEU2021perceptions}), despite their input being recognised as ``vital'' (\cite{KennyRHTB2017}). Accordingly, policymaking functions have produced a range of documents that define and describe principles and practices of science-policy intermediation (e.g. \cite{OECD2015,DottiACDMPSVW2024,KarkkainenLKK2024}). There is also a great body of Policy Studies literature theorising the behaviours of individuals, groups and processes in policymaking (e.g. \cite{Kingdon1993,Hajer2005,Dowding2018}), with \textcite{KernR2018} performing an excellent analysis of how the main 5 theories in this literature (Advocacy Coalitions, Multiple Streams Approach, Punctuated Equilibrium Theory, Discourse Coalitions, Policy Feedback Theory) apply to sustainability transitions. This literature provides theories about practices that improve outcomes of engagement at the \SPI{} (e.g. \cite{RykielEtAl2002,McNie2007,Gluckman2014,BlessenohlS2022}). However, this is not widely used in research into climate policy (\cite{CairneyTS2023}) and empirical evidence about the use and success of practices at the \SPI{} is harder to come by (\cite{JagannathanEtAl2023}). \textcite{OliverHBGC2022} review the empirical evidence of how research impacts policy, but find it lacks rigour. 

\begin{table}[!ht]
\footnotesize
\caption{Tips for engaging at the \SPI{} after \textcite{OliverC2019}, with some examples of where they have been empirically observed}\label{tab:litpractices}
%\begin{tabular}{L{.22\linewidth}L{.56\linewidth}L{.22\linewidth}} 
\begin{tabular}{L{.68\linewidth}L{.32\linewidth}} \hline
\textbf{description} & \textbf{empirical examples}\\ \hline \hline
\multicolumn{2}{c}{\textbf{\small Do high quality research}} \\
Research should be relevant to policy and rigorously performed, determining uncertainties, strengths and weaknesses. 	 & 	\textcite{RoseBOP2018,OjanenBKP2021,IbarraJOBCIMRS2022} \\ \hline
\multicolumn{2}{c}{\textbf{\small Effectively communicate research}} \\
Policymakers need research communicated in ways that are credible, accessible and meaningful, such as by \emph{framing} research in terms of a particular policy priority. & 	\textcite{RoseBOP2018,Obermeister2022} \\ \hline
\multicolumn{2}{c}{\textbf{\small Understand policy processes}} \\
It is helpful to know that different types of evidence are used in decision-making, what the different policymaking roles are of different bodies, and how the priorities and opportunities of policy change. & 	\textcite{Obermeister2022} \\ \hline
\multicolumn{2}{c}{\textbf{\small Be accessible to policymakers}} \\
Scientific advice is often required at specific times to address specific questions.	 & 	\textcite{GogginEtAl2015} \\ \hline
\multicolumn{2}{c}{\textbf{\small Decide if you want to be an issue advocate or honest broker}} \\
Many commentators find the two roles incompatible with each other and recommend being open about advocating for particular causes. & \textcite{ScottRLPAFSRSS2007}\\ \hline
\multicolumn{2}{c}{\textbf{\small Build relationships with policymakers}} \\
Engagements will often arise through network connections. Trust and shared understandings are essential under pressurised situations but take time and effort to develop.	 & 	\textcite{OjanenBKP2021,IbarraJOBCIMRS2022,SaxonbergSL2023} \\ \hline
\multicolumn{2}{c}{\textbf{\small Be `entrepreneurial' or find someone who is}} \\
The opportunities to establishing new relationships and engage with policymakers may be unusual or fleeting, suggesting a need to be ready when a \emph{Window of Opportunity} opens. This may be too much effort for an individual scientist and collaboration with others is recommended.	 & 	\textcite{RoseBOP2018} \\ \hline
\multicolumn{2}{c}{\textbf{\small Reflect continuously}} \\
The science-policy and personal contexts require different approaches and decisions on whether to engage. & 	\textcite{OjanenBKP2021,Obermeister2022} \\
\hline
\end{tabular}
\end{table}

Combining the Policy Studies theoretical and empirical literatures, along with `how to' advice, \textcite{OliverC2019} synthesised 8 tips for engaging with policy, summarised in Table~\ref{tab:litpractices}. These 8 tips are a broad set of practices that scientists in roles at the \SPI{} can apply, such the advice to choose whether or not to be an advocate, being effective at communication and accessible to policymakers, and building networks. They also highlight some useful concepts, such as \emph{Problem Framing}, which is used define the nature of an issue and thus determine the relevant domains of knowledge (\cite{OECD2015,MoallemiZHSMZHKHMGLB2023}), and \emph{Window of Opportunity} (\cite{Kingdon1993}), when it may be possible to cast new ideas into the policy arena (\cite{RoseBOP2018}). The 8 tips also reflect the observation that advice to scientists is to conform to the policymaking process. As such, many of these are ``safe solutions'' that belie several tensions experienced by those engaging with policy (\cite{CairneyO2020}) such as the how political context (\cite{SaxonbergSL2023,WesselinkH2020}) and power imbalances (\cite{TurnhoutMWKL2020,OjanenBKP2021,StrassheimK2014}) shape what knowledge may or may not be admissible; the pressure to censor in order to remain admissible (\cite{Pearce2024,OjanenBKP2021}); and, particular to the natural sciences, the aversion to raising ``false alarms'' (\cite{ReadO2017,PoeS2023}) which may instead underplay the implications of \CAN{} scientific findings (\cite{CalverleyA2022}). The 8 tips, like much of the Policy Studies literature on engaging with policy, tend to imply that these practices are necessary for scientists to influence policy. However, another field, Behavioural Science, has much more to offer on this topic.  

\section{Influencing at the science-policy interface}\label{sec:litinfluencing}

The discipline of Policy Studies is beginning to recognise that Behavioural Science can offer insights into how decisions are made and how to influence them (\cite{CairneyW2017}). \textcite{Darnton2008} provides a detailed resource that brings together a multitude of behaviour change models (alongside theories of change). Such insights into the influences on, and responses of, individuals are largely focused on the behaviour of individuals and society, which is useful for informing the design of public policy. However, a curious oversight is that these insights are also useful for understanding the behaviours and influences occurring within the policymaking process itself. Instead, decisions in policymaking are predominantly referred to as using \emph{Bounded Rationality} (\cite{CairneyO2020,CairneyTS2023}), an economic model of decision-making, which considers that satisficing, rather than optimising, choices are made based on the available information. 

Indeed, the underlying model behind advice on influencing policymakers supposes that they are deficient in relevant information. This is not dissimilar to the so-called \IDM{}, the idea that better decisions on science topics will be made if people have better science information. The \IDM{} has unknown origins (\cite{Nerlich2017}) and may simply be an intuitive response: ``if they knew what I know, they'd behave differently''. However, a number of related models have been developed based on ideas of information deficit (\cite{Darnton2008}). The \IDM{} prompts practices aimed at communicating more information or communicating it better. This has been more formalised regarding \CAN{} issues as the Gateway Belief Model, through which \textcite{vanderLindenLFM2015} and \textcite{vanderLinden2021} posit that communicating scientific consensus on climate change will alter attitudes to the issue. Whilst the \IDM{} has largely been shown to have only short-term relevance (\cite[p24-5]{BA2024trust}), the intuition that more information is what is needed to create change continues to motivate knowledge sharing approaches.

Moving beyond models of information deficit, Behavioural Science and Environmental Psychology propose a wide range of models to describe what motivates and constrains people's behaviours (\cite{Darnton2008}), finding that influences are much more varied than suggested by the \IDM. Recent efforts have created compilations of behaviour change models, with a particular aim to support the social transitions required to address \CAN{} issues (thus, still not specifically about policy decision-making). These include the \ISM{} (\cite{DarntonH2013}, based on \cite{SouthertonME2011}; Figure~\ref{fig:ism}), which divides influences into the individual, social, and material, the Theoretical Domains Framework (\cite{AtkinsFIOPIFDCGLM2017}), which has 14 clusters of behaviour change theories, and the model of \textcite{HamptonW2023} which brings together individual, social, physical and political sources of influence.

\begin{figure}
    \centering
    \includegraphics[width=0.5\linewidth]{figures/ism.png}
    \caption{Schematic diagram of the \ISM{} model of influences on behaviour}
    \label{fig:ism}
\end{figure}

\section{Motivation}\label{sec:litjust}

With scientists already involved in \CAN-related advocacy, it is reasonable to suppose that ``scientists want to be more than chroniclers of a preventable tragedy'' (\cite{WyattGT2024}). There is a significant body of research advising on the roles and practices that scientists should assume in order to engage with policy. This work is frequently motivated by a recognition of the urgency of \CAN{} crises. Yet, much of the theorising about scientists' engagement at the \SPI{} does not admit the complex nature of knowledge flows, the contested nature of Post-Normal \CAN{} science or that these observations suggest a fundamental transformation in how \CAN{} policy is made. Instead, this advice considers the needs of policymakers and suggests how scientists can align to these needs. %However, indications from the literature are that scientists are not enlisting the advice about policy engagement that is available to them. 
The implicit assumption of the advice is that information on how to engage with policy is lacking and, as such, this advice is using an \IDM-type approach to creating change in the behaviour of scientists. However, Behavioural Science suggests that influences other than lack of information is at play, but there is little work understanding what are these influences on scientists at the \CAN{} \SPI. This study aims to complement the advice literature by gaining a deeper understanding of these influences on scientists.
%
