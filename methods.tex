\chapter{Methods}\label{ch:methods}

Intuition that context matters.
Constrained to UK: Knowledge of institutions and language is within easy access for researcher. Including other [regions] introduces more dimensions to the context - funding, culture. \textcite{IbarraJOBCIMRS2022} provide a [cracking] review of the South American context for scientists in climate governance. Evidence that the political subsystem type decision-making. Westminster adversarial system can lead to obstruction of decision making \cite{PierreP2020} From \textcite{SaxonbergSL2023} Ingold \& Gschwend, 2014; Weible et al., 2010 conclude that experts have the greatest influence in an adversarial subsystem, because they provide legitimacy to one side that can use them against the other side. Evidence that issue salience affects attention \cite{OjanenBKP2021} find that in scientists can find their country of origin in relation to the country of policy has both positive or negative impacts on perceptions of scientists' legitimacy. \cite{StrassheimK2014} - demonstrate the different national contexts\unsure{ and their impacts on science policy - check this}

Participants will be identified using several sources: REF and UKRI databases summarising research impact and direct involvement with UK CAN policy. The 2014 and 2021 REF (Research Excellence Framework) provide self-reported summaries of impact from UK universities and contain a number of references to influencing CAN policy. Similarly, UKRI Gateway to Research offers [when the database is fixed] a means to identify research that has policy-influence. From these, I will identify specific projects, and thus scientists, who have worked to influence UK public policy on climate change mitigation. I will extend this list of scientists by identifying those who have specifically worked with the UK Climate Change Committee, for instance in developing Integrated Assessment Models (IAMs) for the UK’s carbon budgets.


Began asking about participants motivation for getting involved in their science to understand whether its the science of the application of the science that is of interest

Also asked what they would like to see as an outcome of their efforts - a prod if you like at "what does success look like"

conversational interview \url{https://methods.sagepub.com/reference/encyclopedia-of-survey-research-methods} with pre-defined topics


description of research
 - motivations and interests
early policy experience
later policy experience if relevant

\section{Labelling}

Each statement (full sentence or clause) from the transcript was labelled first according to the ``high level'' domain(s) to which it pertained and then to the more refined ``category'' labels for each domain.

\subsection{High level domains}
Used the frameworks from \textcite{BuseMW2012} and \textcite{HaynesDCRHGS2011}, with some reference to \textcite{DarntonH2013} and a process of iteration (see below) to tease apart ambiguities that arose in the context of the science policy interface. For instance, neither \textcite{BuseMW2012} nor \textcite{HaynesDCRHGS2011} accounted for \emph{belief}, but this was identified in some of the scientists' narratives and falls in a group with values in \textcite{DarntonH2013}. Further, \textcite{HaynesDCRHGS2011} rather fudges \emph{processes}, making it rather generally about events and including barriers and enablers, as well as \emph{context}, which has a very broad scope. \textcite{BuseMW2012} is more specific about both these domains; The former is more precisely about policymaking, that latter systemic factors. However, a further category of \emph{experiences} was deemed relevant to flag expressions of sentiment and perceptions that did not fit easily into the other categories. This process was further aided using high-level questions to guide ...?

\subsection{Label iteration}
Starting with the 5 domains of interest used by \textcite{HaynesDCRHGS2011}, a refinement was performed by first labelling of the transcripts two rather dissimilar scientists contexts (one with a great deal of experiences and one with less experience). This first labelling identified the domains to which each statement pertained as well as defining a set of category labels in each domain that described the more specific nature of each statement. By considering the category labels, it was possible to identify where domains were too vague. For example, when the \emph{process} domain contained statements of and functional processes, alongside statements about individual experience and perception, it was decided to split this domain into two.

Once this second iteration of domains was settled on, the category labels were grouped and nested. This hierarchy of categories within each domain was tested by applying it to a further transcript, in addition to the original two. At this stage, the statements in all 3 transcripts were labelled using the refined category labels. During this process, decisions that had been made in the first labelling iteration were reconsidered, more category labels were added and others were combined.