\chapter{Motivations and methods}\label{ch:methods}


\section{Positionality and bracketing}\label{sec:metpositionality}

It is pertinent  to include a note on the researcher's positionality (\cite{CreswellP2017}). I have been a Chartered Scientist [ref] for nearly a decade and have over 2 decades of experience working in technology in an arm's length government organisation. My career developed out of broad study of environmental sciences followed by research into the landscape and ocean measurement. Over recent years, my heightened concern about crises in society and the natural world have led me to occasional campaigning and the informal and formal study of organisational and public decision making. I am connected via social media with scientists in a range of \CAN-related fields. This study arose from my sense of an incongruity between the literature about modes of engagement with policy settings and frustrations expressed by experienced scientists on social media about such policy engagements.

%I identify myself as a broad-[spectrum] scientist, having studied environmental sciences, focussing on monitoring the land- and sea-scape using remote sensing. I have spent more than 2 decades focussed on developing algorithms for processing aerial imagery. Much of that time, I worked with neural network algorithms, including introducing deep learning to an arms length government organisation, whilst all the time maintaining a passion for, and occasional activism about, the natural world. The present study emerged out of the mismatch I directly perceived between scientific understanding, both regarding the natural world and technological issues (particularly regarding so-called ``artificial intelligence''), and public decision making.

\section{Identifying potential participants}\label{sec:metidentify}
\info{what do we mean by scientists?}
\info{what do we mean by policymakers?}

Just a dump of thoughts at the moment!!
taking a phenomological approach (\cite{CreswellP2017}) to understanding the experiences of scientists and from this drawing out their roles and strategies

example wording from \cite{KothariME2009} - ``A phenomenological theory of inquiry guided this analysis (Creswell, 1998); as such, we sought to understand the role of research in influencing policy, and researchers in influencing policy actors, as experienced by researchers. To this end, the study was designed as a qualitative study using semi-structured interviews''

\cite{MinamitaniDOI2024} use ISM and provide useful questions for each factor

\section{Interviews}
Participants were identified using several sources: REF and UKRI databases summarising research impact and direct involvement with UK CAN policy. Also used UKRI blog posts and asked participants to suggest others.

Semi structured interviews around a set of prompts. Aimed at creating a relaxed setting that the participants can share what was more pertinent to them, so questions were phrased and ordered to align with the flow of the participants' answers.

Question examples (add prompts)

Describe your field of work\\
Describe how became involved in policy\\
Describe what happened\\
Success - ``what would you like / have liked to see arise as a result of [that engagement]''\\
Frustration\\
What learned\\
Anything else\\
description of research\\
 - motivations and interests\\
early policy experience\\
later policy experience if relevant\\
what would you hope to achieve - (what is a successful outcome)\\
asked about frustrations because other sources (publications, REF evidence submissions, etc.) rarely express these

Began asking about participants motivation for getting involved in their science to understand whether its the science of the application of the science that is of interest

Also asked what they would like to see as an outcome of their efforts - a prod if you like at "what does success look like"

conversational interview \url{https://methods.sagepub.com/reference/encyclopedia-of-survey-research-methods} with pre-defined topics

Transcribed as statement clauses (part or whole sentences or blocks of sentences if single statement is being made)

\section{Transcript labelling}

Loaded as csv into Excel and used columns to record one or more appropriate labels

Each statement (full sentence or clause) from the transcript was labelled first according to the ``high level'' domain(s) to which it pertained and then to the more refined ``category'' labels for each domain.

\improvement{The method has evolved from that below and so this needs updating!} Much of what is below will probably be deleted or put in the appendix.



\subsection{High level domains}
Used the frameworks from \textcite{BuseMW2012} and \textcite{HaynesDCRHGS2011}, with some reference to \textcite{DarntonH2013} and a process of iteration (see below) to tease apart ambiguities that arose in the context of the science policy interface. For instance, neither \textcite{BuseMW2012} nor \textcite{HaynesDCRHGS2011} accounted for \emph{belief}, but this was identified in some of the scientists' narratives and falls in a group with values in \textcite{DarntonH2013}. Further, \textcite{HaynesDCRHGS2011} rather fudges \emph{processes}, making it rather generally about events and including barriers and enablers, as well as \emph{context}, which has a very broad scope. \textcite{BuseMW2012} is more specific about both these domains; The former is more precisely about policymaking, that latter systemic factors. However, a further category of \emph{experiences} was deemed relevant to flag expressions of sentiment and perceptions that did not fit easily into the other categories. This process was further aided using high-level questions to guide ...?

\subsection{Label iteration}
Starting with the 5 domains of interest used by \textcite{HaynesDCRHGS2011}, a refinement was performed by first labelling of the transcripts two rather dissimilar scientists contexts (one with a great deal of experiences and one with less experience). This first labelling identified the domains to which each statement pertained as well as defining a set of category labels in each domain that described the more specific nature of each statement. By considering the category labels, it was possible to identify where domains were too vague. For example, when the \emph{process} domain contained statements of and functional processes, alongside statements about individual experience and perception, it was decided to split this domain into two.

Once this second iteration of domains was settled on, the category labels were grouped and nested. This hierarchy of categories within each domain was tested by applying it to a further transcript, in addition to the original two. At this stage, the statements in all 3 transcripts were labelled using the refined category labels. During this process, decisions that had been made in the first labelling iteration were reconsidered, more category labels were added and others were combined.

\subsection{Intuition that context matters - braindump}
Constrained to UK: Knowledge of institutions and language is within easy access for researcher. Including other [regions] introduces more dimensions to the context - funding, culture. \textcite{IbarraJOBCIMRS2022} provide a [cracking] review of the South American context for scientists in climate governance. Evidence that the political subsystem type decision-making. Westminster adversarial system can lead to obstruction of decision making \cite{PierreP2020} From \textcite{SaxonbergSL2023} Ingold \& Gschwend, 2014; Weible et al., 2010 conclude that experts have the greatest influence in an adversarial subsystem, because they provide legitimacy to one side that can use them against the other side. Evidence that issue salience affects attention \cite{OjanenBKP2021} find that in scientists can find their country of origin in relation to the country of policy has both positive or negative impacts on perceptions of scientists' legitimacy. \cite{StrassheimK2014} - demonstrate the different national contexts\unsure{ and their impacts on science policy - check this}
