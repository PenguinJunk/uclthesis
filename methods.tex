\chapter{Motivations and methods}\label{ch:methods}

\section{Motivation}\label{sec:metmotivation}

This study begins to address the gap in understanding of scientists experiences of the \SPI. It is founded on the model of engagement with policy described by Policy Studies that outlines roles and practices for scientists. It expands this model using behavioural insights to uncover the influences that scientists experience and the practices they use to adapt to, mitigate, or to capitalise on, these influences. This study focuses on the \SPI{} defined by \CAN{} science and policy, conceptualising \CAN{} science as \PNS{} and therefore a domain in which science is one of many knowledges to be used in policymaking. 

\section{Positionality and bracketing}\label{sec:metpositionality}

It is pertinent  to include a note on the researcher's positionality (\cite{CreswellP2017}). I have been a \href{https://sciencecouncil.org/scientists-science-technicians/which-professional-award-is-right-for-me/csci/}{Chartered Scientist} for nearly a decade and have over 2 decades of experience working in technology in an arm's length government organisation. My career developed out of broad study of environmental sciences followed by research into the landscape and ocean measurement. Over recent years, my heightened concern about crises in society and the natural world have led me to occasional campaigning and the informal and formal study of organisational and public decision making. I am connected via social media with scientists in a range of \CAN-related fields. This study arose from an observation of an incongruity between the literature about practices of engagement with policy settings and frustrations expressed by experienced scientists on social media about such policy engagements.

%I identify myself as a broad-[spectrum] scientist, having studied environmental sciences, focussing on monitoring the land- and sea-scape using remote sensing. I have spent more than 2 decades focussed on developing algorithms for processing aerial imagery. Much of that time, I worked with neural network algorithms, including introducing deep learning to an arms length government organisation, whilst all the time maintaining a passion for, and occasional activism about, the natural world. The present study emerged out of the mismatch I directly perceived between scientific understanding, both regarding the natural world and technological issues (particularly regarding so-called ``artificial intelligence''), and public decision making.

\section{Definitions}\label{sec:metdefinitions}
\info{what do we mean by scientists?}
\info{what do we mean by policymakers?}

What I mean by policymaker - from \cite{Obermeister2020} - ``By `policymakers' I broadly mean (influential) actors within government departments, the legislative branch (e.g. Parliament), and/or organisations with statutory powers (e.g. non-departmental public bodies), who are chiefly concerned with policy formulation and evaluation as opposed to enforcement.''

\section{Method}\label{sec:method}

Guided by a phenomological theory of enquiry (\cite{CreswellP2017}), this study sought to understand the experiences of \CAN{} scientists when they have engaged with the \SPI. As such, this was a qualitative investigation using semi-structured interviews to collect data for description and analysis. This study had  methodological phases: identification and invitation of potential participants (Section~\ref{sec:metidentify}), interviewing of participants (Section~\ref{sec:metinterview}), and transcript labelling and analysis (Section~\ref{sec:metlabelling}). %A short quantitative analysis was also undertaken (Section~\ref{sec:metquant}).

\subsection{Identifying potential participants}\label{sec:metidentify}
Potential participants in this study were identified using several sources. Firstly case studies and blog posts on the \REF{} and \UKRI{} websites were used to identify \CAN{} science research that was labelled as having impact on public policy. This included searching the \href{https://results2021.ref.ac.uk/impact}{REF Impact database}\footnote{Unfortunately, at the time the \UKRI{} \href{https://gtr.gtr.ukri.org/}{Gateway to Research database} was not functioning}. Other potential participants were elicited during early interviews.

Potential participants were invited by individual email to take part in a 1-hour interview to understand their experiences of engaging with public policy. Willing participants were asked to complete an ethics consent form and were able to book a suitable time for the interview themselves.

\subsection{Interviews}\label{sec:metinterview}

The interviews were structured around a series of prompts. The aim was to create a relaxed setting in which participants were comfortable sharing their experiences of engaging with policy, with an emphasis was on learning about these in the participants' own words. Therefore, apart from a question related to the first prompt, the interview questions were phrased and ordered to flow as naturally as possible from participants' previous answers, leaving out questions if these were naturally answered in response to a previous question. Table~\ref{tab:metinterview} gives prompts and example questions, although in the interviews these were much less formal (e.g. of the form ``\textit{you mentioned X, please could you tell me more about it?}'').

The interviews took place over Microsoft Teams and were video recorded and automatically transcribed. Each transcription was then refined by the researcher by listening to the video and correcting and annotating the transcription. Transcripts broken down into statements, usually whole or part sentences, each having a singular meaning stored as comma-delimited text (.csv) files with one statement per line. This file format allowed multiple labels to be applied to each statement.

\begin{table}
    \footnotesize
    \caption{Interview prompts and example questions}\label{tab:metinterview}
    \begin{tabular}{L{.4\linewidth}L{.6\linewidth}} 
    \textbf{prompt} & \textbf{example questions} \\ \hline
    field of work & ``\textit{how would you describe your work?}'' \rule[-2ex]{0pt}{6ex}\\
    early policy experience & ``\textit{how did you become involved in policy?}'' \\
     & ``\textit{what happened?}'' \\\rule{0pt}{4ex}
    later policy experience (if relevant) & ``\textit{describe a later involvement in policy?}'' \\\rule{0pt}{4ex}
    the meaning of success & ``\textit{what would you have liked to have come out of that [engagement]?}''\\
     & ``\textit{what went well?}'' \\\rule{0pt}{4ex}
    what is frustrating & ``\textit{what didn't go well?}'' \\
     & ``\textit{in what way was that frustrating?}''\\\rule{0pt}{4ex}
    learned practices  & ``\textit{tell me what you have learned from that experience?}'' \\
     & ``\textit{what would you advise to someone who was interested in engaging with policy?}''\\\rule{0pt}{4ex}
    anything else & ``\textit{what did you expect me to ask about that I haven't?}'' \\
     & ``\textit{is there anything else you'd like to tell me about?}'' \\[2ex] \hline
    \end{tabular}
\end{table}

\subsection{Transcript labelling}\label{sec:metlabelling}

The transcript CSVs were combined into a single CSV and loaded into Microsoft Excel. This permitted multiple labels to be applied to each statement using a spreadsheet column for each factor. The \ISM{} (\cite{DarntonH2013}, see also \cite{MinamitaniDOI2024} for an example application) was used to identify influences on scientists at the \SPI. This required several iterations, with reference to the domains of enquiry from \textcite{BuseMW2012} and \textcite{HaynesDCRHGS2011} to draw out the specific contexts of the \SPI. The final result of this labelling was that each statement had a category label in one or more of the \ISM{} factors. For example, the statement \tquote{In some ways that was quite shocking}{p04}{62} had the category label ``surprise'' in the \ismie{} factor. Statements were also given labels to describe the nature of the knowledge exchange (if any), and whether the statement was about a success, a frustration and/or a strategic practice being described or suggested by the participant.

%\subsection{Quantitative analysis}\label{sec:metquant}

%The labelled data in CSV format was loaded into R to perform a series of quantitative analyses to determine 

