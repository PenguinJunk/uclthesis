\chapter{Methods}\label{ch:methods}

Intuition that context matters.
Constrained to UK: Knowledge of institutions and language is within easy access for researcher. Including other [regions] introduces more dimensions to the context - funding, culture. \textcite{IbarraJOBCIMRS2022} provide a [cracking] review of the South American context for scientists in climate governance. Evidence that the political subsystem type decision-making. Westminster adversarial system can lead to obstruction of decision making \cite{PierreP2020} From \textcite{SaxonbergSL2023} Ingold \& Gschwend, 2014; Weible et al., 2010 conclude that experts have the greatest influence in an adversarial subsystem, because they provide legitimacy to one side that can use them against the other side. Evidence that issue salience affects attention \cite{OjanenBKP2021} find that in scientists can find their country of origin in relation to the country of policy has both positive or negative impacts on perceptions of scientists' legitimacy. \cite{StrassheimK2014} - demonstrate the different national contexts\unsure{ and their impacts on science policy - check this}

Participants will be identified using several sources: REF and UKRI databases summarising research impact and direct involvement with UK CAN policy. The 2014 and 2021 REF (Research Excellence Framework) provide self-reported summaries of impact from UK universities and contain a number of references to influencing CAN policy. Similarly, UKRI Gateway to Research offers [when the database is fixed] a means to identify research that has policy-influence. From these, I will identify specific projects, and thus scientists, who have worked to influence UK public policy on climate change mitigation. I will extend this list of scientists by identifying those who have specifically worked with the UK Climate Change Committee, for instance in developing Integrated Assessment Models (IAMs) for the UK’s carbon budgets.


Began asking about participants motivation for getting involved in their science to understand whether its the science of the application of the science that is of interest

Also asked what they would like to see as an outcome of their efforts - a prod if you like at "what does success look like"

conversational interview \url{https://methods.sagepub.com/reference/encyclopedia-of-survey-research-methods} with pre-defined topics


description of research
 - motivations and interests
early policy experience
later policy experience if relevant