\chapter{Discussion}\label{ch:discussion}

%Scientists are using a wide variety of strategies at the CAN science-policy interface, many relate directly to influences that they are aware of or are suspecting. Given the assumption that the frequency of mentions of an influence relates\unsure{I am not very confident about this assumption} to the degree of its influence over scientists' policy engagement, the ISM can suggest strategies to focus on or develop.
%\tquote{urgent crisis}{p14}{61}
%\tquote{a theme of such importance}{p11}{50}
%\tquote{because you know what needs to be done and yet we keep getting told to go back and do more research on it}{p11}{39}\tquote{It's a displacement activity, you get researchers to go back}{p11}{40}\tquote{We know enough about what's going to happen. We know enough already.}{p11}{42}\tquote{and I feel governments have enough information on which to act already and have quite sufficient evidence. So I find it frustrating [that] what they've already got in their knowledge hasn't really caused massive u-turns}{p04}{48}

Weave in the \IDM!
Weave in that literature is largely correct on roles and practices - but its more complicated
\section{Scientists' roles}\label{sec:disroles}

The literature on roles for scientists at the \SPI{} focus on generating knowledge and insights for policy. These roles imply a one-way transfer of knowledge, as exemplified by \textcite{SteelLLS2004}'s \emph{reporting}, \emph{interpreting}, \emph{integrating}. However, it was clear within the interviews that knowledge and insight goes in many directions as scientists were also gaining and using awareness of policy initiatives and processes, discussing and collaborating on policy, and even setting agendas, disseminating policy and managing the consequences of policy (e.g. Section~\ref{sec:resskipoli}). This corresponds to observations about the \SPI{} that it comprises complex and dynamic interactions and flows of information (Section~\ref{sec:litspi}). It further suggests that normative instructions to scientists to align with the needs of policy may be somewhat challenging in the complex and dynamic reality. As well as this general observation about knowledge interactions, participants also identified a range of tensions within their roles. 

\subsection{Pure Scientists}\label{sec:disscientist}
A distinction was made by several participants between ``fundamental'' science and ``strategic'' science, the latter also being referred to as ``impact work''. Several participants' engagement with policy arose as a consequence of producing fundamental science and there was a firm belief expressed by some participants that fundamental science was necessary for policy impact, \tquote{This stuff's important and I always felt that I wanted a job [that] makes a difference, you're not just finding out new knowledge, which is what's fundamental science is and that's important, but actually this is an urgent crisis and we need to know that there are good people providing good evidence into government}{p14}{61}. Here, then, is evidence of the necessity of Pielke's \emph{Pure Scientist} role. Another motivation for being in this role was a hint that it is more esteemed than the other roles, as \tref{one participant reflected that scientists in their organisation were not interested in an opportunity to undertake a role as \emph{Science Adviser}}{p06}. 

Several participants commented that certain scientific fields seem more highly regarded than others, noting that technology tended to be most favoured by government. Participants who researched in technology fields discussed the effort required to generate engagements much less. Biophysical measurement and modelling research also seemed to ease engagement, such as where it could be used to demonstrate government investment. In this case, there was a theme across several participants that the interest was for more scientific research when they felt existing findings are already sufficient, \tquote{it's not clear to me, as a scientist, what other evidence is really necessary from that point of view}{p04}{49} (e.g. Section~\ref{sec:resskifram}). Participants undertaking behavioural and social sciences discussed their efforts to engage with policy a great deal. In fact, mentions of ``fundamental'' science were always related to the biophysical sciences. As one social scientist observed about their work, \tquote{whilst there are some interesting theories and ideas, it was always very applied to me}{p03}{14}. 

Thus, even this most basic \SPI{} role for scientists encounters tensions because of a ranking of scientific fields, it being unclear if social sciences can indeed be ``fundamental'', as well as suggestions that pure science may be being used as a \tquote{displacement activity}{p11}{40} from political action.

\subsection{Advisers and Advocates}\label{sec:disadvocacy}

As expressed in the literature, participants identified tensions between the roles of \emph{Science Adviser} (or \emph{Knowledge Broker}) and \emph{Advocate}, e.g. \tquote{With science policy, there's always this question and clarifications that one needs to make between advice and advocacy, and advocacy and then campaigning}{p09}{101}. The conflict was often regarding the impact of advocacy on the credibility of the scientist (\tquote{There's some climate scientists that are very opinionated ... they are very highly regarded, by some ... but most of the time, that's not by people who actually are involved in policy making}{p09}{115}) 
and, by implication, the evidence they conveyed (\tquote{the science evidence very rapidly gets tainted if you try and use the science to argue for a particular line of action}{p13}{40}). However, other participants endorsed advocacy (\tquote{The fact that you're choosing to research this means you think this is good and valuable, and so why won't you advocate for it in a clear way?}{p06}{88}), identified it as a necessary evil (\tquote{you really have to invest in, essentially, lobbying, tragically}{p08}{65}) because \tquote{ministers are getting lobbied from everybody else, we're not lobbying them}{p06}{91}. A couple of participants identified an unwanted pressure to advocate from people in their network who are not as close to the interface (\tquote{we shouldn't be advocates, but we should be technical experts}{p01}{71}).

This issue is nuanced. On the one hand, when policymakers are well aware of an issue, indeed are asking questions, advocating a related cause or course of action may undermine a scientists' credibility. However, those pushing for more advocacy were often describing situations when considered that the cause for which they wished to advocate was not on the policy agenda, such as behaviour change in \CAN{} policy. Also, a few participants noted that funding of research is increasingly tied to demonstrating impact, which limits the ability of scientists to research topics that are not on the policy agenda. In such cases, advocacy may be necessary to draw policy attention to wider issues. This does not preclude possible impacts to a scientists' credibility but for some there will be a moral imperative to draw attention to the knowledge they create, even if the benefits of credibility are only bestowed on those who come later. 

%\tquote{I wrote the IPCC chapters, but yet other young people were going out and getting themselves arrested}{p06}{94} \tquote{I don't disagree in a certain degree that if we can keep science neutral, we can keep evidence-informed policy making. I see the balance and the need for it and the neutrality of science. But it's weird, it's a tricky line}{p06}{95}

\subsection{Citizens and Scientists}

A few participants identified themselves as \emph{Citizen}s in addition to their professional role. This tended to be in relation to feelings about how \CAN{} issues are being addressed by policy. Those participants who appeared to feel agency within their role, were expressing frustration with policy as a citizen: \tquote{I think [I] probably feel more [disappointed by the UK policy system] as a citizen or a taxpayer ... than as a researcher}{p08}{32}. This contrasted with the curiosity they felt within their professional role: \tquote{Whereas of course, as a scientist you can again nicely study why that is the case and therefore the scientific frustration is low but the societal frustration high}{p09}{59}. Others, who indicated feeling less agency within their professional roles, described becoming involved with campaigning organisations: \tquote{The reason I joined Greenpeace was because ... young people were going out and getting themselves arrested ... because of what I wrote ...}{p06}{94}.

\subsection{Roles are a complex and dynamic reality}

Much as the \SPI{} is, in reality, complex and dynamic, so are the roles that scientists are taking on. Several participants described how they have pivoted in their role, particularly in response to what they felt was as an imperative given the challenges exposed by \CAN{} sciences. %p08, p07, ?p13
One participant expressed how different roles are a necessity at the \CAN{} \SPI: \tquote{you need the boundary people more than ever to translate the work for the policymakers because academics aren't going to be able to do it. It's not our job and we don't understand it. We don't understand government, so we write these stupid policy briefs with no target person, there's no target department, there's no target policy process, there's no window of opportunity for change, no levers or anything that they're targeting, it's just here's some information and that doesn't help}{p06}{103}. \CAN{} issues are creating tensions for scientists in many roles at the \SPI{} who many be more conflicted than scientists in other fields by trying to balance scientific principles of objectivity with their deep concern for the natural world.


\section{Scientists' practices}\label{sec:disstrategies}

Many of the practices identified by participants fit into the 8 tips described in \textcite{OliverC2019} (Table~\ref{tab:litstrategies}). There were a few practices that participants mentioned that appear to fall outside of these precepts. The first was to challenge the perspectives of policymakers including by challenging traditional economic approaches to decision-making (Section~\ref{sec:resskipers}), using physical or behavioural science when framing issues (Section~\ref{sec:resskifram}) or by applying existing components of the policy landscape to scientific research (Section~\ref{sec:resskipoli}). By challenging policymaking, the practices add to the dynamism of the \SPI. The second practice was to demand recognition (Section~\ref{sec:resskiinst}). This reflects the well-established practice within science of citation, which not only ensures proper credit for work but also establishes the provenance and rigour of ideas.

Other practices mentioned by participants take the published advice a little further. These include creating diverse engagements that include dialogue and collaboration beyond policy to industry and citizens (Sections~\ref{sec:resskiagen},~\ref{sec:resskipers} and~\ref{sec:resskitech}), collaborating with policymakers on a particular topic or at a particular venue (Sections~\ref{sec:resskinetw} and~\ref{sec:resskiknow}), even establishing joint resources (Section~\ref{sec:resskiinfr}). These practices are only possible once relationships have been established at the \SPI{} but they suggest greater parity of science with policy than is suggested by much of the advice to scientists. Another way that the scientific findings are being interpreted for specific contexts. This is a recognised part of the role of a \emph{Knowledge Broker} but even participants based in academia were using knowledge about the policy landscape to research and communicate their science according to the remits of different ministries, governments, economic sectors or public bodies: \tquote{you've got to get to that level of granularity to be useful to them, you have to do several more steps to get actionable knowledge}{p08}{59} (Section~\ref{sec:resskitech} and~\ref{sec:resskifram}).

There was a strong contrast between those participants who were particularly frustrated by how difficult it was to build relationships with policymakers and those who found they were frequently being sought out by policymakers. One participant noted that this can vary over time: \tquote{don't take access or a listening ear for granted. Appreciate the opportunity every time that have been given it}{p09}{87} and another that the importance of trust during critical moments in the policymaking process results in policymakers calling on the same individuals: \tquote{the faster that the process operates, the more that particular actors don't have time to to go and look and open up a a catalogue of experts, or even to do stuff online. It's just literally \emph{we need something now} and [they] just go to the go-to people or the go-to providers of that [evidence/knowledge]}{p10}{60}. There is little comfort here for those scientists who struggle to engage (in some instances in this study this was related to the [interest in] different sciences described in Section~\ref{sec:disscientist}).

\subsection{Practices need effort, experience and engagement}

The diverse strategies described by participants chronicled a great deal of effort and gained experience researching and communicating at the \SPI. Some of the most interesting practices relied on having established, trusted relationships and hinted at the determination of participants to find approaches that work to shift policy in line with how they perceived the challenges of \CAN.  
%To effectively communicate knowledge into policy appears to require a great deal of effort, as as \textcite{BednarekSHG2015} demonstrate, this may be too much for any single scientist without the support of a dedicated knowledge brokerage organisation. ``connecting science and policy is not a part-time, `do it yourself' enterprise, but instead benefits from the skills and experience of practitioners who are immersed in the process''

\section{Scientists' experiences}\label{sec:disexperience}

Participants described a wide variety of experiences of engagement, from more structured responses to calls for evidence and appearing before select committees, to less formal direct approaches by government staff and ministers, as well as venues created by non-policy actors such as debates and ``round tables''. All the participants spoke of a desire to engage with policy. 
The range of opportunities that they experienced speaks of a desire for policymakers to speak to experts%, and the authority and credibility of knowledge gained directly from scientists who are deeply immersed in their topic
. Several participants had experience of \SPI{} engagement from very early in their academic careers, unaware that it was unusual at the time. This led them into a career with a strong focus on engaging with policy. Another participant had had only limited experience engaging with policy but, as a consequence, was keen to find more ways to work with policy and tended to think more about the relevance to policy of their work. Thus there is a good justification for drawing together knowledge from scientists about how to best perform in this undertaking. 

A very common experience across the interviews were the intricacies of the different components of national and international policymaking, particularly how unfathomable some aspects are (Section~\ref{sec:resskiinst}). From four different participants: \tquote{it's really hard to know about government from the outside. You can't really blame academics for not}{p06}{102}, \tquote{I don't know whether it's DECC or BEIS or DESNZ (I suspect it's the same group of people)}{p04}{18}, \tquote{it's about understanding how policymaking works (and no one really understands it) but understanding it a little bit}{p12}{82}, \tquote{Even just looking at UK the patchwork of government departments and POST and the committees and the Lords and the Commons and everything, just trying to work out ... it is incredibly complex}{p05}{94}. Here one theme stood out - that national UK policy settings are the hardest to engage with compared to both local and devolved, and international settings.

Participants with established networks described how they learned about the intricacies of policymaking through this network (Section~\ref{sec:resskinetw}) and all participants identified some aspect of their network and relationships that were essential to their capacity to engage with the \SPI. Being recommended to give advice by other members of their network, or recommending others themselves were also common themes. Much as is described in the literature, scientists' networks and relationships are fundamental to the experiences engaging at the \SPI.  

There were contentions specific to \CAN{} science and policy that all participants\footnote{The quotes in this paragraph illustrate perspectives from 6 participants} were cognisant of (Section~\ref{sec:resskifram}). They noted that \CAN{} science is more than the simplistic focus on carbon, trees or 1.5\degree C and they grappled with the consequences of these simplistic measures. For example: \tquote{the evidence coming back from my colleagues is that planting trees is not a panacea}{p11}{53}. Yet they found it difficult to convey this knowledge to policymakers because it is contrary to popular understandings of the science. Particular \skifram{} of \CAN{} create tensions:  \tquote{We're having some traction ... on the possible bad security implications or impacts ...}{p08}{37} yet \tquote{... the securitization of climate change ... was well meaning, but ... it gives you the wrong solutions if you think of it in a securitized way}{p05}{47}. %They considered this to be down to a \tquote{sleight of hand at the policy level internationally ... 30 years ago they went from saying `we've got to do something about fossil fuels' to `we've got to count carbon'}{p11}{56}. 
%Even fundamental climate and Earth system modelling is somewhat contentious in that it isn't providing immediate solutions. \tquote{It's getting more holistic ... at a stage where they are delivering results and hopefully contributing to understanding what actions we can take ... but it actually takes decades to develop these things and you can imagine that actually, if we just tweak it a bit more and then you need more funding. So it does feel like a continuum really ... you do sometimes feel like you're just finding problems}{p07}{18-19}. Another participant suspected that funding modelling may be popular because \tquote{it's quite a convenient thing because models look good and they provide data and all the other stuff [such as adaptation] is so messy}{p11}{43}.
And participants described \tquote{quite a strong debate}{p13}{43} within society about some negative emissions technologies: \tquote{people sometimes might say [a negative emissions technology] is a moral hazard because governments can say `oh, it's fine, we will have these magical greenhouse gas removals, we'll have these carbon hoovers ... we can carry on doing what we're doing'}{p01}{118} and \tquote{biomass is seen as competition for [food production]. The farming lobby and the food side of the farming lobby is very strong}{p14}{92}. \tquote{Some people saying we just shouldn't be messing with the ocean for instance, we shouldn't be changing anything in the ocean that's environmentally not sensible}{p13}{43} but those researching negative emissions technologies felt \tquote{we need to be consuming less \emph{and} we need to also be sucking out the greenhouse gases that we've already emitted}{p01}{119}, noting that \tquote{people ... will say if we don't then the planet is going to get very warm and the environment will be damaged anyway}{p13}{43}. %Further, \tquote{nearly everything's controversial ... even wind energy ... things that seem pretty uncontroversial when you start actually trying to deliver them are controversial}{p13}{42}. This appreciation of the complexity was integrated into scientific research \tquote{We recognise [transition] is a new thing, there will be consequences of that, let's research that}{p01}{73}.
The resolution of these contentions is about much more than science and speaks to the 

classic post-normal science!

Perhaps it is also a feature of \CAN{} science-policy that it is difficult to identify impact, particularly when the ultimate measure of impact would be a reduction in the severity of the \CAN{} challenges (Section~\ref{sec:resskiinfr}). The motivation to have this ultimate impact led to bruising experiences for some participants. Two spoke of direct engagements within governments in which lack of necessary backing had risked their professional reputations. Another compared the experiences of trying to engage with policy with those of developing nations: \tquote{We are now the people [being] told: ``You need more training. You need to be entrepreneurs. You have to find out how to make money ... This is stuff we told people in developing countries to do for 30, 40, 50, 60, 70 years: ``what you were doing, your traditions and your way of doing is not good enough ... you are responsible for your plight''}{p11}{92}. These contrasted with the positive encounters of other participants, especially those who spoke of being focused more on the process than the outcome. Participants with expertise in working at the \SPI{} emphasised not only how important science is to decision-making, but also that particular skills are essential to convey that science: \tquote{it's really important that we have scientists in the system who understand that policy interface and can work at it}{p13}{75} but \tquote{the process of bringing science to policy is not typically something that every scientist is able to, or should be able to, in all honesty}{p09}{63}. 

As these experiences, and the short quantitative analysis (Section~\ref{sec:resroles}) demonstrated, whilst participants of this study had some common experiences, each participant's perspective was quite unique in terms of their scientific discipline, how engagement at the \SPI{} was initiated and evolved, as well as their motivations for, and reactions to, the engagement. This is not unexpected, even when limited to \CAN{}, the science and issues relevant at the \SPI{} are diverse. However, guides and literature on science-policy practice tend to treat scientists as a homogeneous group, to the extent that whilst policymakers are often defined, scientists are not (e.g. \cite{BA2024trust}). 

But there is a further matter here that was not spoken about in the interviews. Whilst social identity was not sought as part of this study, all of the scientists who participated presented as racially white. This may be a artefact of the small sample or selection process (Section~\ref{sec:metidentify}) but it speaks to the privileges that underlie roles within science and also within policy. The significance here relates to the contentions within \CAN{} science-policy, which are largely about the societal tensions between \CAN{} challenges and their potential solutions. As noted by several participants, the public need to be involved in the decisions that affect them. However, if science and policy do not fully represent society, these tensions are not being fully represented at the \SPI{}. 

\section{Kicking the Information Deficit Model habit}\label{sec:disdeficit}

The literature, and this study, have demonstrated that rational model of policy is \tquote{dead and buried}{p03}{126} (if it ever was a living thing), There is probably not a deficit of information for \CAN{} policymaking. Yet, much of academia and beyond, continues to work on the \IDM{} premise that if enough of the right information is supplied, change will come. Naturally it was a behaviour science participant who identified that \tquote{sometimes we have to remind ourselves that ``surely you're just converted by my two page briefing paper'' and oh! it doesn't work}{p05}{96}\footnote{although participants did express frustration that some parts of government were still operating using related instruments, such as product labelling}. Thus it is reasonable to also accept that scientists should not \tquote{just lob their results at policy makers and expect them to have traction}{p13}{74}.

An parallel anecdote from this study is that one participant referred me several times to their (excellent) impact case study (\tquote{so it was interesting and you can read all about it in the impact case study}{p10}{38}). I had, in fact, read this case at least as deeply as I imagine a diligent policy official would read a relevant briefing note. However, the case study was not written for a student wanting to learn about the roles and experiences of scientists at the \CAN{} \SPI. Therefore, it did not provide the answers to my questions. The interview process enabled me to gain more relevant insights.

Importantly, using models from Behavioural Science, this study found a range of factors other than mere information about engagement that influenced how scientists engaged at the \SPI. Evidence in the literature that scientists do not always refer to theories and guidance about engaging with the policy process was born out by only a few references to these in the interviews. Instead of being influenced by such information, participants in this study spoke about their values and beliefs, how they perceived themselves and others, and how their particular science discipline affected their ability to engage. They identified the people, networks and techniques that eased their engagement. Equally, they experienced frustrations arising from lack of knowledge and lack of resources. Participants experienced positive and negative engagements as a result of particular institutional practices or framings of \CAN{} science and policy. They also devised and applied a range of practices to enhance the positive and counteract the negative influences on their engagements. Some of these practices parallel advice to science, other run counter to it. Consequently, this study not only demonstrates that there are many dimensions to engaging and influencing \CAN{} science policy, it presents a challenge about how best to support scientists to engage at the \SPI{} - because providing another ``how to'' advisory note is clearly inadequate.

%As a scientist, I found this literature difficult to absorb, not only was it describing roles in a language that was not directly meaningful, it also implied the failure was wholly on the side of scientists. 

\section{Potential directions}\label{sec:disdirections}

There is a wealth of literature describing how scientists should behave at the \SPI{} and this study has identified that these roles and practices are largely born out \emph{in vivo}. It has also found that providing information to scientists is not sufficient to support better engagements at the \SPI. Instead, this section suggests .... directions for developing better experiences of the \SPI{} for scientists and policymakers... 

\paragraph{Networks and allyships:}


creating communities that bring together different scientific perspectives ... even other disciplines, multidisciplinarity, which aligns with Gibbons et al's Mode 2 knowledge production, perhaps also wider perspectives, post-normal science \cite{FuntowiczR1993}, extended peer communities, \cite{Jasanoff2003} recommending and advocating for? Transdisciplinarity even (\textcite{RussellWC2008} discusses constraints that current institutions create)
Where scientists may feel unheard, this is nothing to how the experience of the wider publics.
There is a psychological safety issue here, perhaps reflected in the comments of another about thinking they should perhaps had turned to activism. We should be concerned that some scientists - people who daily have to face the harsh realities of CAN science - experience the interface with policy [so brutally].
Scientists in more privileged fields should consider advocating for those in less privileged fields. This is analogous to advocacy in other contexts - the knowledge of those who tend to be marginalised by the current system is essential to [overthrow] that system.


\paragraph{Encourage citation:}
\tquote{making sure policymakers cite your research - that's probably something that we should tell people who are getting into this field}{p12}{}

Whilst ``scientists doing science for scientists'' may seem like a \emph{habit} that doesn't support policy engagement, the scientific discipline of citation benefits many actors across the science-policy interface. Firstly, it ensures that originators of evidence are easier to identify by both policymakers and scientists (\emph{opinion leaders}). Secondly, it supports the credibility of policy by allowing the provenance of policy to be traced (\emph{values, beliefs, attitudes}). Thirdly, where citation becomes a cultural norm, it can prevent duplication of effort and build legacy even where staff turnover is high (\emph{times and schedules}).


\paragraph{Science outreach to potential policymakers:}
With a common frustration among many of the participants being the high turnover of policy staff and the need to ``reeducate'' each newcomer (\emph{skills}), a opportunity to increase the science understanding of future policy staff is to have more CAN-specific civic education and education of political science students (\cite{DykeM2024})

\paragraph{Trust building:}

Psychology finds that building the trust of citizens in decision making during peace time leads to a more effective response to a crisis \cite{BollykyP2024}. One participant, a [behaviour scientist], observed that what is true for citizens' behaviours is just as true for policymakers. Thus, it would be reasonable for science and scientists to build and maintain the trust of policymakers ... CRELE ... it is a long game.

\paragraph{Deepening understanding of experiences}
This strengthens the justification for a deeper study of the experiences of scientists at the interface with policy, since it is only with such understanding can [approaches, strategies, things] be developed that can create truly effective and rewarding engagements.

\section{}



%\tquote{[some scientists are really brilliant at] the engagement side, public engagement and putting things very clearly, others are just like - I don't understand you and I'm in the project - those skills are really important as well. You can wheel out certain people, some people you wouldn't and that's why you have a team really, isn't it? So people have different skills, but you need balance.}{p07}{63}

How much more knowledge is required before policy catches up with the science?



\section{Limitations to this research}
ISM used as a counter to this model - all the other influences. However, there are influences that remain rather implicit, in particular power relations. \cite{HamptonW2023}'s model includes governance and democracy (but not roles and identity explicitly)

time and resource constraints. other studies, e.g. \cite{HaynesDCRHGS2011,OjanenBKP2021} have more researchers and thus possibility to confer over coding as well as time for more and longer in depth interviews

Useful approach to selection influential scientists used by \textcite{HaynesDCRHGS2011} by nomination - not enough time to do this in the present study but [would perhaps create proxy for trying to determine what is success]

The refinement of \ISM{} factors into factors easier to [rationalise] with \SPI{} was helpful because it [allayed] the sense of [brute forcing] the data into a ill-fitting framework. However there is more to do here. The factor \skiskil{} was left unchanged (except by name) from the original data labelling, leaving the nature of skills and knowledge only superficially investigated. Further, \ismie{} were not deeply investigated due to time constraints and yet there is a great deal of valuable information in terms of both motivations for and reations to engaging with the \SPI.

Despite setting constraints by topic to \CAN{} scientists, the sample of participants resulted in a diverse set of experiences. There were common themes but ...




\section{braindump}

patient science - good quality science is essential for policy and can take a long time to accumulate - requires long term investment (no expectation of returning quick impact). This may be difficult with [highly specified funding requirements]

\subsection{To institutions}
incentives are not fully there for frank policy engagement - e.g. \cite{ElsensohnACDGGKPRS2019} ``biases and limitations at scientific institutions, including but not limited to, a lack of incentive structures, institutional guidelines, and employment limitations'' - example of programme of training in science advice, maybe advocacy \cite{RussellWC2008}

\subsection{science versus evidence}
science versus evidence p09
evidence review p10, synthesising p01..., hub p06/3 p06/29

\subsection{Frames and windows}
meanings - all about framing
policy institutions - frame the topic and the times very tightly - windows of opportunity
this runs counter to science, which may attempt to transgress disciplinary boundaries and has always proved difficult to timeframe
frames seen being applied more in science - what will be funded, what is easier to publish
such framings, each being concluded with a decision, structure the policy process
- some see this as a constraint, meaning some topics are always off the table but it may also open up policy to prepared researchers who can prepare in advance for forthcoming framings [KEU advice]

\subsection{citizen engagement}
citizens as stakeholders and subjects of CAN policy decisions were mentioned.... The adverse impacts of [the waterfall of policy] at the community level were described in detail by one p11 ... for this reason, others expressed a belief that people should be involved in changes that will affect them (such as through CAs) p03  p01 aware of impact of their technologies on people, and considering researching how to address this, as well as being very open and keen to demonstrate to anyone who is willing to ask

\subsection{intimate understanding}
of policymakers and policy contexts p01, ...
of people and society ..., of the science and projections of CAN for so many years [p02, ... , p08]
\subsection{long time} {p01}{115} {p02}{61} {p03}{102} {p05}{88} {p07}{18} {p08}{55} {p09}{51} {p10}{61} {p11}{7} {p13}{09} {p14}{78}{112}

\subsection{types of science}
Differences between physical, technology and social science contexts?
Considering WesselinkH2020 typology of problems, at first glance climate change appears to be moderately structured since much of the knowledge now has a high certainty. However, it is in the solutions that the science becomes uncertain. Technology claims with confidence that is contested [refs to comparisons on NET and GGR papers laying out uncertainty of these]. The social science, in terms of the responses of individuals and society to impacts of the CAN crisis and to policies addressing it, is very uncertain. This is less likely to be expressed with certainty and does this thus explain the lower engagement of social scientists with policymaking?

Knowledge that is digestible policy - but also knowledge that is needed by policy. One of the major bottlenecks is surely around society and behaviour - these are the questions that still need answering. Therefore, presenting knowledge on social and behavioural consequences of current and potential policy. Such as what messages are out current emphasis of technological solutions (EVs, CSS, SAFs) implying for individuals and organisations.

\subsection{Academic impact on policy}
One of the main means of incentivising and rewarding the engagement of academia with [public decision making] has been to include policy impact in the REF, particularly the most recent REF [ref]. [very short summary of how impact case studies are selected by institutions]. An observation that came up several times was that the work that the participants had been involved in was selected for inclusion in their institution's REF submissions but they had never designed the work to have such an outcome. Moreover, there was [often/always] [an expression of surprise] that their work had ended up in the direction of impacting policy. For instance, In one case the work was actually designed to inform citizens. 

REF ``builds largely on linear models of the policy process'' \cite{CairneyO2020}

The serendipity and chance nature of policy engagement means that the highest impact research may never be designed to have such impact. And yet the nature of incentivisations such as REF and [UKRI paths to impact] may result in rather na\"ive attempts to impact policy which could be a distraction and hindrance rather than a benefit. [More scientists sending their outputs into policy individually] [needs to be coordinated?]

\subsection{what works}
There are some activities that seem to be more successful, at least the few participants who were using those strategies seemed to rate highly their value. 

It is tempting to dismiss the myriad other strategies that scientists mentioned in the course of this research - e.g.s - which had inconclusive outcomes. However, it is also possible that the diversity of strategies has been valuable, creating a range of events, venues and media through which the richness of the science is conveyed. Whilst the more targetted strategies ultimately show signs of [hitting home], this may be due to these previous ``pre-softening'' activities (\cite{Cairney2018}).

Forget the deficit model, build networks and relationships across divides, collaborate on topics, create meaning, \tquote{we have to write in to our bids, given that we know that information by itself is going to be limited, it's going to be relational, it's going to be messy, it's going to be long term, all these other things we're going to create mechanisms that hopefully align with that as far as we can. So we'll have placements and we will have stakeholder coalition workshops where we bring together from lots of different departments and sectors and things and try and break down silos and we'll have some training and master classes so that we're actually building capacity and all these things. This isn't just a tick box, we didn't just have journal papers, we also had a couple of briefing papers, a whole other gigantic set of activities to build capacity and relationships}{p05}{143}

but as \textcite{CairneyO2020} observe, success has more to do with context and entrepreneurship

By nature, participants are people who respond to requests to provide information

Passion for their subject evident across participants

Keen for policy to use best knowledge and feel they have that knowledge

Evidence of awareness of policy cycle? P5

Comments on political context and/or change of government

Involvement of other organisations - industry, NGOs, civil society, citizens?
\subsection{Measures of success}
Participants had been selected based on some external statement of ``success'' of their engagement, such as the engagement becoming a REF impact case study, being reported in a blog post or mentioned by a contact. However, the meaning of success emerged, and more latterly was enquired about within the interviewing process (with questions similar to ``what would you like to have seen arise from [that action]?'').

Being more readily rewarded by the process, rather than seeking an outcome, may feel more successful (NOT! Section~\ref{sec:resskifram} also p08 v p03). In complex issues, outcomes are unpredictable and emergent (maybe \cite{SnowdenB2007} and it is well know that it is difficult to trace input to outcomes \cite{BednarekSHG2015}, perhaps something in the public participation literature on this e.g. \cite{Sprain2016}) and so 

Those who are less concerned about standard academic measures of success such as publications and REF impact case studies are possibly able to draw on other means of demonstrating success - those working in technology for instance are able to demonstrate prototypes p01. Several participants [p09, p13] who work closely with policy identified that they derived a sense of success from knowing that good science had been made available to decision makers within the policymaking process.

Those who tended to indicate more frustration, and less of a feeling of being successful [p03, p11], were [perhaps identifying frustrations more with the system of decisionmaking]. Indeed, participants what were experiencing success, were working, often very strategically, within the existing system. Their successes were in relation to an existing system, and even those working towards changing that system (such as by proposing the economic foundations of decisionmaking) were careful to work within current implicit and explicit rules. This is a pragmatic approach, not only from the perspective that some change that benefits CAN is better than none, but also for maintaining psychological resilience. Yet, [difficult to ignore the perspective of those frustrated by the marginalisation of more radical perspectives on the evidence that is used and how that evidence is ingested into decisionmaking]  

And yet, means to measure the impact of science, such as the REF, rather assume the linear / deficit models are dominant [ref]. 
\subsection{What place more diverse knowledge?}

Knowledge is an institution [Leviathan and the Air Pump], most of us are unaware of the particular perspective our knowledge creates, and this idea wasn't generally picked up in the interviews

\cite{Bendell2024} most scientists are limited by their own privilege, institutional contexts, and position ``within the system'', and thus fail to advocate for the changes that are required for truly just transition  
\cite{StoddardEtAl2021} problems arise due to powerful interests and resulting institutions leading to a pervasive failure to [question many of the core tenets of modern, industrialised societies]. Whether or not this is true, the influence of vested interests, alongside lack of political will, it is given by climate scientists as the main reasons for inadequate action on climate change \cite{Carrington2024} 
\cite{TurnhoutMWKL2020} - power and politics in shaping processes and outcomes
