\chapter{Discussion}\label{ch:discussion}

Should scientists be involved at the interface?
Clearly desire for decision makers to speak to experts given the range opportunities highlighted in this short study. There is perhaps little to [beat] the  authority and credibility of knowledge gained directly from scientists who are deeply immersed in their topic. Further, a number of participants expressed their enjoyment in the engagement, and their desire to do more. Thus there is a good justification for drawing together knowledge from scientists about how to best perform in this undertaking.

\section{behaviours, strategies and roles}
To effectively communicate knowledge into policy appears to require a great deal of effort, as as \textcite{BednarekSHG2015} demonstrate, this may be too much for any single scientist without the support of a dedicated knowledge brokerage organisation. ``connecting science and policy is not a part-time, `do it yourself' enterprise, but instead benefits from the skills and experience of practitioners who are immersed in the process''

If scientists are PE what does this mean for the objectivity of science?
If they are not, what does this mean when science knowledge is up against the influences of PEs?

However, frustrations were expressed by scientists that they didn't know if their efforts had any impact. One scientist even commented ``they already know all of this, I don't know what more we can tell them''. There is a psychological safety issue here, perhaps reflected in the comments of another about thinking they should perhaps had turned to activism. We should be concerned that some scientists - people who daily have to face the harsh realities of CAN science - experience the interface with policy [so brutally].

It may not be appropriate for scientists to be advocating directly, but their organisations should be advocating much more on their behalves.

\section{case classifications}
\subsection{types of science}
Differences between physical, technology and social science contexts?
Considering WesselinkH2020 typology of problems, at first glance climate change appears to be moderately structured since much of the knowledge now has a high certainty. However, it is in the solutions that the science becomes uncertain. Technology claims with confidence that is contested [refs to comparisons on NET and GGR papers laying out uncertainty of these]. The social science, in terms of the responses of individuals and society to impacts of the CAN crisis and to policies addressing it, is very uncertain. This is less likely to be expressed with certainty and does this thus explain the lower engagement of social scientists with policymaking?

Knowledge that is digestible policy - but also knowledge that is needed by policy. One of the major bottlenecks is surely around society and behaviour - these are the questions that still need answering. Therefore, presenting knowledge on social and behavioural consequences of current and potential policy. Such as what messages are out current emphasis of technological solutions (EVs, CSS, SAFs) implying for individuals and organisations.

\section{network effects}
It's who you know, network, hob nob

\improvement{how to engage when science is not a chosen topic?}

\section{systemic issues}
\subsection{the science system}
\cite{Bendell2024} most scientists are limited by their own privilege, institutional contexts, and position ``within the system'', and thus fail to advocate for the changes that are required for truly just transition  
\cite{StoddardEtAl2021} problems arise due to powerful interests and resulting institutions leading to a pervasive failure to [question many of the core tenets of modern, industrialised societies]. Whether or not this is true, the influence of vested interests, alongside lack of political will, it is given by climate scientists as the main reasons for inadequate action on climate change \cite{Carrington2024} 
\cite{TurnhoutMWKL2020} - power and politics in shaping processes and outcomes

Scientists in more privileged fields should consider advocating for those in less privileged fields. This is analogous to advocacy in other contexts - the knowledge of those who tend to be marginalised by the current system is essential to [overthrow] that system.

This is a wider unspoken matter here. Whilst social identity was not part of this study, it was plain that [the majority/all] of the scientists who participated presented as racially white. Whilst this may be a artefact of the selection process (as previously stated, I used REF and online reports of scientists engaging with policymakers, followed by snowball identification of further participants)... privileges that lead to this role ... also play in policy. The significance here returns to the point about the bottlenecks in CAN science policy, which are largely societal. Therefore, why is society not represented at the interface tasked with overcoming these issues? [science and policy needs to take a long hard look at itself]. Where science may feel unheard, this is nothing to how the experience of others is ignored \cite{IbarraJOBCIMRS2022}. Issue of representation and power in who decides what is researched and how it is applied \cite{McNiePS2017}.

Why are scientists invited to present their knowledge to policymakers? Are scientists presenting knowledge that is not already available to them? How much do they act on the knowledge? How much more knowledge is required before policy catches up with the science?

I'm sure the scientists who expressed frustrations would agree that these frustrations are relative, and as nothing to the frustrations of scientists in less privileged settings or indeed policymakers constrained by [lack of fiscal flexibility e.g. subservient to dollar]

\information{gap is not just in policy, for CAN it is also between research and education \cite{DykeM2024} ... gap between the ``production and use of scientific information'' (Kirchhoff et al., 2013, p. 407; see Sarewitz and Pielke, 2007) \cite{McNiePS2017}}
\information{my positionality and privilege}

\section{what works}
There are some activities that seem to be more successful, at least the few participants who were using those strategies seemed to rate highly their value. 

It is tempting to dismiss the myriad other strategies that scientists mentioned in the course of this research - e.g.s - which had inconclusive outcomes. However, it is also possible that the diversity of strategies has been valuable, creating a range of events, venues and media through which the richness of the science is conveyed. Whilst the more targetted strategies ultimately show signs of [hitting home], this may be due to these previous ``pre-softening'' activities (\cite{Cairney2018}).

\section{limitations}
time and resource constraints. other studies, e.g. \cite{HaynesDCRHGS2011,OjanenBKP2021} have more researchers and thus possibility to confer over coding as well as time for more and longer in depth interviews

Useful approach to selection influential scientists used by \textcite{HaynesDCRHGS2011} by nomination - not enough time to do this in the present study but [would perhaps create proxy for trying to determine what is success]

\section{Recommendations}\improvement{Maybe this should be a chapter}
One scientist mentioned that high turnover of policymakers, that some scientists may be frustrated by having to convey the same information repeatedly. Perhaps there's a level of knowledge that needs to be conveyed more widely, to account for current and future policymakers as well as those whom they represent and would hold them to account. Civic education and education to political science etc students (\cite{DykeM2024}) 

creating communities that bring together different scientific perspectives ... even other disciplines, multidisciplinarity, which aligns with Gibbons et al's Mode 2 knowledge production, perhaps also wider perspectives, post-normal science \cite{FuntowiczR1993}, extended peer communities, \cite{Jasanoff2003} recommending and advocating for? Transdisciplinarity even (\textcite{RussellWC2008} discusses constraints that current institutions create)

reframe the science-policy interface - this looks good: \url{https://agupubs.onlinelibrary.wiley.com/doi/10.1029/2020EF001628}

be prepared for interactions - e.g. elevator pitch: \url{https://academic.oup.com/aesa/article/112/2/75/5363856?login=true}

\subsection{to institutions}
incentives are not fully there for frank policy engagement - e.g. \cite{ElsensohnACDGGKPRS2019} ``biases and limitations at scientific institutions, including but not limited to, a lack of incentive structures, institutional guidelines, and employment limitations'' - example of programme of training in science advice, maybe advocacy \cite{RussellWC2008}