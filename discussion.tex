\chapter{Discussion}\label{ch:discussion}

If scientists are PE what does this mean for the objectivity of science?
If they are not, what does this mean when science knowledge is up against the influences of PEs?

Differences between physical, technology and social science contexts?

It's who you know, network, hob nob

creating communities that bring together different scientific perspectives, also wider perspectives, post-normal science \cite{FuntowiczR1993}, extended peer communities, \cite{Jasanoff2003} recommending and advocating for?

It's not all in our hands
\cite{Bendell2024} most scientists are limited by their own privilege, institutional contexts, and position ``within the system'', and thus fail to advocate for the changes that are required for truly just transition  
\cite{StoddardEtAl2021} problems arise due to powerful interests and resulting institutions leading to a pervasive failure to [question many of the core tenets of modern, industrialised societies]. Whether or not this is true, the influence of vested interests, alongside lack of political will, it is given by climate scientists as the main reasons for inadequate action on climate change \cite{Carrington2024} 
\cite{TurnhoutMWKL2020} - power and politics in shaping processes and outcomes

