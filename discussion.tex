\chapter{Discussion}\label{ch:discussion}

%Scientists are using a wide variety of strategies at the CAN science-policy interface, many relate directly to influences that they are aware of or are suspecting. Given the assumption that the frequency of mentions of an influence relates\unsure{I am not very confident about this assumption} to the degree of its influence over scientists' policy engagement, the ISM can suggest strategies to focus on or develop.
%\tquote{urgent crisis}{p14}{61}
%\tquote{a theme of such importance}{p11}{50}
%\tquote{because you know what needs to be done and yet we keep getting told to go back and do more research on it}{p11}{39}\tquote{It's a displacement activity, you get researchers to go back}{p11}{40}\tquote{We know enough about what's going to happen. We know enough already.}{p11}{42}\tquote{and I feel governments have enough information on which to act already and have quite sufficient evidence. So I find it frustrating [that] what they've already got in their knowledge hasn't really caused massive u-turns}{p04}{48}

\section{Scientists' roles}\label{sec:disroles}

Participants within this study experienced and observed scientists' roles that parallel those defined within the literature (Section~\ref{sec:litroles}). However, in contrast to the implication of a one-way transfer of knowledge, it was clear within the interviews that knowledge and insight goes in many directions. Participants described gaining and using awareness of policy initiatives and processes, discussing and collaborating on policy, and even setting agendas, disseminating policy and managing the consequences of policy (e.g. Section~\ref{sec:resskipoli}). This corresponds to observations about the \SPI{} that it comprises complex and dynamic interactions and flows of information (Section~\ref{sec:litspi}). It further suggests that normative instructions to scientists to align with the needs of policy may be somewhat challenging given the complex and dynamic reality. Further, a number of tensions were identified within these roles, as described in the following sections. 

\subsection{Pure Scientists}\label{sec:disscientist}
A distinction was made by several participants between ``fundamental'' science and ``strategic'' science, the latter also being referred to as ``impact work''. Several participants' engagement with policy arose as a consequence of producing fundamental science and there was a firm belief expressed by some participants that fundamental science was necessary for policy impact, \tquote{This stuff's important and I always felt that I wanted a job [that] makes a difference, you're not just finding out new knowledge, which is what's fundamental science is and that's important, but actually this is an urgent crisis and we need to know that there are good people providing good evidence into government}{p14}{61}. Here, then, is evidence of the necessity of Pielke's \emph{Pure Scientist} role. Another motivation for being in this role was a hint that it is more esteemed than the other roles, as \tref{one participant reflected that scientists in their organisation were not interested in an opportunity to undertake a role as \emph{Science Adviser}}{p06}. 

Several participants commented that certain scientific disciplines seem more highly regarded than others, noting that technology tended to be most favoured by government. Participants who had been involved in research related to the development of technologies discussed the effort required to generate engagements much less. Biophysical measurement, monitoring and modelling research also seemed to ease engagement, such as where it could be used to demonstrate government investment. In this case, there was a theme across several participants that the interest was for more scientific research when they felt existing findings are already sufficient, \tquote{it's not clear to me, as a scientist, what other evidence is really necessary from that point of view}{p04}{49} (e.g. Section~\ref{sec:resskifram}). Participants undertaking behavioural and social sciences discussed their efforts to engage with policy a great deal. In fact, mentions of ``fundamental'' science were always related to the biophysical sciences. As one social scientist observed about their work, \tquote{whilst there are some interesting theories and ideas, it was always very applied to me}{p03}{14}. 

Thus, even this most basic \SPI{} role for scientists encounters tensions because of a ranking of scientific disciplines, it being unclear if social sciences can indeed be ``fundamental'', as well as suggestions that pure science may be being used as a \tquote{displacement activity}{p11}{40} from political action.

\subsection{Advisers and Advocates}\label{sec:disadvocacy}

As expressed in the literature, participants identified tensions between the roles of \emph{Science Adviser} (or \emph{Knowledge Broker}) and \emph{Advocate}, e.g. \tquote{With science policy, there's always this question and clarifications that one needs to make between advice and advocacy, and advocacy and then campaigning}{p09}{101}. The conflict was often regarding the impact of advocacy on the credibility of the scientist (\tquote{There's some climate scientists that are very opinionated ... they are very highly regarded, by some ... but most of the time, that's not by people who actually are involved in policy making}{p09}{115}) 
and, by implication, the evidence they conveyed (\tquote{the science evidence very rapidly gets tainted if you try and use the science to argue for a particular line of action}{p13}{40}). However, other participants endorsed advocacy (\tquote{The fact that you're choosing to research this means you think this is good and valuable, and so why won't you advocate for it in a clear way?}{p06}{88}), identified it as a necessary evil (\tquote{you really have to invest in, essentially, lobbying, tragically}{p08}{65}) because \tquote{ministers are getting lobbied from everybody else, we're not lobbying them}{p06}{91}. A couple of participants identified an unwanted pressure to advocate from people in their network who are not as close to the interface (\tquote{there's a lot of frustration in the industry sector and they want our project to be saying ``you need this policy ... you should be giving this money because of the biodiversity benefits''}{p14}{65}).

This issue is nuanced. On the one hand, when policymakers are well aware of an issue, indeed are asking questions, advocating a related cause or course of action may undermine a scientists' credibility. However, those pushing for more advocacy were often describing situations when the cause for which they wished to advocate was not on the policy agenda, such as behaviour change in \CAN{} policy. Also, a few participants noted that funding of research is increasingly tied to demonstrating impact, which limits the ability of scientists to research topics that are not on the policy agenda. In such cases, advocacy may be necessary to draw policy attention to wider issues. This does not preclude possible impacts to a scientists' credibility but for some there will be a moral imperative to draw attention to the knowledge they create, even if the benefits of credibility are only bestowed on those who come later. 

%\tquote{I wrote the IPCC chapters, but yet other young people were going out and getting themselves arrested}{p06}{94} \tquote{I don't disagree in a certain degree that if we can keep science neutral, we can keep evidence-informed policy making. I see the balance and the need for it and the neutrality of science. But it's weird, it's a tricky line}{p06}{95}

\subsection{Citizens and Scientists}

A few participants identified themselves as \emph{Citizen}s in addition to their professional role. This tended to be in relation to feelings about how \CAN{} issues are being addressed by policy. Those participants who appeared to feel agency within their role, were expressing frustration with policy as a citizen: \tquote{I think [I] probably feel more [disappointed by the UK policy system] as a citizen or a taxpayer ... than as a researcher}{p08}{32}. Such feelings could contrast with how they felt within their professional role: \tquote{whereas of course, as a scientist you can again nicely study why that is the case and therefore the scientific frustration is low but the societal frustration high}{p09}{59}. Others, who indicated feeling responsible (\tquote{young people were going out and getting themselves arrested ... because of what I wrote}{p06}{94}) but with little agency within their professional roles, described becoming involved with campaigning organisations, indicating that they sought to influence policy via this less direct route.

\subsection{Roles are a complex and dynamic reality}

Much as the \SPI{} is, in reality, complex and dynamic, so are the roles that scientists are taking on. Several participants described how they have pivoted in their role, particularly in response to what they felt was as an imperative given the challenges exposed by \CAN{} sciences. %p08, p07, ?p13
One participant expressed how the knowledge of scientists is inadequate at the \CAN{} \SPI: \tquote{you need the boundary people more than ever to translate the work for the policymakers because academics aren't going to be able to do it. It's not our job and we don't understand it. We don't understand government, so we write these stupid policy briefs with no target person, there's no target department, there's no target policy process, there's no window of opportunity for change, no levers or anything that they're targeting, it's just here's some information and that doesn't help}{p06}{103}. Many of these tensions are specific to, or intensified by the \CAN{} crises, and \CAN{} scientists many be more conflicted than scientists in other disciplines by trying to balance scientific principles of objectivity with their deep concern for the natural world.

\section{Scientists' practices}\label{sec:disstrategies}

Many of the practices identified by participants fit into the 8 tips described in \textcite{OliverC2019} (Table~\ref{tab:litpractices}). There were a few innovative practices that participants mentioned that appear to fall outside of these precepts (except, perhaps, ``be entrepreneurial''). The first was to challenge the perspectives of policymakers including by challenging traditional economic approaches to decision-making (Section~\ref{sec:resskipers}), using physical or behavioural science when framing issues (Section~\ref{sec:resskifram}) or by applying existing components of the policy landscape to scientific research (Section~\ref{sec:resskipoli}). By challenging policymaking, the practices add to the dynamism of the \SPI. The second practice was to demand recognition (Section~\ref{sec:resskiinst}). This reflects the well-established practice within science of citation, which not only ensures proper credit for work but also establishes the provenance and rigour of an idea.

Other innovative practices mentioned by participants take the published advice a little further. These include creating diverse engagements that such as dialogue and collaboration beyond policy to industry and citizens (Sections~\ref{sec:resskiagen},~\ref{sec:resskipers} and~\ref{sec:resskitech}), collaborating with policymakers on a particular topic or at a particular venue (Sections~\ref{sec:resskinetw} and~\ref{sec:resskiknow}), even establishing joint resources (Section~\ref{sec:resskiinfr}). Another practice is to interpret scientific findings for specific contexts. This is a recognised part of the role of a \emph{Knowledge Broker} but even participants based in academia were using knowledge about the policy landscape to research and communicate their science to contextualised to the remits of different ministries, governments, economic sectors or public bodies: \tquote{you've got to get to that level of granularity to be useful to them, you have to do several more steps to get actionable knowledge}{p08}{59} (Section~\ref{sec:resskitech} and~\ref{sec:resskifram}). These practices they suggest greater parity of science with policy than is suggested by much of the advice to scientists. 

Most of these innovative practices were only possible once relationships have been established at the \SPI{}. understandably, there was a strong contrast between those participants who were particularly frustrated by how difficult it was to build relationships with policymakers and those who found they were frequently being sought out by policymakers. One participant noted that this can vary over time: \tquote{don't take access or a listening ear for granted. Appreciate the opportunity every time that have been given it}{p09}{87} and another that the importance of trust during critical moments in the policymaking process results in policymakers calling on the same individuals: \tquote{the faster that the process operates, the more that particular actors don't have time to to go and look and open up a a catalogue of experts, or even to do stuff online. It's just literally \emph{we need something now} and [they] just go to the go-to people or the go-to providers of that [evidence/knowledge]}{p10}{60}. There is little comfort here for those scientists who struggle to engage and, given the observation (Section~\ref{sec:disscientist}) that some scientific disciplines struggle more to engage with policy, it also suggests that innovative practices are somewhat limited by discipline.

\subsection{Innovative practices need effort, enterprise and engagement}

The diverse strategic practices described by participants chronicled a great deal of effort and enterprise in researching and communicating at the \SPI. Some of the most innovative practices relied on having established, trusted relationships and demonstrated a determination of participants to find approaches to influencing policy in response to the complexity of the \SPI{} and line with the scale of the challenges of \CAN.
%To effectively communicate knowledge into policy appears to require a great deal of effort, as as \textcite{BednarekSHG2015} demonstrate, this may be too much for any single scientist without the support of a dedicated knowledge brokerage organisation. ``connecting science and policy is not a part-time, `do it yourself' enterprise, but instead benefits from the skills and experience of practitioners who are immersed in the process''

\section{Scientists' experiences}\label{sec:disexperience}

Participants described a wide variety of experiences of engagement, from more structured responses to calls for evidence and appearing before select committees, to less formal direct approaches by government staff and ministers, as well as venues created by non-policy actors such as debates and ``round tables''. All the participants spoke of a desire to engage with policy. 
The range of opportunities that they experienced speaks of a desire for policymakers to speak to experts%, and the authority and credibility of knowledge gained directly from scientists who are deeply immersed in their topic
. Several participants had experience of \SPI{} engagement from very early in their academic careers, sometimes unaware that it was unusual at the time. This led them into a career with a strong focus on engaging with policy. Nevertheless, a participant who had only limited experience engaging with policy was, as a consequence, keen to find more ways to work with policy and tended to think more about the relevance to policy of their work. Within the small cohort of this study, scientists were (or at least at some stage in their career, had been) keen to engage with policy. Several other themes emerged across the cohort related to the specifics of policymaking governance, network benefits, \CAN{} contentions, and scientists themselves, as described below. 

A very common experience across the interviews were the intricacies of the different components of national and international policymaking, particularly how unfathomable some aspects are (Section~\ref{sec:resskiinst}). From four different participants: \tquote{it's really hard to know about government from the outside. You can't really blame academics for not}{p06}{102}, \tquote{I don't know whether it's DECC or BEIS or DESNZ (I suspect it's the same group of people)}{p04}{18}, \tquote{it's about understanding how policymaking works (and no one really understands it) but understanding it a little bit}{p12}{82}, \tquote{even just looking at UK the patchwork of government departments and POST and the committees and the Lords and the Commons and everything, just trying to work out ... it is incredibly complex}{p05}{94}. Further, one theme stood out -- that national UK policy settings are the hardest to engage with, compared to local, devolved, or international settings.

The main means by which participants learned about the intricacies of policymaking was through established networks and relationships (Section~\ref{sec:resskinetw}) and all participants identified some aspect of their network and relationships that were essential to their capacity to engage with the \SPI. Being recommended for an advisory role by other members of their network, or recommending others themselves, were also common themes. Much as is described in the literature, scientists' networks and relationships are fundamental to the experiences engaging at the \SPI. Participants also turned to relationship building, particularly in terms of creating dialogues, to overcome negative influences of mismatched perspectives between scientists, policymakers, and the public  (Section~\ref{sec:resskipers}).

There were contentions specific to \CAN{} science and policy that all participants\footnote{The quotes in this paragraph are from 6 different participants} were cognisant of (Section~\ref{sec:resskifram}). They noted that \CAN{} science is more than the simplistic focus on carbon, trees or 1.5\degree C and they grappled with the consequences of these simplistic measures. For example: \tquote{the evidence coming back from my colleagues is that planting trees is not a panacea}{p11}{53}. Yet they found it difficult to convey this knowledge to policymakers because it is contrary to popular understandings of the science. Particular \skifram{} of \CAN{} create tensions:  \tquote{we're having some traction ... on the possible bad security implications or impacts ...}{p08}{37} yet \tquote{... the securitization of climate change ... was well meaning, but ... it gives you the wrong solutions if you think of it in a securitized way}{p05}{47}. %They considered this to be down to a \tquote{sleight of hand at the policy level internationally ... 30 years ago they went from saying `we've got to do something about fossil fuels' to `we've got to count carbon'}{p11}{56}. 
%Even fundamental climate and Earth system modelling is somewhat contentious in that it isn't providing immediate solutions. \tquote{It's getting more holistic ... at a stage where they are delivering results and hopefully contributing to understanding what actions we can take ... but it actually takes decades to develop these things and you can imagine that actually, if we just tweak it a bit more and then you need more funding. So it does feel like a continuum really ... you do sometimes feel like you're just finding problems}{p07}{18-19}. Another participant suspected that funding modelling may be popular because \tquote{it's quite a convenient thing because models look good and they provide data and all the other stuff [such as adaptation] is so messy}{p11}{43}.
And participants described \tquote{quite a strong debate}{p13}{43} within society about some negative emissions technologies: \tquote{people sometimes might say [a negative emissions technology] is a moral hazard because governments can say `oh, it's fine, we will have these magical greenhouse gas removals, we'll have these carbon hoovers ... we can carry on doing what we're doing'}{p01}{118} and \tquote{biomass is seen as competition for [food production]. The farming lobby and the food side of the farming lobby is very strong}{p14}{92}. \tquote{some people saying we just shouldn't be messing with the ocean for instance, we shouldn't be changing anything in the ocean that's environmentally not sensible}{p13}{43} but those researching negative emissions technologies felt \tquote{we need to be consuming less \emph{and} we need to also be sucking out the greenhouse gases that we've already emitted}{p01}{119}, noting that \tquote{people ... will say if we don't then the planet is going to get very warm and the environment will be damaged anyway}{p13}{43}. %Further, \tquote{nearly everything's controversial ... even wind energy ... things that seem pretty uncontroversial when you start actually trying to deliver them are controversial}{p13}{42}. This appreciation of the complexity was integrated into scientific research \tquote{We recognise [transition] is a new thing, there will be consequences of that, let's research that}{p01}{73}.

Perhaps it is also a feature of \CAN{} science-policy that it is difficult to identify impact, particularly when the ultimate measure of impact would be a reduction in the severity of the \CAN{} challenges (Section~\ref{sec:resskiinfr}). The motivation to have this ultimate impact led to bruising experiences for some participants. Two spoke of direct engagements within governments in which lack of necessary backing had risked their professional reputations. Another compared the experiences of a scientist trying to engage with policy with those of developing nations: \tquote{We are now the people [being] told: ``You need more training. You need to be entrepreneurs. You have to find out how to make money ... This is stuff we told people in developing countries to do for 30, 40, 50, 60, 70 years: ``what you were doing, your traditions and your way of doing is not good enough ... you are responsible for your plight''}{p11}{92}. These contrasted with the positive encounters of other participants, especially those who spoke of being focused more on the process than the outcome. Participants with expertise in working at the \SPI{} emphasised not only how important science is to decision-making, but also that particular skills are essential to convey that science: \tquote{it's really important that we have scientists in the system who understand that policy interface and can work at it}{p13}{75} but \tquote{the process of bringing science to policy is not typically something that every scientist is able to, or should be able to, in all honesty}{p09}{63}. 

As these experiences, and the short quantitative analysis (Section~\ref{sec:resroles}) demonstrated, whilst participants of this study had some common experiences, each participant's perspective was quite unique in terms of their scientific discipline, how engagement at the \SPI{} was initiated and evolved, as well as their motivations for, and reactions to, the engagement. This is not unexpected, even when limited to \CAN{}, the science and issues relevant at the \SPI{} are diverse. However, guides and literature on science-policy practice tend to treat scientists as a homogeneous group, to the extent that whilst policymakers are often defined, scientists are not (e.g. \cite{BA2024trust}). 

There is a further matter here that was not spoken about in the interviews. Whilst social identity was not sought as part of this study some basic inference of identity could be obtained from the interviews and, for instance, all of the scientists who participated presented as racially white. This may be a artefact of the small sample or selection process (Section~\ref{sec:metidentify}) but it speaks to the privileges that underlie roles within science and also within policy. The significance here relates to the contentions within \CAN{} science-policy, which as both identified within this study and more widely as \PNS, are largely about the societal impacts of \CAN{} challenges \emph{and} their potential solutions. As noted by several participants, the public need to be involved in the decisions that affect them, which hints at the \emph{extended peer review} indicated by \PNS{} (Section~\ref{sec:litspi}). However, if science and policy do not fully represent society, these tensions are not being fully represented at the \SPI{}. 

\section{Kicking the Information Deficit Model habit}\label{sec:disdeficit}

The literature, and this study, have demonstrated that rational model of policy is \tquote{dead and buried}{p03}{126} (if it ever was a living thing). There are also indications that there is not a deficit of information for \CAN{} policymaking. Yet, much of academia and beyond, continues to work on the \IDM{} premise that if enough of the right information is supplied, change will come. Naturally it was a behaviour science participant describing their own attempts to influence policy, who identified that \tquote{sometimes we have to remind ourselves that ``surely you're just converted by my two page briefing paper'' and oh! it doesn't work}{p05}{96}\footnote{analogously, participants did express frustration that some parts of government were still operating using related instruments, such as product labelling}. Thus it is reasonable to also accept that scientists should not \tquote{just lob their results at policy makers and expect them to have traction}{p13}{74}.

A parallel anecdote from this study is that one participant referred me several times to their (excellent) impact case study (\tquote{so it was interesting and you can read all about it in the impact case study}{p10}{38}). I had, in fact, read this case at least as deeply as I imagine a diligent policy official would read a relevant briefing note. However, the case study was not written for a student wanting to learn about the roles and experiences of scientists at the \CAN{} \SPI. Therefore, it did not provide the answers to my questions. The interview process enabled me to gain insights contextualised to this study, much as policymakers are seeking evidence contextualised to their policy remit. 

Moreover, using models from Behavioural Science, this study found a range of factors other than mere ``how to'' information that influenced how scientists engaged at the \SPI. Evidence in the literature that scientists do not always refer to theories and guidance about engaging with the policy process was born out by only a few references to these in the interviews. Instead of being influenced by such information, participants in this study spoke about their values and beliefs, how they perceived themselves and others, and how their particular science discipline affected their ability to engage. They identified the people, networks and techniques that eased their engagement. Equally, they experienced frustrations arising from lack of knowledge and lack of resources. Participants experienced positive and negative engagements as a result of particular institutional practices or framings of \CAN{} science and policy. They also devised and applied a range of practices to enhance the positive and counteract the negative influences on their engagements. Some of these practices parallel advice to science, other run counter to it. Consequently, this study not only demonstrates that there are many dimensions to engaging and influencing \CAN{} science policy, it presents a challenge about how best to support scientists to engage at the \SPI{} -- because providing another ``how to'' advisory note is clearly inadequate.


%As a scientist, I found this literature difficult to absorb, not only was it describing roles in a language that was not directly meaningful, it also implied the failure was wholly on the side of scientists. 

\section{Potential directions}\label{sec:disdirections}

There is a wealth of literature describing how scientists should behave at the \SPI{} and this study has identified that these roles and practices largely uphold \emph{in vivo}. Moreover, this study's descriptive-analytic approach, that was curious about scientists' real-world experiences, revealed tensions and innovative practices that don't feature in the current body of normative-prescriptive work. This reflects the complex and dynamic reality of the \SPI{} as well as the intimate personal involvement of scientists whose work relates to the \CAN{} crises. 

Another important message from this study is that simply providing ``how to'' advice to scientists is not sufficient to support better engagements at the \SPI. Instead, this section suggests some further directions for developing better experiences of the \SPI{} for scientists and policymakers. The potential directions suggested here are derived directly from the interviews in this study and are therefore credited to the anonymous participants.

\paragraph{Better allyship:}
\CAN{} science is far from a dispassionate enterprise, individuals are invested more than professionally, their \skirole{} can be both a scientist or an advisor, \emph{and} a concerned citizen in their workplace. Potential role conflicts has implications for the psychological safety of scientists. Further, the experience of scientists is not uniform, that some disciplines may find it more difficult to achieve impact, and even that society is not well represented at the \SPI{}. This suggests scientists working beyond \skinetw{}, even beyond the extended peer communities or citizens juries suggested by the literature (and participants), to building the allyship within and beyond science that embraces the diversity of knowledge, promotes underheard scientific and citizen perspectives, and supports individuals in their diverse experiences of \CAN-related decisions.

\paragraph{Policy citation:}
Whilst \emph{scientists doing science for scientists} may seem like a habit of science \skiinst{} that doesn't support policy engagement, the scientific discipline of citation benefits many actors across the science-policy interface. Firstly, it ensures that originators of evidence are easier to identify policymakers and scientists, who may identify as \emph{undervalued}. Secondly, it supports the credibility of policy by enabling the exposure of policy provenance. Thirdly, where citation becomes a cultural norm, it can prevent duplication of effort and build legacy even where \emph{staff turnover} is high.

\paragraph{Science training for policy:}
With a common frustration among many of the participants being the high turnover of policy staff and the need to ``reeducate'' each newcomer, scientists are identifying a need to increase the science understanding of current and future policy staff, with \CAN{} science encompassing some of the most critical disciplines. This is needed before people experience the frenzy of decision-making and perhaps could take the form of \CAN-specific civic education and even education of political science students (also \cite{DykeM2024}).

\paragraph{Policy training for scientists:}
Similarly, policy knowledge is increasingly being appreciated within the teaching of science disciplines. From this study, the more useful aspects to teach about policymaking would be the workings of different levels of policy and governance, concepts of policy theory, decisionmaking practices, and with practical assistance with network development, especially for those in disciplines that are less recognised in policy settings. 

\paragraph{Deepening our understanding of experiences:}
This study described and analysed scientists' roles and practices at the \SPI{} related to \CAN{} science, revealing a diversity of experiences. There was not the scope to prescribe ``what works'' in terms of roles and practices -- indeed there remain open questions about how to measure success -- although participants suggested themselves that some practices are more successful than others. To gain a better understanding of how scientists, and experts in other disciplines, can be influential at the \SPI, deeper research is needed to understand how practices and contexts influence policymaking.

\section{Limitations to this research}
The research for this Masters' thesis was undertaken by one student during the Summer of 2024. The method by which potential participants were identified was pragmatic, using online indicators of impact at the \SPI{} (Section~\ref{sec:metidentify}). Had there been time, a better approach would have been to use nomination to find suitable candidates, such as in \textcite{HaynesDCRHGS2011}. The timing of the study also limited the number of participants to the 14 who had capacity to be interviewed during a period when many academics are on annual leave. The quantitative analysis indicated that this may be too few participants to extract discernible clusters of experiences of scientists, if such clusters exist.

Labelling of transcripts was performed by one person, meaning there was no option to confer over coding. Repeated iterations of the labelling process were performed to effect some rigour to the process. These iterations used a range of frameworks, as described in Section~\ref{sec:metlabelling}. The final iteration derived a framework more suited to the \SPI. However, this framework needs further development. For instance, the \skiskil{} factor remained identical to the \ISM{} \ismis{} factor, yet it may be argued that whilst \emph{knowledge} fits comfortably within the \skiknow{}, \emph{skills} pertains more to \skiscip. Further, \skiemot{} was underexplored due to time constraints.

