\chapter{Discussion}\label{ch:discussion}

There is much to do here, but as I work through the results, I am weeving together findings into themes in this section.
%\section{Scientists strategies at the CAN science-policy interface}\label{sec:disroles}
%Scientists are using a wide variety of strategies at the CAN science-policy interface, many relate directly to influences that they are aware of or are suspecting. Given the assumption that the frequency of mentions of an influence relates\unsure{I am not very confident about this assumption} to the degree of its influence over scientists' policy engagement, the ISM can suggest strategies to focus on or develop.
%\subsection{Individual factors}\label{sec:disindividual}
%\subsection{Social factors}\label{sec:dissocial}
%\subsection{Material factors}\label{sec:dismaterial}

\section{Scientists' roles at the CAN science-policy interface}\label{sec:disroles}

\subsection{Knowledge conveyance}\label{sec:disknowrole}
The are a range of roles at the interface with policy that the literature suggests for scientists. May of these are presented from the standpoint of policy needing scientific evidence and thus the flow of information is from scientists into policy. This study, in fact, identified a range of knowledge transfer roles that scientists are playing:
\begin{itemize}
    \item awareness of policy initiative
    \item observing policy process
    \item briefed by policy
    \item reporting and demonstrating
    \item interpreting
    \item discussing
    \item advocating
    \item agenda setting
    \item disseminating policy
\end{itemize}

\subsection{Pure scientist}
Many participants recognised a difference between ``fundamental'' science and ``strategic'' science (also: ``impact work''). Several participants' engagement with policy arose as a consequence of producing fundamental science, specifically in biophysical fields. Until the moment of engagement, they fit Pielke's \emph{Pure Scientist} role.  One such scientist, who now has a great deal of experience at the interface with policy, expressed: \tquote{I believe, quite strongly, that this is one of the best ways that you can actually have long term impact on policy is by doing the the fundamental science that is going to be needed to inform long term policy decisions
}{p13}{18}. However, other participants revealed a frustration with policy's demand for ever more science or more evidence: \tquote{It's not clear to me, as a scientist, what other evidence is really necessary from that point of view}{p04}{49}.
%\tquote{because you know what needs to be done and yet we keep getting told to go back and do more research on it}{p11}{39}\tquote{It's a displacement activity, you get researchers to go back}{p11}{40}\tquote{We know enough about what's going to happen. We know enough already.}{p11}{42}\tquote{and I feel governments have enough information on which to act already and have quite sufficient evidence. So I find it frustrating [that] what they've already got in their knowledge hasn't really caused massive u-turns}{p04}{48}\tquote{}{p06}{94}
Curiously, mentions of fundamental science were always related to the biophysical. It is unclear if social sciences fall into this category, despite these being recognised as an essential component of CAN science.

Thus, even this most basic role for scientists is conflicted. Not only is there [contention] about the value of pure science in an \tquote{urgent crisis}{p14}{61}, but, at least in this limited study, it is not clear that all scientists can be pure scientists if this role does not include the social sciences. %\tquote{a theme of such importance}{p11}{50}

\tquote{This stuff's important and I always felt that I wanted a job [that] makes a difference, you're not just finding out new knowledge, which is what's fundamental science is and that's important, but actually this is an urgent crisis and we need to know that there are good people providing good evidence into government
}{p14}{61}

\subsection{The knotty problem of advocacy}\label{sec:disadvocacy}
Some of the strongest opinions expressed in the interviews related to whether or not scientists should advocate. Superficially, it appears that there is conflict between these different opinions. However, the issue is more nuanced. On the one hand, when policymakers are well aware of an issue, indeed are asking questions, it may not be expedient for scientists to be directly advocating a course of action, unless they can do so in strictly evidence-based terms %(a nice example can be found in \cite{RogeljLPLWXXXX}). 
. However, those pushing for more advocacy had a subtly different perspective. In these cases, the questions are not being asked, the issues or options are not on the policy agenda. Further, they noted that funding of research is increasingly tied to demonstrating impact, which limits the ability of scientists to research topics that are not on the policy agenda. In such cases, advocacy may be a reasonable role. Even so, it seems it is likely to be a thankless role and one that benefits only those who come later. 

\section{Scientists' experiences at the CAN science-policy interface}\label{sec:disexperience}

\subsection{Bruising experiences}\label{sec:disbruise}
However, frustrations were expressed by scientists that they didn't know if their efforts had any impact. One scientist even commented ``they already know all of this, I don't know what more we can tell them''. There is a psychological safety issue here, perhaps reflected in the comments of another about thinking they should perhaps had turned to activism. We should be concerned that some scientists - people who daily have to face the harsh realities of CAN science - experience the interface with policy [so brutally].

Another, frustrated at the implication that its always scientists who need to change themselves said \tquote{It's like [scientists have] been developed. We are now the people [being] told: ``You need more training. You need to be entrepreneurs. You have to find out how to make money. You've been under the illusion that what you've been doing for the last 30 years has been good or valuable or worthwhile. Change yourselves and everything will get better and your job will get better''. This is stuff we told people in developing countries to do for 30, 40, 50, 60, 70 years: ``what you were doing, your traditions and your way of doing is not good enough, you have to adopt our way of doing and you have to become entrepreneurs because you are responsible for your plight''}{p11}{}

Two participants spoke of experiences that were evidently still quite sore, with direct experience within governments in which they had ?staked their reputations? on a particular initiative which were then thwarted by lack of support from p06, p12 s89, s122, 

\section{Kicking the deficit model habit}\label{sec:disdeficit}
The linear model of policy is \tquote{dead and buried}{p03}{}, if it ever was a living thing. Yet, much of academia, and beyond, continues to work on the premise that if enough of the right information is supplied, change will come. Naturally it was a behaviour science participant who identified that \tquote{in the same way that it doesn't work with the public, it doesn't work with the policymakers - sometimes we have to remind ourselves that ``surely you're just converted by my two page briefing paper'' and oh! it doesn't work}{p05}{}\footnote{although participants did express frustration that some parts of government were still operating using related instruments, such as product labelling}. Thus it is reasonable to also accept that scientists should not \tquote{just lob their results at policy makers and expect them to have traction}{p13}{}

An interesting anecdote that parallels this [situation] is that one participant referred me several times to their (excellent) impact case study (\tquote{so it was interesting and you can read all about it in the impact case study}{p10}{}). I had, in fact, read this case at least as deeply as I imagine a diligent policy official would read a relevant briefing note. However, the case study was not written for a student wanting to learn about the roles and experiences of scientists at the CAN policy interface. Therefore, it did not provide the answers to my questions. The interview process enabled me to gain more relevant insights.

Instead, as has been found in this study, actions and behaviours are influenced by a wide range of factors. The scientists who participated in this study identified the influence of opinion leaders in half of the stories, and a range of other individual, social and material factors were also identified. These all indicate the kinds of strategies that may be used to engage with policy - many of which are already being used by scientists. Thus, this study not only demonstrates that there are many dimensions to engaging and influencing the CAN science-policy interface, it presents a challenge to the authors about how best to engage with science and scientists

\section{Limitations to this research}
time and resource constraints. other studies, e.g. \cite{HaynesDCRHGS2011,OjanenBKP2021} have more researchers and thus possibility to confer over coding as well as time for more and longer in depth interviews

Useful approach to selection influential scientists used by \textcite{HaynesDCRHGS2011} by nomination - not enough time to do this in the present study but [would perhaps create proxy for trying to determine what is success]

\section{Potential directions}\improvement{Maybe this should be a chapter}

\paragraph{Encourage citation:}
\tquote{making sure policymakers cite your research - that's probably something that we should tell people who are getting into this field}{p12}{}

Whilst ``doing science to be cited by scientists'' may seem like a \emph{habit} that doesn't support policy engagement, the scientific discipline of citation benefits many actors across the science-policy interface. Firstly, it ensures that originators of evidence are easier to identify by both policymakers and scientists (\emph{opinion leaders}). Secondly, it supports the credibility of policy by allowing the provenance of policy to be traced (\emph{values, beliefs, attitudes}). Thirdly, where citation becomes a cultural norm, it can prevent duplication of effort and build legacy even where staff turnover is high (\emph{times and schedules}).

\paragraph{Science outreach to potential policymakers:}
With a common frustration among many of the participants being the high turnover of policy staff and the need to ``reeducate'' each newcomer (\emph{skills}), a opportunity to increase the science understanding of future policy staff is to have more CAN-specific civic education and education of political science students (\cite{DykeM2024})

\paragraph{Networks and allyships:}

creating communities that bring together different scientific perspectives ... even other disciplines, multidisciplinarity, which aligns with Gibbons et al's Mode 2 knowledge production, perhaps also wider perspectives, post-normal science \cite{FuntowiczR1993}, extended peer communities, \cite{Jasanoff2003} recommending and advocating for? Transdisciplinarity even (\textcite{RussellWC2008} discusses constraints that current institutions create)

\paragraph{Trust building:}

Psychology finds that building the trust of citizens in decision making during peace time leads to a more effective response to a crisis \cite{BollykyP2024}. One participant, a [behaviour scientist], observed that what is true for citizens' behaviours is just as true for policymakers. Thus, it would be reasonable for science and scientists to build and maintain the trust of policymakers ... CRELE ... it is a long game.



\subsection{To institutions}
incentives are not fully there for frank policy engagement - e.g. \cite{ElsensohnACDGGKPRS2019} ``biases and limitations at scientific institutions, including but not limited to, a lack of incentive structures, institutional guidelines, and employment limitations'' - example of programme of training in science advice, maybe advocacy \cite{RussellWC2008}


\section{braindump}

reframe the science-policy interface - this looks good: \url{https://agupubs.onlinelibrary.wiley.com/doi/10.1029/2020EF001628}

be prepared for interactions - e.g. elevator pitch: \url{https://academic.oup.com/aesa/article/112/2/75/5363856?login=true}

patient science - good quality science is essential for policy and can take a long time to accumulate - requires long term investment (no expectation of returning quick impact). This may be difficult with [highly specified funding requirements]

\subsection{Should scientists be involved at the interface?}
Clearly desire for decision makers to speak to experts given the range opportunities highlighted in this short study. There is perhaps little to [beat] the  authority and credibility of knowledge gained directly from scientists who are deeply immersed in their topic. Further, a number of participants expressed their enjoyment in the engagement, and their desire to do more. Thus there is a good justification for drawing together knowledge from scientists about how to best perform in this undertaking.

But only some: \tquote{but I also believe that the process of bringing science to policy is not typically something that every scientist is able to, or should be able to, in all honesty
}{p09}{63}

\tquote{[some scientists are really brilliant at] the engagement side, public engagement and putting things very clearly, others are just like - I don't understand you and I'm in the project - those skills are really important as well. You can wheel out certain people, some people you wouldn't and that's why you have a team really, isn't it? So people have different skills, but you need balance.}{p07}{63}

\tquote{When you get into the system - and I've talked about how, for me, it's really important that we have scientists in the system who understand that policy interface and can work at it - one of the big key messages is make sure you know how to communicate to the people who make the decisions}{p13}{75}

\subsection{behaviours, strategies and roles}
To effectively communicate knowledge into policy appears to require a great deal of effort, as as \textcite{BednarekSHG2015} demonstrate, this may be too much for any single scientist without the support of a dedicated knowledge brokerage organisation. ``connecting science and policy is not a part-time, `do it yourself' enterprise, but instead benefits from the skills and experience of practitioners who are immersed in the process''

If scientists are PE what does this mean for the objectivity of science?
If they are not, what does this mean when science knowledge is up against the influences of PEs?


It may not be appropriate for scientists to be advocating directly, but their organisations should be advocating much more on their behalves.

\tquote{You need the boundary people more than ever to translate the work for the policymakers because academics aren't going to be able to do it. It's not our job and we don't understand it. We don't understand government, so we write these stupid policy briefs with no target person, there's no target department, there's no target policy process, there's no window of opportunity for change, no levers or anything that they're targeting, it's just here's some information and and that doesn't help}{p06}{}

\subsection{influence}
influence is difficult because
- \emph{} science is nuanced and complex
- \emph{topic} policymakers only interested in evidence that pertains to their current, often very specific, focus
- \emph{timing} windows of opportunity open and close quickly

\subsection{its who you know}
a frustration for some ...
participant identified how this is down to trust - as the pace as policymaking process increases, policymakers turn to those they know and trust: \tquote{the faster that the process operates, the more that particular actors don't have time to to go and look and open up a a catalogue of experts, or even to do stuff online. It's just literally \emph{we need something now} and [they] just go to the go-to people or the go-to providers of that [evidence/knowledge]}{p10}{}

\subsection{decision levels}
many authors would talk about ``levels of policy'' without [implication]. Some described specifically how they would engage more readily and successfully at a more local level ... one described in some depth the relationship (or lack of) between these levels in particular policy settings. They described how they'd been involved in policy at an international level and then later seen the consequences of international decision making in the field p11 double disconnect, street level bureaucrats

\subsection{a hierarchy of governance}
Nobody said that working with UK government was easy. In fact several participants commented that it was more difficult than local and devolved governments, European Commission and US [?Senate]. 

Actors within the policy system - \tquote{spent a lot of time talking to what at the time I thought were policy makers, but actually I guess most of them were rather unrepresentative elites}{p10}{}

\subsection{shrouded governance}
\tquote{it's really hard to know about government from the outside. You can't really blame academics for not}{p06}{}

Several participants identified that it was difficult to keep up with department names and remits \tquote{I don't know whether it's DECC or BEIS or DESNZ (I suspect it's the same group of people)}{p04}{}

\tquote{it's about understanding how policymaking works (and no one really understands it) but understanding it a little bit}{p12}{}

p05/94

\subsection{a hierarchy of science}
Several participants commented that certain fields seem more highly regarded than others. Technology tends to the most favoured by government, certainly technology research seemed in this small study to have easily engagements. Physical measurement modelling came next, mainly where it can be used to demonstrate investment. Finally behavioural and social science cam at the bottom of the pile, both when it came to engagement with policy, but also to some extent within the scientific community.

There was also a hint at those who are not practically researching were not so highly regarded - or at least a lack of understanding by colleagues why they may choose to focus more on policy engagement p06. This may be compounded by a distancing from the science leading to a shallowing of understanding of the more current detail (whether or not it is needed for policy engagement) p07. 

Reflected in the knowledge brokers who have influence in policy settings - e.g. RS over BA. Although this may also related to trust - that BA haven't that long-establed trusted relationship yet?

\subsection{its in the timing}
There was a contrast between the fast and the slow pace of policy development. Often comments about slow pace, which might align with the slow pace of scientific knowledge accumulation. However, when asked for evidence, scientists were often [side swiped] by how little time they had to provide it p04.  

\subsection{stay on topic}
Importance of staying on topic ... but also constrains the possibility of influencing with new perspectives and insights [Luke's Non-decision-making power]

Setting the agenda - topics that are meaningful - actionable - to policymakers. They know the issue but what are the steps they take to implement solutions? Granularity p08/59 and practically relevant p06/29


\subsection{speak when you're spoken to}
They've got be be asking, you can't foist p08, perfectly hone advice to meet the existing needs p09 and interests of policymakers against the inability to speak truth to power P11. Two sides of windows of opportunity?

relates to advice v advocacy debate - objective ``pros and cons '' advice is perhaps more relevant when policy is asking the question. In those cases when when an area of science is less well represented or understood - how to get the science represented? perhaps there is greater need for advocacy but only on certain topics? p12? v p13. Perhaps the advocates should be different to the advisers - similar to allyship approach - if you have a seat at the table, who can you invite to join you that wouldn't otherwise be invited? At the moment, this looks a little more like social and behavioural science needs some allyship from technology and biophysical scientists?

\improvement{how to engage when my science is not a chosen topic?}

\subsection{issue framing}
Relatedly, [under meaning factor] how the issue is framed has impact - security p08/37 v ?p05

\subsection{science versus evidence}
science versus evidence p09
evidence review p10, synthesising p01..., hub p06/3 p06/29




\subsection{pivoting}
In order to have greater influence on decisions regarding the science that they've observed?
p08, p07, ?p13

\subsection{hooked on engagement}
p06?, p09, p10 all experienced policy from very early in their academic career, expressing that they were unaware that it was unusual at the time. This led them into a career with a strong forus on engaging with policy. p04 - engaged with local policymakers early and continued to ... p02 only recently been engaged but now very interested in [doing more]

\subsection{bruising experiences}


\subsection{citizen engagement}
citizens as stakeholders and subjects of CAN policy decisions we're mentioned.... The adverse impacts of [the waterfall of policy] at the community level were described in detail by one p11 ... for this reason, others expressed a belief that people should be involved in changes that will affect them (such as through CAs) p03  p01 aware of impact of their technologies on people, and considering researching how to address this, as well as being very open and keen to demonstrate to anyone who is willing to ask

\subsection{intimate understanding}
of policymakers and policy contexts p01, ...
of people and society ..., of the science and projections of CAN for so many years [p02, ... , p08]

\subsection{types of science}
Differences between physical, technology and social science contexts?
Considering WesselinkH2020 typology of problems, at first glance climate change appears to be moderately structured since much of the knowledge now has a high certainty. However, it is in the solutions that the science becomes uncertain. Technology claims with confidence that is contested [refs to comparisons on NET and GGR papers laying out uncertainty of these]. The social science, in terms of the responses of individuals and society to impacts of the CAN crisis and to policies addressing it, is very uncertain. This is less likely to be expressed with certainty and does this thus explain the lower engagement of social scientists with policymaking?

Knowledge that is digestible policy - but also knowledge that is needed by policy. One of the major bottlenecks is surely around society and behaviour - these are the questions that still need answering. Therefore, presenting knowledge on social and behavioural consequences of current and potential policy. Such as what messages are out current emphasis of technological solutions (EVs, CSS, SAFs) implying for individuals and organisations.

\subsection{network effects}
It's who you know, network, hob nob

\subsection{making it happen}
How much are these all related?
prototyping
learning by doing
praxis
practice-based theorising

\subsection{the science system}
\cite{Bendell2024} most scientists are limited by their own privilege, institutional contexts, and position ``within the system'', and thus fail to advocate for the changes that are required for truly just transition  
\cite{StoddardEtAl2021} problems arise due to powerful interests and resulting institutions leading to a pervasive failure to [question many of the core tenets of modern, industrialised societies]. Whether or not this is true, the influence of vested interests, alongside lack of political will, it is given by climate scientists as the main reasons for inadequate action on climate change \cite{Carrington2024} 
\cite{TurnhoutMWKL2020} - power and politics in shaping processes and outcomes

Scientists in more privileged fields should consider advocating for those in less privileged fields. This is analogous to advocacy in other contexts - the knowledge of those who tend to be marginalised by the current system is essential to [overthrow] that system.

This is a wider unspoken matter here. Whilst social identity was not part of this study, it was plain that [the majority/all] of the scientists who participated presented as racially white. Whilst this may be a artefact of the selection process (as previously stated, I used REF and online reports of scientists engaging with policymakers, followed by snowball identification of further participants)... privileges that lead to this role ... also play in policy. The significance here returns to the point about the bottlenecks in CAN science policy, which are largely societal. Therefore, why is society not represented at the interface tasked with overcoming these issues? [science and policy needs to take a long hard look at itself]. Where science may feel unheard, this is nothing to how the experience of others is ignored \cite{IbarraJOBCIMRS2022}. Issue of representation and power in who decides what is researched and how it is applied \cite{McNiePS2017}.

Why are scientists invited to present their knowledge to policymakers? Are scientists presenting knowledge that is not already available to them? How much do they act on the knowledge? How much more knowledge is required before policy catches up with the science?

I'm sure the scientists who expressed frustrations would agree that these frustrations are relative, and as nothing to the frustrations of scientists in less privileged settings or indeed policymakers constrained by [lack of fiscal flexibility e.g. subservient to dollar]

\subsection{A measure of success}
Participants had been selected based on some external statement of ``success'' of their engagement, such as the engagement becoming a REF impact case study, being reported in a blog post or mentioned by a contact. However, the meaning of success emerged, and more latterly was enquired about within the interviewing process (with questions similar to ``what would you like to have seen arise from [that action]?'').

Being more readily rewarded by the process, rather than seeking an outcome, may feel more successful (p08 v p03). In complex issues, outcomes are unpredictable and emergent (maybe \cite{SnowdenB2007} and it is well know that it is difficult to trace input to outcomes \cite{BednarekSHG2015}, perhaps something in the public participation literature on this e.g. \cite{Sprain2016}) and so 

Those who are less concerned about standard academic measures of success such as publications and REF impact case studies are possibly able to draw on other means of demonstrating success - those working in technology for instance are able to demonstrate prototypes p01. Several participants [p09, p13] who work closely with policy identified that they derived a sense of success from knowing that good science had been made available to decision makers within the policymaking process.

Those who tended to indicate more frustration, and less of a feeling of being successful [p03, p11], were [perhaps identifying frustrations more with the system of decisionmaking]. Indeed, participants what were experiencing success, were working, often very strategically, within the existing system. Their successes were in relation to an existing system, and even those working towards changing that system (such as by proposing the economic foundations of decisionmaking) were careful to work within current implicit and explicit rules. This is a pragmatic approach, not only from the perspective that some change that benefits CAN is better than none, but also for maintaining psychological resilience. Yet, [difficult to ignore the perspective of those frustrated by the marginalisation of more radical perspectives on the evidence that is used and how that evidence is ingested into decisionmaking]  

\info{gap is not just in policy, for CAN it is also between research and education \cite{DykeM2024} ... gap between the ``production and use of scientific information'' (Kirchhoff et al., 2013, p. 407; see Sarewitz and Pielke, 2007) \cite{McNiePS2017}}
\info{my positionality and privilege}

And yet, means to measure the impact of science, such as the REF, rather assume the linear / deficit models are dominant [ref]. 



\subsection{Academic impact on policy}
One of the main means of incentivising and rewarding the engagement of academia with [public decision making] has been to include policy impact in the REF, particularly the most recent REF [ref]. [very short summary of how impact case studies are selected by institutions]. An observation that came up several times was that the work that the participants had been involved in was selected for inclusion in their institution's REF submissions but they had never designed the work to have such an outcome. Moreover, there was [often/always] [an expression of surprise] that their work had ended up in the direction of impacting policy. For instance, In one case the work was actually designed to inform citizens. 

REF ``builds largely on linear models of the policy process'' \cite{CairneyO2020}

The serendipity and chance nature of policy engagement means that the highest impact research may never be designed to have such impact. And yet the nature of incentivisations such as REF and [UKRI paths to impact] may result in rather naive attempts to impact policy which could be a distraction and hindrance rather than a benefit. [More scientists sending their outputs into policy individually] [needs to be coordinated?]

\subsection{what works}
There are some activities that seem to be more successful, at least the few participants who were using those strategies seemed to rate highly their value. 

It is tempting to dismiss the myriad other strategies that scientists mentioned in the course of this research - e.g.s - which had inconclusive outcomes. However, it is also possible that the diversity of strategies has been valuable, creating a range of events, venues and media through which the richness of the science is conveyed. Whilst the more targetted strategies ultimately show signs of [hitting home], this may be due to these previous ``pre-softening'' activities (\cite{Cairney2018}).

but as \textcite{CairneyO2020} observe, success has more to do with context and entrepreneurship
