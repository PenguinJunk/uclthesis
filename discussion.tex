\chapter{Discussion}\label{ch:discussion}

Should scientists be involved at the interface?
Clearly desire for decision makers to speak to experts given the range opportunities highlighted in this short study. There is perhaps little to [beat] the  authority and credibility of knowledge gained directly from scientists who are deeply immersed in their topic. Further, a number of participants expressed their enjoyment in the engagement, and their desire to do more. Thus there is a good justification for drawing together knowledge from scientists about how to best perform in this undertaking.

\section{roles}
If scientists are PE what does this mean for the objectivity of science?
If they are not, what does this mean when science knowledge is up against the influences of PEs?

\section{case classifications}
\subsection{types of science}
Differences between physical, technology and social science contexts?
Considering WesselinkH2020 typology of problems, at first glance climate change appears to be moderately structured since much of the knowledge now has a high certainty. However, it is in the solutions that the science becomes uncertain. Technology claims with confidence that is contested [refs to comparisons on NET and GGR papers laying out uncertainty of these]. The social science, in terms of the responses of individuals and society to impacts of the CAN crisis and to policies addressing it, is very uncertain. This is less likely to be expressed with certainty and does this thus explain the lower engagement of social scientists with policymaking?

Knowledge that is digestible policy - but also knowledge that is needed by policy. One of the major bottlenecks is surely around society and behaviour - these are the questions that still need answering. Therefore, presenting knowledge on social and behavioural consequences of current and potential policy. Such as what messages are out current emphasis of technological solutions (EVs, CSS, SAFs) implying for individuals and organisations.

\section{network effects}
It's who you know, network, hob nob

creating communities that bring together different scientific perspectives, also wider perspectives, post-normal science \cite{FuntowiczR1993}, extended peer communities, \cite{Jasanoff2003} recommending and advocating for?

\section{systemic issues}
\subsection{the science system}
\cite{Bendell2024} most scientists are limited by their own privilege, institutional contexts, and position ``within the system'', and thus fail to advocate for the changes that are required for truly just transition  
\cite{StoddardEtAl2021} problems arise due to powerful interests and resulting institutions leading to a pervasive failure to [question many of the core tenets of modern, industrialised societies]. Whether or not this is true, the influence of vested interests, alongside lack of political will, it is given by climate scientists as the main reasons for inadequate action on climate change \cite{Carrington2024} 
\cite{TurnhoutMWKL2020} - power and politics in shaping processes and outcomes

Scientists in more privileged fields should consider advocating for those in less privileged fields. This is analogous to advocacy in other contexts - the knowledge of those who tend to be marginalised by the current system is essential to [overthrow] that system.

This is a wider unspoken matter here. Whilst social identity was not part of this study, it was plain that [the majority/all] of the scientists who participated presented as racially white. Whilst this may be a artefact of the selection process (as previously stated, I used REF and online reports of scientists engaging with policymakers, followed by snowball identification of further participants)... privileges that lead to this role ... also play in policy. The significance here returns to the point about the bottlenecks in CAN science policy, which are largely societal. Therefore, why is society not represented at the interface tasked with overcoming these issues? [science and policy needs to take a long hard look at itself]

