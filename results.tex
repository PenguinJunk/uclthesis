\chapter{Results}\label{ch:results}
\section{Participant profiles}\label{sec:profiles}
These are provided to aid discussion and early analysis and probably won't be retained in the final version.

\subsection{Participant 1}\label{sec:p1}
[technology, academic] Scientist got into field out of very personal interest with no thought of climate relevance. Had been invited to engage in the past and this has developed over the years with ongoing conversations and understanding about what is of value to policymakers. Main driver seems to be to supply public value, in return for the long term funding and because this is the civic thing to do. Also, where previously they'd solely focussed on working with industry to have impact, they've realised that working with policy is as important. Continues to work in science but now with a clear view of how this can best serve policy. Sees others' frustrations with policy but does not find it frustrating because they have a strong insight into the processes involved in making policy.

\subsection{Participant 2}\label{sec:p2}
[measurement/modelling, academic] Scientist has been working the field for a couple of decades with no involvement in policy. Their field has recently become much higher on the policymaking agenda and was invited to present evidence to a committee. This came about due to connections to others who are more established in giving evidence. Process was enjoyable and they would be interested in being involved more. Now thinking about policy relevance of research including in publications as keen to add value to policy with their work.

\subsection{Participant 3}\label{sec:p3}
[social, academic] Scientist who is highly motivated to support transformational policy regarding climate. As such they have contributed in many ways to calls for information, directly their output in ways that they believe are palatable by policy. Yet their experience is largely frustrating because they have not seen a great deal of traction regarding the inclusion of their scientific evidence in policy making.

\subsection{Participant 4}\label{sec:p4}
[measurement/modelling, pubic servant] scientist, public servant, who has provided scientific information in response to requests. Enjoyed the recognition of some of those engagements but has also experienced the cultural gap between science and policy. Frustration arises from not understand what more information decision makers need to close the gap between science and policy.

\subsection{Participant 5}\label{sec:p5}
[social, academic] scientist with a very keen interest in influencing policy. Has consciously engaged decision makers since very early in research career and developed a range of strategies to attempt to influence policy. So much so that they feel they spend more energy on policy than science and can also feel uncomfortable with the ``sales'' nature of trying to influence policymakers. Has had successes particularly with local and devolved governments and so enjoyed being involved with committee for House of Lords.

\subsection{Participant 6}\label{sec:p6}
\subsection{Participant 7}\label{sec:p7}
\subsection{Participant 8}\label{sec:p8}

\section{General notes}
By nature, participants are people who respond to requests to provide information

%Passion for their subject evident across participants
%Keen for policy to use best knowledge and feel they have that knowledge
%Evidence of awareness of policy cycle? P5
%Comments on political context and/or change of government
%Involvement of other organisations - industry, NGOs, civil society, citizens?

%
